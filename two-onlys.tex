\section{DP-internal \textit{only} and adverbial \textit{only} \label{sec:two-onlys}}
So far in this paper I have exclusively discussed the so-called ``DP-internal'' usage of the word \textit{only}. There is also a more common usage which I will term ``adverbial \textit{only}'' for convenience.\footnote{Although \textit{only} in fact has a different distribution than other adverbs, for example: \begin{exe} \ex (Only/*Quickly) John finished the race. \ex John finished the race (*only/quickly). \end{exe} The exact syntactic status of this usage of \textit{only} is outside the scope of this paper.} An example of the adverbial usage is (\ref{adverb-only}).

\begin{exe}
	\ex \label{adverb-only} Only Omar takes the bus.
\end{exe}

DP-internal and adverbial \textit{only} share many properties, including the basic inference pattern of existence and uniqueness and the licensing of negative polarity, but also differ in some respects, notably in that adverbial \textit{only} more reliably induces an alternative set.

\subsection{Cross-linguistic perspective}
We have seen that DP-internal \textit{only} is translated as \textit{edinstvennyj} in Russian. Adverbial \textit{only} corresponds to a different lexical item, \textit{tol'ko}:

\begin{exe}
	\ex \gll Tol'ko studenty pri\v{s}li.\\
	only students came\\
	\glt `Only the students came.'
\end{exe}

The same lexical distinction is made in Spanish and German \citep{mcnally08} and Chinese (Shizhe Huang, p.c.). On the other hand, French uses the same word, \textit{seul}:\footnote{I thank Ma\"{e}lys G\"{u}ck for this data. French also has an adverb \textit{seulement}, morphologically derived from \textit{seul}, which corresponds to some usages of adverbial \textit{only} in English.}

\begin{exe}
	\ex \gll Seuls les \'{e}tudiants sont venus.\\
	Only the students are came\\
	\glt `Only the students came.'
	\ex \gll Julia a \'{e}crit le seul bon essai.\\
	Julia has written the only good essay\\
	\glt `Julia wrote the only good essay.'
\end{exe}

English is therefore not unique in having a single lexical item for DP-internal and adverbial \textit{only}, but the presence of a lexical distinction in a number of languages, including some that are genealogically unrelated, suggests that deeper differences in the meaning of the two usages may be present.

\subsection{Alternative sets and focus}
Alternative sets are a fundamental part of the semantics of adverbial \textit{only}. We have seen in section \ref{sec:no-anti-unique} that alternative sets may help explain some intriguing Russian data with DP-internal \textit{only}. However, they do not seem to be universally present with DP-internal \textit{only}. (\ref{raj-cookie}), for instance, does not involve any kind of alternative set. It is synonymous with the equivalent sentence without \textit{only}, except with an extra emphasis on the uniqueness of the cookie. But there is no implication of any alternatives to the cookie that Raj ate.

\begin{exe}
	\ex \label{raj-cookie} Raj ate the only cookie.
\end{exe}

\subsection{Negative polarity items and downward entailment}
Section \ref{sec:edinstvennyj} showed that DP-internal \textit{only}, like adverbial \textit{only}, licenses negative polarity items.

Negative polarity items are typically permitted only in downward entailment environments, which are environments that allow an inference from supersets to subsets. Sentential negation is the canonical example of a downward-entailing environment, as the entailment from the superset \textit{reptiles} in (\ref{reptiles}) to the subset \textit{snakes} in (\ref{snakes}) shows.

\begin{exe}
	\ex \label{reptiles} No reptiles give birth to live young.
	\ex \label{snakes} No snakes give birth to live young.
\end{exe}

The scope of adverbial \textit{only} is not a traditional downward entailment environment. (\ref{vegetables}) does not entail (\ref{broccoli}).\footnote{I thank Courtney Dalton for making this observation to me.}

\begin{exe}
	\ex \label{vegetables} Only John eats vegetables.
	\ex \label{broccoli} Only John eats broccoli.
\end{exe}

However, a weaker version of downward entailment called Strawson entailment does apply \citep{fintel99}. Strawson entailment carries the additional requirement that a sentence's presuppositions be satisfied. In (\ref{broccoli}), if the presupposition that John eats broccoli is satisfied, then (\ref{vegetables}) does in fact entail (\ref{broccoli}).

An identical pattern emerges with DP-internal \textit{only}. (\ref{ca-30}) entails (\ref{ca-35}) so long as the existence presupposition of \textit{only} is satisfied with regard to states with more than 30 million and 35 million inhabitants. (\ref{russian-book}) entails (\ref{gogol-book}) if \textit{book in the library written by Gogol} exists.

\begin{exe}
	\ex \label{ca-30} The only state with more than 30 million inhabitants is California.
	\ex \label{ca-35} The only state with more than 35 million inhabitants is California.
	\ex \label{russian-book} The only book in the library written by a Russian is already checked out.
	\ex \label{gogol-book} The only book in the library written by Gogol is already checked out.
\end{exe}

Both \textit{only}s share the same pattern of Strawson downward entailment in their licensing of negative polarity items.

% TODO: Conclude this section
