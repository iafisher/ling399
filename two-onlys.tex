\section{DP-internal \textit{only} and adverbial \textit{only} \label{sec:two-onlys}}
So far I have exclusively discussed the DP-internal usage of the word \textit{only}. There is also a usage which I will term ``adverbial \textit{only}'' for convenience.\footnote{Although \textit{only} has a different distribution than other adverbs, for example: \begin{exe} \ex (Only/??Quickly) John finished the race. \ex John finished the race (*only/quickly). \end{exe} The exact syntactic status of this usage of \textit{only} is not relevant to my argument.} An example is (\ref{adverb-only}).

\begin{exe}
	\ex \label{adverb-only} Only Omar takes the bus.
\end{exe}

DP-internal and adverbial \textit{only} share many properties, including the basic inference pattern of existence and uniqueness, association with focus, alternative sets and the licensing of negative polarity items. The purpose of this section is to argue that my analysis of DP-internal \textit{only} is consistent with a unified account of the two \textit{only}s, though the details of what a unified account would look like are beyond the scope of this paper.

\subsection{Compositional semantics}
(\ref{only-lf}), repeated below, is the logical form of DP-internal \textit{only} proposed by \citet{cb2015}.

\begin{exe}
	\exr{only-lf} \textit{only}: $ \lambda P . \lambda x . [ \partial(P(x)) \land \forall y [ x \ne y \to \neg P(y) ] ] $
\end{exe}

The formula for DP-internal \textit{only} in (\ref{only-lf}) also adequately captures the meaning of adverbial \textit{only} in (\ref{adverb-only}). Suppose the following translations:

\begin{exe}
	\ex \textit{Omar}: $o$
	\ex \textit{takes the bus}: $\lambda x . \textsc{Take-bus}(x)$
\end{exe}

If \textit{only} composes first with \textit{takes the bus} and then with \textit{Omar}, then the logical form of the entire sentence would be (\ref{omar-lf}).

\begin{exe}
	\ex \label{omar-lf} \textit{Only Omar takes the bus}: $\partial(\textsc{Take-bus}(o)) \land \forall y [ o \ne y \to \neg \textsc{Take-bus}(y) ]$
\end{exe}

(\ref{omar-lf}) works out to presuppose that Omar takes the bus and assert that no one else does, which is the correct meaning of the sentence. This division of presupposition and assertion is the standard analysis of simple sentences with \textit{only} \citep{horn69}, and it mirrors the existence presupposition-uniqueness assertion inference pattern of DP-internal \textit{only}.

The one caveat is that it would be more plausible syntactically for \textit{only} to compose first with \textit{Omar} and then with \textit{takes the bus}. Of course this could be accomplished by switching the order of the parameters $P$ and $x$ in (\ref{only-lf}), but then a similar but mirrored problem would arise with the DP-internal usage of \textit{only}, since it must combine with $P$ before $x$.

There may be technical issues (though seemingly tractable ones) in realizing a unified account of the two \textit{only}s in the framework of compositional semantics. Nonetheless, the difficulty resides mostly in the syntax, and the semantics of the two words are near-identical.

\subsection{Alternative sets and association with focus}
Alternative sets are a fundamental part of the semantics of adverbial \textit{only} \citep{rooth85, rooth92}. A sentence with adverbial \textit{only} like (\ref{minji}) asserts that no one in a set of contextually constrained alternatives to Min-ji goes to the movies (in addition to presupposing that Min-ji herself does go).

\begin{exe}
	\ex \label{minji} Only Min-ji goes to the movies.
\end{exe}

\textit{Only} NPs with restricting clauses or PPs also involve alternative sets. (\ref{rebecca-only-bus}), for instance, carries an implication of the existence of alternative buses to other destinations which (\ref{rebecca-bus}), lacking \textit{only}, does not.

\begin{exe}
	\ex \label{rebecca-only-bus} Rebecca took the only bus to Wichita.
	\ex \label{rebecca-bus} Rebecca took the bus to Wichita.
\end{exe}

When \textit{only} NPs lack a restricting clause, like in (\ref{raj-only-cookie}), then the presence of an alternative set becomes harder to diagnose. But again, the crucial difference between (\ref{raj-only-cookie}) and (\ref{raj-cookie}), its counterpart without \textit{only}, is that (\ref{raj-only-cookie}) carries a wider entailment of uniqueness (in the sense that (\ref{raj-cookie}) but not (\ref{raj-only-cookie}) would be licensed at an event with more than one cookie), and to a certain degree expresses the expectation that there would have been more than one cookie. It is with regard to these alternative, non-existent cookies that (\ref{raj-only-cookie}) expresses its meaning.

\begin{exe}
	\ex \label{raj-only-cookie} Raj ate the only cookie.
	\ex \label{raj-cookie} Raj ate the cookie.
\end{exe}

Besides inducing alternative sets, \citet{rooth85} showed that adverbial \textit{only} associates with focus. Consider (\ref{focused-bill}) and (\ref{focused-sue}), where capitalization marks semantic focus.

\begin{exe}
	\ex \label{focused-bill} Mary only introduced BILL to Sue.
	\ex \label{focused-sue} Mary only introduced Bill to SUE.
\end{exe}

In a situation where Mary introduced Bill and Dave to Sue, and Dave to Mary, (\ref{focused-bill}) would be false but (\ref{focused-sue}) would be true. But without \textit{only}, both sentences would be true. The crux of association with focus is that \textit{only} interacts with semantic focus in a manner that affects truth conditions.

DP-internal \textit{only} likewise associates with focus. If Mary has a red hatchback, Mike has a red pick-up truck, and Susan has a blue sedan, then (\ref{focused-hatchback}) would be true but (\ref{focused-red}) would be false insofar as it implies that other people also have hatchbacks. Again, removing \textit{only} renders both sentences true in the given context.

\begin{exe}
	\ex \label{focused-hatchback} Margaret is the only one with a red HATCHBACK.
	\ex \label{focused-red} Margaret is the only one with a RED hatchback.
\end{exe}

The evidence is fairly strong that DP-internal \textit{only} associates with focus and induces alternative sets, although the alternation in unrestricted \textit{only} NPs like (\ref{raj-only-cookie}) is more difficult to detect.

\subsection{Negative polarity items and downward entailment}
Section \ref{sec:edinstvennyj} showed that DP-internal \textit{only}, like adverbial \textit{only}, licenses negative polarity items. The scope of adverbial \textit{only} is not a traditional downward entailment environment, however. (\ref{vegetables}) does not entail (\ref{broccoli}).

\begin{exe}
	\ex \label{vegetables} Only John eats vegetables.
	\ex \label{broccoli} Only John eats broccoli.
\end{exe}

However, a weaker version of downward entailment called Strawson entailment does apply. Strawson entailment carries the additional requirement that a sentence's presuppositions be satisfied \citep{fintel99}. In (\ref{broccoli}), if the presupposition that John eats broccoli is satisfied, then (\ref{vegetables}) does entail (\ref{broccoli}).

An identical pattern emerges with DP-internal \textit{only}. (\ref{ca-30}) entails (\ref{ca-35}) so long as the existence presupposition of \textit{only} is satisfied with regard to states with more than 30 million and 35 million inhabitants. (\ref{russian-book}) entails (\ref{gogol-book}) if an entity satisfying the description \textit{book in the library written by Gogol} exists.

\begin{exe}
	\ex \label{ca-30} The only state with more than 30 million inhabitants is California.
	\ex \label{ca-35} The only state with more than 35 million inhabitants is California.
	\ex \label{russian-book} The only book in the library written by a Russian is already checked out.
	\ex \label{gogol-book} The only book in the library written by Gogol is already checked out.
\end{exe}

Both \textit{only}s are Strawsonian downward-entailing NPI licensers.

\subsection{Cross-linguistic perspective}
A different strategy for evaluating the plausibility of a unified account of the two usages of \textit{only} is to see whether other languages have them as the same lexical item.

DP-internal \textit{only} is translated as \textit{edinstvennyj} in Russian. Adverbial \textit{only} corresponds to a different lexical item, \textit{tol'ko}:

\begin{exe}
	\ex \gll Tol'ko studenty pri\v{s}li.\\
	only students came\\
	\glt `Only the students came.'
\end{exe}

The same lexical distinction between adverbial and DP-internal \textit{only} is made in Chinese (Shizhe Huang, p.c.), Spanish and German \citep{mcnally08}. On the other hand, French uses the same word, \textit{seul}:\footnote{I thank Ma\"{e}lys Gl\"{u}ck for this data. French also has an adverb \textit{seulement}, morphologically derived from \textit{seul}, which corresponds to some instances of adverbial \textit{only} in English.}

\begin{exe}
	\ex \gll Seuls les \'{e}tudiants sont venus.\\
	Only the students are came\\
	\glt `Only the students came.'
	\ex \gll Julia a \'{e}crit le seul bon essai.\\
	Julia has written the only good essay\\
	\glt `Julia wrote the only good essay.'
\end{exe}

The cross-linguistic evidence on the lexical identity of the two \textit{only}s is mixed and does not clearly support either a unified or a distinct account. However, the range of properties that DP-internal and adverbial \textit{only} share suggest that a unified account is feasible, and at any rate my analysis of DP-internal \textit{only} does not seem to be inconsistent with one.