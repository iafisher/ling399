\section{Indeterminate \textit{only} \label{sec:indeterminate-only}}
Recall from section \ref{sec:anti-uniqueness-english} that the absence of a definite article in Russian should allow indeterminate readings for DP-internal \textit{only} to be more widely available, because the principle of Maximize Presupposition associated with the weak uniqueness presupposition of the definite article in English does not apply to generally block indeterminate readings.

Russian lacks an indefinite article, but instances where English uses an indefinite article with the exclusive adjectives listed above generally cannot be translated with \textit{edinstvennyj} in Russian. In (\ref{sole-director}), for instance, the preferred translation of the English sentence with the indefinite phrase \textit{a sole director} uses the regular cardinal number \textit{odin} `one' rather than \textit{edinstvennyj}, which is marginal.

\begin{exe}
	\ex \label{sole-director} \gll U \`{e}toj kompanii --- (odin/??edinstvennyj) direktor.\\
	At this company {} (one/only) director\\
	\glt `This company has a sole director.'
\end{exe}

In (\ref{not-a-sole}), another example of indefinite \textit{sole} in English, the translation with \textit{edinstvennyj} is outright ungrammatical. \textit{Odin} must be used.

\begin{exe}
	\ex \label{not-a-sole} \gll Ni (odin/*edinstvennyj) \v{c}elovek ne pri\v{s}\"{e}l.\\
	Not one/only person not came\\
	\glt `Not a sole person came.'\footnote{Russian \textit{ni} is a negative concordance particle in this case, rather than double negation.}
\end{exe}

(\ref{sole-director}) and (\ref{not-a-sole}) indicate that indeterminate readings for \textit{edinstvennyj} are dispreferred if not outright impossible. There are at least two exceptions to this generalization, however. The first is that an indeterminate reading for \textit{edinstvennyj} can be achieved in the compound expression \textit{odin-edinstvennyj}:

\begin{exe}
	\ex \label{odin-edinstvennyj} \gll Vra\v{c}i rekomendovali odin-edinstvennyj podxod.\\
	doctors recommended one-only approach\\
	`The doctors recommended one single approach.'
\end{exe}

The second is that \textit{edinstvennyj} can combine with \textit{reb\"{e}nok} `child' to mean `an only child' (i.e., a child with no siblings):

\begin{exe}
	\ex \label{only-child-ru} Marija --- edinstvennyj reb\"{e}nok.\\
	Maria {} only child
	\glt `Maria is an only child.'\footnote{As is generally the case with bare nominals in Russian, \textit{edinstvennyj reb\"{e}nok} also has a determinate reading, meaning `Maria is the only child.'}
\end{exe}

It is consistent with the other evidence to conclude that the source of the indeterminate import of the NP in (\ref{odin-edinstvennyj}) is the numeral \textit{odin} rather than \textit{edinstvennyj}, so (\ref{odin-edinstvennyj}) is not a true counterexample.

(\ref{only-child-ru}) is more problematic, as there is no other candidate for licensing the indeterminate reading. However, it is also true that DP-internal \textit{only} cannot generally be indeterminate in English:

% TODO: Note though that 'edinstvennyj rebyonok' is only indeterminate insofar as other predicative DPs are indeterminate.

\begin{exe}
	\ex Examples (32)-(34) from \citet{cb2012a} \begin{xlist}
		\ex If the business is owned by a(n) sole/*only owner, only the owner is eligible to be the managing officer.
		\ex This company has a(n) sole/*only director.
		\ex There was a(n) sole/*only piece of cake left.
	\end{xlist}
\end{exe}

In English, \textit{only} can be indeterminate only when it combines with the noun \textit{child} (and derived nouns like \textit{grandchild}). (\ref{sole-director}) shows that it is the same case in Russian. Since DP-internal \textit{only} does not productively allow indeterminate readings, \textit{an only child} may be considered idiomatic in both languages and not indicative of the general properties of DP-internal \textit{only}.\footnote{It is nonetheless curious that the same idiom should surface in both languages. I have no comment on this coincidence at the moment.}

Thus, despite the two objections, the generalization remains that \textit{edinstvennyj} does not independently allow an indeterminate reading, contrary to the predictions that \citet{cb2015} make in their conclusion about languages lacking articles.