\section{Indeterminate \textit{only} \label{sec:indeterminate-only}}
Recall from section \ref{sec:anti-uniqueness-english} that the absence of a definite article in Russian should allow indeterminate readings for DP-internal \textit{only} to be more widely available, because the principle of Maximize Presupposition associated with the weak uniqueness presupposition of the definite article in English would not apply in Russian to block indeterminate readings.

The prediction is generally borne out in Russian. \textit{Edinstvennyj} can be used in place of the numeral \textit{odin} `one' in many (though not all) contexts, and in particular in context where \textit{only} would be disallowed in English and a different exclusive adjective would have to be used. (\ref{sole-director}) is one such context. (\ref{single-good-essay}) and (\ref{single-good-speech}) are others.

\begin{exe}
	\ex \label{sole-director} \gll U \`{e}toj kompanii --- (odin/edinstvennyj) direktor.\\
	At this company {} (one/only) director\\
	\glt `This company has a sole director.'

	\ex \label{single-good-essay} \gll Marija napisala edinstvennoe xoro\v{s}oe so\v{c}inenie za vsju \v{z}izn'.\\
	Maria wrote only good essay in entire life\\
	\glt `Maria wrote a single good essay in her entire life.'

	\ex \label{single-good-speech} \gll Boris ne proizn\"{e}s ni edinstvennoj xoro\v{s}oj re\v{c}i na svad'be.\\
	Boris not gave not only good speech at wedding\\
	\glt `Boris didn't give a single good speech at the wedding.'
\end{exe}

\textit{Edinstvennyj} can also be used indeterminately in the compound expression \textit{odin-edinstvennyj}:

\begin{exe}
	\ex \label{odin-edinstvennyj} \gll Vra\v{c}i rekomendovali odin-edinstvennyj podxod.\\
	doctors recommended one-only approach\\
	`The doctors recommended one single approach.'
\end{exe}

Finally, \textit{edinstvennyj} can combine with \textit{reb\"{e}nok} `child' to mean `an only child' (i.e., a child with no siblings):

\begin{exe}
	\ex \label{only-child-ru} Marija --- edinstvennyj reb\"{e}nok.\\
	Maria {} only child
	\glt `Maria is an only child.'\footnote{As is generally the case with bare nominals in Russian, \textit{edinstvennyj reb\"{e}nok} also has a determinate reading, meaning `Maria is the only child.'}
\end{exe}

In (\ref{not-a-sole}), however, another example of indefinite \textit{sole} in English, the translation with \textit{edinstvennyj} is ungrammatical. \textit{Odin} must be used.

\begin{exe}
	\ex \label{not-a-sole} \gll Ni (odin/*edinstvennyj) \v{c}elovek ne pri\v{s}\"{e}l.\\
	Not one/only person not came\\
	\glt `Not a sole person came.'\footnote{Russian \textit{ni} is a negative concordance particle in this case, rather than double negation.}
\end{exe}

In general, \textit{edinstvennyj}, unlike DP-internal \textit{only} in English, allows an indeterminate reading, in keeping with the predictions about languages lacking articles made in the conclusion of \citet{cb2015}.
