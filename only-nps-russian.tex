\section{\textit{Edinstvennyj} NPs in Russian \label{sec:only-nps-russian}}
Russian lacks articles and allows bare nominals in argument positions. \textit{Kniga} `book' in (\ref{kniga}), for instance, may be interpreted as either indeterminate or determinate depending on context.

\begin{exe}
	\ex \label{kniga} \gll Anna \v{c}itaet knigu.\\
	Anna reads book\\
	\glt `Anna is reading a book' or `Anna is reading the book.'
\end{exe}

The standard account of bare nominals in Russian is that they receive their determinate or indeterminate import by covert type-shifting \citep{chierchia98}, specifically with \citegen{partee86} \textsc{Iota} and \textsc{A} shifts for determinate and indeterminate nominals respectively. These type shifts are presumably embodied by phonetically null determiners in the syntax.

Unlike in English, \textit{edinstvennyj} NPs (the Russian equivalent of \textit{only} NPs) may be indeterminate even outside the scope of negation, as (\ref{sole-director}) and (\ref{single-good-essay}) show.

\begin{exe}
	\ex \label{sole-director} \gll U \`{e}toj kompanii --- (odin/edinstvennyj) direktor.\\
	At this company {} (one/only) director\\
	\glt `This company has a sole director.'
	
	\ex \label{single-good-essay} \gll Marija napisala edinstvennoe xoro\v{s}oe so\v{c}inenie za vsju \v{z}izn'.\\
	Maria wrote only good essay in entire life\\
	\glt `Maria wrote a single good essay in her entire life.'
	
	% TODO: Move this somewhere else or remove it entirely
	%\ex \label{single-good-speech} \gll Boris ne proizn\"{e}s ni edinstvennoj xoro\v{s}oj re\v{c}i na svad'be.\\
	%Boris not gave not only good speech at wedding\\
	%\glt `Boris didn't give a single good speech at the wedding.'
\end{exe}

The problem with \textit{only} NPs is that they do not meet the licensing conditions of the definite article. Since Russian has no definite article, indeterminate \textit{edinstvennyj} NPs are not surprising at all.

Nonetheless, I will show that there is a similar paradigm of forced determinate and indeterminate readings of \textit{edinstvennyj} NPs in Russian. Specifically, there is a pattern of examples where a \textit{edinstvennyj} NP in an affirmative sentence must be interpreted as determinate, while its counterpart in a negated sentence must be interpreted as indeterminate. (\ref{tolstoy}) and (\ref{tolstoy2}) are just such a pair.

\begin{exe}
	\ex \label{tolstoy} \gll Tolstoj --- edinstvennyj avtor \textit{Vojny i mira}.\\
	Tolstoy {} only author \textit{War and Peace}\\
	\glt `Tolstoy is the only author of \textit{War and Peace}.'

	\ex \label{tolstoy2} \gll Tolstoj ne edinstvennyj avtor \textit{Vojny i mira}.\\
	Tolstoy not only author \textit{War and Peace}\\
	\glt `Tolstoy is not the only author of \textit{War and Peace}.'
\end{exe}

In (\ref{tolstoy}), \textit{edinstvennyj avtor Vojny i mira} `only author of \textit{War and Peace}' must be determinate: the sentence cannot mean that Tolstoy is one of several only authors of \textit{War and Peace}. In (\ref{tolstoy2}), \textit{edinstvennyj avtor Vojny i mira} must be indeterminate, because the sentence entails the existence of more than one author of \textit{War and Peace}.

% TODO: I think edinstvennyj avtor can be determinate or indeterminate in tolstoy2, since it also has an equative reading.

Now compare the same two sentences but without \textit{edinstvennyj}:

\begin{exe}
	\ex \gll Tolstoj --- avtor \textit{Vojny i mira}.\\
	Tolstoy {} author \textit{War and Peace}\\
	\glt `Tolstoy is the/an author of \textit{War and Peace}.'
	
	\ex \gll Tolstoj ne avtor \textit{Vojny i mira}.\\
	Tolstoy not author \textit{War and Peace} \\
	\glt `Tolstoy is not the/an author of \textit{War and Peace}.'
\end{exe}

In both sentences, \textit{avtor Vojny i mira} `author of \textit{War and Peace}' may be interpreted as either determinate or indeterminate. So the forced determinate reading in (\ref{tolstoy}) and the forced indeterminate reading in (\ref{tolstoy2}) is unexpected. That indeterminacy is mandatory in (\ref{tolstoy2}) is especially surprising given that indeterminate NPs are generally more restricted in their distribution than determinate ones in Russian \citep{geist2010}.

Similar effects occur in the argument position, although instead of pairs of affirmative-negated sentences, there are triplets of affirmative, sentential negated, and constituent negated sentences. The affirmative and sentential negated sentences require determinate readings of \textit{edinstvennyj} NPs and the constituent negated sentences require indeterminate readings.

The basic paradigm is established by (\ref{anna1})-(\ref{anna3}). The second sentence in each numbered example, separated for clarity, should be read as a continuation of the first. (\ref{anna1}) and (\ref{anna2}) both entail a single lecture, so \textit{edinstvennuju lekciju} `the only lecture' is determinate and the pronoun \textit{ona} `it' in the continuation sentence can refer to it. (\ref{anna3}) entails multiple lectures, so \textit{edinstvennuju lekciju} should be indeterminate and, as predicted, the pronoun in the continuation sentence cannot refer to it.

% TODO: Need to explain this diagnostic

\begin{exe}
	\ex \label{anna1} \begin{xlist}
		\ex \gll Anna posetila edinstvennuju lekciju, kotoruju pro\v{c}ital Xomskij, kogda byl v na\v{s}em universitete.\\
		Anna attended only lecture which gave Chomsky when was at our university\\
		\glt `Anna went to the only lecture that Chomsky gave at our university.'

		\ex \gll Ona byla o lingvistike.\\
		it was about linguistics\\
		\glt `It was about linguistics.'
	\end{xlist}

	\ex \label{anna2} \begin{xlist}
		\ex \gll Anna ne posetila edinstvennuju lekciju, kotoruju pro\v{c}ital Xomskij, kogda byl v na\v{s}em universitete.\\
		Anna not attended only lecture which gave Chomsky when was at our university\\
		\glt `Anna didn't go to the only lecture that Chomsky gave at our university.'

		\ex \gll Ona byla o lingvistike.\\
		it was about linguistics\\
		\glt `It was about linguistics.'
	\end{xlist}

	\ex \label{anna3} \begin{xlist}
		\ex \gll Anna posetila ne edinstvennuju lekciju, kotoruju pro\v{c}ital Xomskij, kogda byl v na\v{s}em universitete.\\
		Anna attended not only lecture which gave Chomsky when was at our university\\
		\glt `Anna went to one of the lectures that Chomsky gave at our university.'

		\ex \gll \# Ona byla o lingvistike.\\
		{} it was about linguistics\\
		\glt `It was about linguistics.'
	\end{xlist}
\end{exe}

As in (\ref{tolstoy}) and (\ref{tolstoy2}), a determinate reading of the \textit{edinstvennyj} NPs is forced in (\ref{anna1}) and (\ref{anna2}) and disallowed in (\ref{anna3}).

Note that Russian uses two different sentences to express the same two readings as sentences that are ambiguous in English, like (\ref{only-goal}) (though see below for an exception). This is because Russian allows the negation particle \textit{ne} to occur post-verbally, so that the ambiguity in English is realized as a difference of the position of the negation operator in Russian.

The Russian data is more complicated than the tidy contrast in (\ref{anna1})-(\ref{anna3}) would suggest, however. For one thing, some speakers \textit{do} permit the continuation in (\ref{anna3}). This phenomenon will be addressed in section \ref{sec:no-anti-unique}. For another, some speakers find (\ref{anna2}) marginal and prefer to state it as in (\ref{anna2.1}), which has a stronger implication that someone else went to Chomsky's only lecture. In examples to follow, I will use the variants corresponding to (\ref{anna2}) and not (\ref{anna2.1}) for consistency, as all of my consultants agreed that one or the other was grammatical.

\begin{exe}
	\ex \label{anna2.1} \gll Ne Anna posetila edinstvennuju lekciju, kotoruju pro\v{c}ital Xomskij, kogda byl v na\v{s}em universitete.\\
	Not Anna attended only lecture which gave Chomsky when was at our university\\
	\glt `It wasn't Anna that went to the only lecture that Chomsky gave at our university.'
\end{exe}

Some speakers interpret (\ref{anna3}) as entailing that Anna attended more than one of Chomsky's lectures, while others interpret it to mean that she attended only one (but that Chomsky gave more than one). In either case, the \textit{edinstvennyj} NP must be indeterminate since Chomsky gave more than one lecture.

The word \textit{edinstvennyj} requires greater context in Russian than DP-internal \textit{only} does in English. Most speakers judge (\ref{bad-edinstvennyj}) to be bad, for example, because of the lack of a restrictive clause or PP attached to \textit{edinstvennyj gol}. For that reason, the \textit{edinstvennyj} phrases throughout this section bear relative clauses or prepositional phrases to give the necessary context for their use to be grammatical in Russian.

\begin{exe}
	\ex[*] { \label{bad-edinstvennyj} \gll Anna zabila edinstvennyj gol.\\
	Anna scored only gol.\\
	\glt Intended: `Anna scored the only goal.'
	}
\end{exe}

% TODO: "must not be determinate otherwise" have to clarify since edinstvennyj can sometimes be indeterminate

Despite these caveats, the generalization remains that \textit{edinstvennyj} NPs must be indeterminate when in the scope of negation and must not be determinate otherwise. This generalization is borne out across a range of different verbs and NPs:

\begin{exe}
	\ex \begin{xlist}
		\ex \gll Marija napisala edinstvennoe xoro\v{s}oe so\v{c}inenie vo vs\"{e}m klasse.\\
		Maria wrote only good essay in entire class\\
		\glt `Maria wrote the only good essay in the entire class.'

		\ex \gll Ono bylo o russkoj literature.\\
		it was about Russian literature\\
		\glt `It was about Russian literature.'
	\end{xlist}

	\ex \begin{xlist}
		\ex \gll Marija ne napisala edinstvennoe xoro\v{s}oe so\v{c}inenie vo vs\"{e}m klasse.\\
		Maria not wrote only good essay in entire class\\
		\glt `Maria didn't write the only good essay in the entire class.'

		\ex \gll Ono bylo o russkoj literature.\\
		it was about Russian literature\\
		\glt `It was about Russian literature.'
	\end{xlist}

	\ex \label{maria3} \begin{xlist}
		\ex \gll Marija napisala ne edinstvennoe xoro\v{s}oe so\v{c}inenie vo vs\"{e}m klasse.\\
		Maria wrote not only good essay in entire class\\
		\glt `Maria wrote one of the good essays in the class.'

		\ex \gll \# Ono bylo o russkoj literature.\\
		{} it was about Russian literature\\
		\glt `It was about Russian literature.'
	\end{xlist}

	\ex \begin{xlist}
		\ex \gll Boris proizn\"{e}s edinstvennuju xoro\v{s}uju re\v{c}' na svad'be.\\
		Boris gave only good speech at wedding\\
		\glt `Boris gave the only good speech at the wedding.'

		\ex \gll Ono bylo o molodo\v{z}\"{e}nax.\\
		it was about newlyweds\\
		\glt `It was about the newlyweds.'
	\end{xlist}

	\ex \begin{xlist}
		\ex \gll Boris ne proizn\"{e}s edinstvennuju xoro\v{s}uju re\v{c}' na svad'be.\\
		Boris not gave only good speech at wedding\\
		\glt `Boris didn't give the only good speech at the wedding.'

		\ex \gll Ono bylo o molodo\v{z}\"{e}nax.\\
		it was about newlyweds\\
		\glt `It was about the newlyweds.'
	\end{xlist}

	\ex \label{boris3} \begin{xlist}
		\ex \gll Boris proizn\"{e}s ne edinstvennuju xoro\v{s}uju re\v{c}' na svad'be.\\
		Boris gave not only good speech at wedding\\
		\glt `Boris gave one of the good speeches at the wedding.'

		\ex \gll \# Ono bylo o molodo\v{z}\"{e}nax.\\
		{} it was about newlyweds\\
		\glt `It was about the newlyweds.'
	\end{xlist}
\end{exe}

Interestingly, \textit{edinstvennyj} NPs also must be indeterminate with verbs of non-creation like \textit{uvidet'} `to see' and \textit{probovat'} `to taste':

\begin{exe}
	\ex \begin{xlist}
		\ex \gll Lena uvidela edinstvennogo krokodila, kotoryj byl v zooparke.\\
		Lena saw only crocodile which was at zoo\\
		\glt `Lena saw the only crocodile at the zoo.'

		\ex \gll On byl dlinoy tri metra.\\
		it was lengthwise three meters\\
		\glt `It was three meters long.'
	\end{xlist}

	\ex \begin{xlist}
		\ex \gll Lena ne uvidela edinstvennogo krokodila, kotoryj byl v zooparke.\\
		Lena not saw only crocodile which was at zoo\\
		\glt `Lena didn't see the only crocodile at the zoo.'

		\ex \gll On byl dlinoy tri metra.\\
		it was lengthwise three meters\\
		\glt `It was three meters long.'
	\end{xlist}

	\ex \label{lena3} \begin{xlist}
		\ex \gll Lena uvidela ne edinstvennogo krokodila, kotoryj byl v zooparke.\\
		Lena saw not only crocodile which was at zoo\\
		\glt `Lena saw one of the crocodiles at the zoo.'

		\ex \gll \# On byl dlinoy tri metra.\\
		{} it was lengthwise three meters\\
		\glt `It was three meters long.'
	\end{xlist}

	\ex \begin{xlist}
		\ex \gll Ol'ga poprobovala edinstvennyj tort, kotoryj byl na ve\v{c}erinke.\\
		Olga tasted only cake which was at party\\
		\glt `Olga tasted the only cake at the party.'

		\ex \gll On byl \v{s}okoladnyj.\\
		it was chocolate\\
		\glt `It was chocolate.'
	\end{xlist}

	\ex \begin{xlist}
		\ex \gll Ol'ga ne poprobovala edinstvennyj tort, kotoryj byl na ve\v{c}erinke.\\
		Olga not tasted only cake which was at party\\
		\glt `Olga didn't taste the only cake at the party.'

		\ex \gll On byl \v{s}okoladnyj.\\
		it was chocolate\\
		\glt `It was chocolate.'
	\end{xlist}

	\ex \label{olga3} \begin{xlist}
		\ex \gll Ol'ga poprobovala ne edinstvennyj tort, kotoryj byl na ve\v{c}erinke.\\
		Olga tasted not only cake which was at party\\
		\glt `Olga tasted one of the cakes at the party.'

		\ex \gll \# On byl \v{s}okoladnyj.\\
		{} it was chocolate\\
		\glt `It was chocolate.'
	\end{xlist}
\end{exe}

In summary, although \textit{edinstvennyj} NPs may sometimes be indeterminate without the help of negation, there is a systematic class of examples where \textit{edinstvennyj} NPs must be determinate in affirmative and sentential negated sentences and must be indeterminate in constituent negated sentences. This pattern parallels the data from English from section \ref{sec:only-nps-english}, in that negation (constituent in Russian, sentential in English) forces an indeterminate reading of NPs with \textit{only} and \textit{edinstvennyj}.

Again, it is not the fact that \textit{edinstvennyj} NPs may be indeterminate in Russian that is notable. It is the fact that negation forces determinate \textit{edinstvennyj} NPs to become indeterminate.

% TODO: Say something about constituent vs. sentential relating to the ambiguity in English?

\subsection{Subsequent reference is possible in some idiolects\label{sec:no-anti-unique}}
The data presented above will be analyzed in greater depth in section \ref{sec:my-theory}. But a set of limited but systematic counterexamples should be dealt with first. As noted previously, not all Russian speakers find the continuation in (\ref{anna3}) ungrammatical. The combination in (\ref{no-anti-unique}) is grammatical for these speakers, and similarly for (\ref{maria3}), (\ref{boris3}), (\ref{lena3}) and (\ref{olga3}).

\begin{exe}
	\ex \label{no-anti-unique} \begin{xlist}
		\ex \gll Anna posetila ne edinstvennuju lekciju, kotoruju pro\v{c}ital Xomskij, kogda byl v na\v{s}em universitete.\\
		Anna attended not only lecture which gave Chomsky when was at our university\\
		\glt `Anna went to one of the lectures Chomsky gave at our university.'

		\ex \gll Ona byla o lingvistike.\\
		it was about linguistics\\
		\glt `It was about linguistics.'
	\end{xlist}
\end{exe}

In order to express these sentences idiomatically in English, one has to resort to using the indefinite \textit{one of the lectures} in (\ref{no-anti-unique}), but in Russian the same definite descriptions is used as in (\ref{anna1}) and (\ref{anna2}).
