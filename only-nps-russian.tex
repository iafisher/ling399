\section{\textit{Edinstvennyj} NPs in Russian \label{sec:only-nps-russian}}
\citet{cb2015} observed that negated \textit{only} NPs do not presuppose uniqueness in English. Russian also evinces an unexpected interaction between DP-internal \textit{only} and definiteness.

% TODO: Transition

Russian lacks articles and allows bare nominals in argument positions. \textit{Kniga} `book' in (\ref{kniga}), for instance, may be interpreted as either indefinite or definite depending on context.

\begin{exe}
	\ex \label{kniga} \gll Anna \v{c}itaet knigu.\\
	Anna reads book\\
	\glt `Anna is reading a book' or `Anna is reading the book.'
\end{exe}

The standard account of bare nominals in Russian is that they receive their determinate or indeterminate import by covert type-shifting \citep{chierchia98}, with \citegen{partee86} \textsc{Iota} and \textsc{A} shifts for definite and indefinite nominals respectively. These type shifts are presumably embodied by phonetically null determiners in the syntax.

% TODO: Transition

I will show that there is an unexpected paradigm of forced definite and indefinite readings of \textit{edinstvennyj} NPs (the Russian equivalent of \textit{only} NPs). Whenever a \textit{edinstvennyj} NP is negated as a constituent (which I will occasionally refer to as \textit{ne edinstvennyj} NPs), an indefinite reading is forced, while other argumental \textit{edinstvennyj} NPs, in both affirmative and negative sentences, must be interpreted as definite without extra semantic help. Unlike in English, argumental \textit{edinstvennyj} NPs in negated sentences are not ambiguous between indefinite and definite readings.

The basic paradigm is established by (\ref{anna1})-(\ref{anna3}). The second sentence in each numbered example, separated for clarity, should be read as a continuation of the first. (\ref{anna1}) and (\ref{anna2}) both entail a single lecture, so \textit{edinstvennuju lekciju} `the only lecture' is definite. (\ref{anna3}) entails multiple lectures, so \textit{edinstvennuju lekciju} is indefinite.

\begin{exe}
	\ex \label{anna1} \begin{xlist}
		\ex \gll Anna posetila edinstvennuju lekciju, kotoruju pro\v{c}ital Xomskij, kogda byl v na\v{s}em universitete.\\
		Anna attended only lecture which gave Chomsky when was at our university\\
		\glt `Anna went to the only lecture that Chomsky gave at our university.'

		\ex \gll Ona byla o lingvistike.\\
		it was about linguistics\\
		\glt `It was about linguistics.'
	\end{xlist}

	\ex \label{anna2} \begin{xlist}
		\ex \gll Anna ne posetila edinstvennuju lekciju, kotoruju pro\v{c}ital Xomskij, kogda byl v na\v{s}em universitete.\\
		Anna not attended only lecture which gave Chomsky when was at our university\\
		\glt `Anna didn't go to the only lecture that Chomsky gave at our university.'

		\ex \gll Ona byla o lingvistike.\\
		it was about linguistics\\
		\glt `It was about linguistics.'
	\end{xlist}

	\ex \label{anna3} \begin{xlist}
		\ex \gll Anna posetila ne edinstvennuju lekciju, kotoruju pro\v{c}ital Xomskij, kogda byl v na\v{s}em universitete.\\
		Anna attended not only lecture which gave Chomsky when was at our university\\
		\glt `Anna went to one of the lectures that Chomsky gave at our university.'

		\ex \gll \# Ona byla o lingvistike.\\
		{} it was about linguistics\\
		\glt `It was about linguistics.'
	\end{xlist}
\end{exe}

A more robust indication that \textit{edinstvennuju lekciju} in is definite in (\ref{anna2}) and indefinite in (\ref{anna3}) is that it can be the antecedent of the pronoun in the continuation sentence in (\ref{anna2}) but not (\ref{anna3}). Definites in negated sentences still license subsequent reference, while indefinites do not:

\begin{exe}
	\ex Anna didn't go to (the/*a) lecture, because it was too late in the evening.
\end{exe}

In Russian, negation forces the definite reading:

\begin{exe}
	% TODO: Run this by Sasha
	\ex \gll Anna ne posetila lekciju {potomu, \v{c}to} ona byla sli\v{s}kom pozdno ve\v{c}erom.\\
	Anna not attended lecture because it was too late {in the evening}\\
	\glt `Anna didn't go to the lecture, because it was too late in the evening.'
\end{exe}

Note that in (\ref{anna1}) and (\ref{anna2}) Russian uses two different sentences to express the two ambiguous readings in English sentences like (\ref{only-goal}).

The Russian data is more complicated than the tidy contrast in (\ref{anna1})-(\ref{anna3}) would suggest, however. For one thing, some speakers \textit{do} permit the continuation in (\ref{anna3}). This phenomenon will be addressed in section \ref{sec:my-theory}. For another, some speakers find (\ref{anna2}) marginal and prefer to state it as in (\ref{anna2.1}), which has a stronger implication that someone else went to Chomsky's only lecture. In examples to follow, I will use the variants corresponding to (\ref{anna2}) and not (\ref{anna2.1}) for consistency, as all of my consultants agreed that one or the other was grammatical.

\begin{exe}
	\ex \label{anna2.1} \gll Ne Anna posetila edinstvennuju lekciju, kotoruju pro\v{c}ital Xomskij, kogda byl v na\v{s}em universitete.\\
	Not Anna attended only lecture which gave Chomsky when was at our university\\
	\glt `It wasn't Anna that went to the only lecture that Chomsky gave at our university.'
\end{exe}

Some speakers interpret (\ref{anna3}) as entailing that Anna attended more than one of Chomsky's lectures, while others interpret it to mean that she attended only one (but that Chomsky gave more than one). In either case, the \textit{edinstvennyj} NP must be indefinite since Chomsky gave more than one lecture.

The word \textit{edinstvennyj} requires greater context in Russian than DP-internal \textit{only} does in English. Most speakers judge (\ref{bad-edinstvennyj}) to be bad, for example, because of the lack of a restrictive clause or PP attached to \textit{edinstvennyj gol}. For that reason, the \textit{edinstvennyj} phrases throughout this section bear relative clauses or prepositional phrases to give the necessary context for their use to be grammatical in Russian.

\begin{exe}
	\ex[*] { \label{bad-edinstvennyj} \gll Anna zabila edinstvennyj gol.\\
	Anna scored only gol.\\
	\glt Intended: `Anna scored the only goal.'
	}
\end{exe}

Despite these caveats, the generalization remains that argumental \textit{edinstvennyj} NPs must be indefinite when in the scope of constituent negation and must not be definite otherwise. This generalization is borne out across a range of different verbs and NPs:

\begin{exe}
	\ex \begin{xlist}
		\ex \gll Marija napisala edinstvennoe xoro\v{s}oe so\v{c}inenie vo vs\"{e}m klasse.\\
		Maria wrote only good essay in entire class\\
		\glt `Maria wrote the only good essay in the entire class.'

		\ex \gll Ono bylo o russkoj literature.\\
		it was about Russian literature\\
		\glt `It was about Russian literature.'
	\end{xlist}

	\ex \begin{xlist}
		\ex \gll Marija ne napisala edinstvennoe xoro\v{s}oe so\v{c}inenie vo vs\"{e}m klasse.\\
		Maria not wrote only good essay in entire class\\
		\glt `Maria didn't write the only good essay in the entire class.'

		\ex \gll Ono bylo o russkoj literature.\\
		it was about Russian literature\\
		\glt `It was about Russian literature.'
	\end{xlist}

	\ex \label{maria3} \begin{xlist}
		\ex \gll Marija napisala ne edinstvennoe xoro\v{s}oe so\v{c}inenie vo vs\"{e}m klasse.\\
		Maria wrote not only good essay in entire class\\
		\glt `Maria wrote one of the good essays in the class.'

		\ex \gll \# Ono bylo o russkoj literature.\\
		{} it was about Russian literature\\
		\glt `It was about Russian literature.'
	\end{xlist}
\end{exe}

With \textit{proiznesti} `to give (a speech)':

\begin{exe}
	\ex \begin{xlist}
		\ex \gll Boris proizn\"{e}s edinstvennuju xoro\v{s}uju re\v{c}' na svad'be.\\
		Boris gave only good speech at wedding\\
		\glt `Boris gave the only good speech at the wedding.'

		\ex \gll Ono bylo o molodo\v{z}\"{e}nax.\\
		it was about newlyweds\\
		\glt `It was about the newlyweds.'
	\end{xlist}

	\ex \begin{xlist}
		\ex \gll Boris ne proizn\"{e}s edinstvennuju xoro\v{s}uju re\v{c}' na svad'be.\\
		Boris not gave only good speech at wedding\\
		\glt `Boris didn't give the only good speech at the wedding.'

		\ex \gll Ono bylo o molodo\v{z}\"{e}nax.\\
		it was about newlyweds\\
		\glt `It was about the newlyweds.'
	\end{xlist}

	\ex \label{boris3} \begin{xlist}
		\ex \gll Boris proizn\"{e}s ne edinstvennuju xoro\v{s}uju re\v{c}' na svad'be.\\
		Boris gave not only good speech at wedding\\
		\glt `Boris gave one of the good speeches at the wedding.'

		\ex \gll \# Ono bylo o molodo\v{z}\"{e}nax.\\
		{} it was about newlyweds\\
		\glt `It was about the newlyweds.'
	\end{xlist}
\end{exe}

Not only verbs of creation force the indefinite reading of \textit{edinstvennyj} NPs. Other verbs like \textit{uvidet'} `to see' and \textit{poprobovat'} `to taste' do so also:

\begin{exe}
	\ex \begin{xlist}
		\ex \gll Lena uvidela edinstvennogo krokodila, kotoryj byl v zooparke.\\
		Lena saw only crocodile which was at zoo\\
		\glt `Lena saw the only crocodile at the zoo.'

		\ex \gll On byl dlinoy tri metra.\\
		it was lengthwise three meters\\
		\glt `It was three meters long.'
	\end{xlist}

	\ex \begin{xlist}
		\ex \gll Lena ne uvidela edinstvennogo krokodila, kotoryj byl v zooparke.\\
		Lena not saw only crocodile which was at zoo\\
		\glt `Lena didn't see the only crocodile at the zoo.'

		\ex \gll On byl dlinoy tri metra.\\
		it was lengthwise three meters\\
		\glt `It was three meters long.'
	\end{xlist}

	\ex \label{lena3} \begin{xlist}
		\ex \gll Lena uvidela ne edinstvennogo krokodila, kotoryj byl v zooparke.\\
		Lena saw not only crocodile which was at zoo\\
		\glt `Lena saw one of the crocodiles at the zoo.'

		\ex \gll \# On byl dlinoy tri metra.\\
		{} it was lengthwise three meters\\
		\glt `It was three meters long.'
	\end{xlist}

	\ex \begin{xlist}
		\ex \gll Ol'ga poprobovala edinstvennyj tort, kotoryj byl na ve\v{c}erinke.\\
		Olga tasted only cake which was at party\\
		\glt `Olga tasted the only cake at the party.'

		\ex \gll On byl \v{s}okoladnyj.\\
		it was chocolate\\
		\glt `It was chocolate.'
	\end{xlist}

	\ex \begin{xlist}
		\ex \gll Ol'ga ne poprobovala edinstvennyj tort, kotoryj byl na ve\v{c}erinke.\\
		Olga not tasted only cake which was at party\\
		\glt `Olga didn't taste the only cake at the party.'

		\ex \gll On byl \v{s}okoladnyj.\\
		it was chocolate\\
		\glt `It was chocolate.'
	\end{xlist}

	\ex \label{olga3} \begin{xlist}
		\ex \gll Ol'ga poprobovala ne edinstvennyj tort, kotoryj byl na ve\v{c}erinke.\\
		Olga tasted not only cake which was at party\\
		\glt `Olga tasted one of the cakes at the party.'

		\ex \gll \# On byl \v{s}okoladnyj.\\
		{} it was chocolate\\
		\glt `It was chocolate.'
	\end{xlist}
\end{exe}

In summary, although \textit{edinstvennyj} NPs may sometimes be indefinite without the help of negation, there is a systematic class of examples where \textit{edinstvennyj} NPs must be definite in affirmative and sentential negated sentences and must be indefinite in constituent negated sentences. This pattern parallels the data from English from section \ref{sec:only-nps-english}, in that negation (constituent in Russian, sentential in English) forces an indefinite reading of NPs with \textit{only} and \textit{edinstvennyj}.
