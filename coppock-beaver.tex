\section{Critique of \citet{cb2015}\label{sec:coppock-beaver}}
\citet{cb2015} offer an alternative theory of the interpretation of \textit{only} NPs. The cornerstone of their theory is that \textsc{Iota} and \textsc{A} are the source of the uniqueness and existence entailments for all definite and indefinite descriptions. On their account, the definite article encodes a weak uniqueness presupposition, while the indefinite article is an identity function on predicates. Weak uniqueness means uniqueness or non-existence, i.e. if uniqueness is $|P| = 1$ then weak uniqueness is $|P| \le 1$.

\begin{exe}
	\ex $\textit{the} = \lambda P . \lambda x . [\partial(|P| \le 1) \land P(x)]$
	\ex $\textit{a(n)} = \lambda P . \lambda x . P(x)$
\end{exe}

A consequence of these formulae is that definite and indefinite descriptions are of type $\langle e, t \rangle$. In order to appear in argument positions, they must undergo either type-shifting to become type $e$, or syntactic raising.

\citeauthor{cb2015} distinguish between definiteness, a morphological category signaled by the definite article in English, and determinacy, the property of denoting an individual. Except for \textit{only} NPs, definites in English are always determinate, and indefinites are always indeterminate, so they need an explanation of why \textsc{Iota} can never apply to indefinites, and \textsc{A} can only apply to definites when they are negated \textit{only} NPs. They accomplish this with an appeal to the principles of Maximize Presupposition and Type Simplicity. Informally, Maximize Presupposition requires that if there are two possible words whose meanings are identical, then the one with the greater presupposition must be chosen. The indefinite and definite article have the same assertive content, but the definite article has an additional presupposition (weak uniqueness), so it is favored by Maximize Presupposition.

Concretely, Maximize Presupposition applies to (\ref{helen}) to rule out the undesired derivation where \textsc{Iota} applies to \textit{a chair} to yield a determinate reading. \textsc{Iota} can only apply when uniqueness is met, but if it had been met, then Maximize Presupposition would have forced the choice of \textit{the} instead of \textit{a}.

\begin{exe}
	\ex \label{helen} Helen took a chair.
\end{exe}

Type Simplicity is the preference for simpler types, all else being equal. \textsc{Iota} has type $\langle et, e \rangle$ while \textsc{A} has type $\langle et, \langle et, t \rangle \rangle$, so \textsc{Iota} would be preferred unless its licensing condition (uniqueness) is not met.

In summary, \citeauthor{cb2015}'s theory of definiteness has the following components:

\begin{itemize}
	\item The definite article encodes a weak uniqueness presupposition. The indefinite article is an identity function.
	\item Definite descriptions are fundamentally predicative and are of type $\langle e, t \rangle$ before type-shifting.
	\item A DP receives its determinacy or indeterminacy through an \textsc{Iota} or \textsc{A} type-shift. The principles of Maximize Presupposition and Type Simplicity ensure that definiteness and determinacy, and indefiniteness and indeterminacy, coincide in English except in the case of negated \textit{only} NPs.
\end{itemize}

\citeauthor{cb2015}'s theory is more expansive than my proposal in section \ref{sec:my-theory}, because it reimagines definiteness not just for \textit{only} NPs but for all other definite and indefinite expressions as well. One might question whether the universal application of covert type-shifts in English is sufficiently motivated by the evidence of negated \textit{only} NPs, especially considering that there are two perfectly good overt operators, \textit{the} and \textit{a(n)}, which do the same work. Additionally, uniqueness is redundantly encoded in their system by the weak uniqueness presupposition of the definite article and the strong uniqueness presupposition of \textsc{Iota}.

The reason that \citeauthor{cb2015} offer such a wide-ranging proposal is that negated \textit{only} NPs directly contradict any account of definiteness in which uniqueness is presupposed by the definite article. As long as one maintains that \textit{only} NPs are regular definite descriptions, then indeterminate \textit{only} NPs are a flat contradiction.

However, as I have shown in section \ref{sec:my-theory}, if one abandons the assumption that \textit{only} NPs are proper definites, then no modifications to the theory of definiteness are necessary, and the semantics of \textit{only} NPs can be formulated in a manner that is consistent with their limited scope of application.