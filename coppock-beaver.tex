\section{\citet{cb2015} \label{sec:coppock-beaver}}
The cornerstone of \citeauthor{cb2015}'s theory is that the definite article encodes a weak uniqueness presupposition, while the indefinite article is an identity function on predicates. Weak uniqueness means uniqueness or non-existence, i.e. if uniqueness is $|P| = 1$ then weak uniqueness is $|P| \le 1$.

\begin{exe}
	\ex $\textit{the} = \lambda P . \lambda x . [\partial(|P| \le 1) \land P(x)]$
	\ex $\textit{a(n)} = \lambda P . P$
\end{exe}

A typical definite like \textit{the table} would be given the formula in (\ref{the-table}).

\begin{exe}
	\ex \label{the-table} $\textit{the table} = \lambda x . [ \partial(|\textsc{Table}| \le 1) \land \textsc{Table}(x) ]$
\end{exe}

A consequence of \citeauthor{cb2015}'s definition of the definite article is that definite descriptions are of type $\langle e, t \rangle$. Of course, definite descriptions commonly appear in argument positions where they are of type $e$. To allow for this, \citeauthor{cb2015} propose that the \textsc{Iota} and \textsc{A} type shifts apply in English to yield determinate and indeterminate readings of DPs.

% TODO: "type shifts" or "type shifters"?

Definites in English are always determinate, except for \textit{only} NPs in the scope of negation, and indefinites are always indeterminate, so \citeauthor{cb2015} need an account of why \textsc{Iota} can never apply to indefinites, and \textsc{A} can only apply to negated \textit{only} NPs. They account for this with the principles of Maximize Presupposition and Type Simplicity. Informally, Maximize Presupposition states that if there are two possible words whose meanings are identical, then the one with the greater presupposition must be chosen. Per \citeauthor{cb2015}, the indefinite and definite article have the same meaning, but the definite article has an extra presupposition of weak uniqueness, so in a situation where weak uniqueness is in the common ground, the definite article must be chosen.

Type Simplicity is the preference for simpler types, all else being equal. \textsc{Iota} has type $\langle et, e \rangle$ while \textsc{A} has type $\langle et, \langle et, t \rangle \rangle$, so \textsc{Iota} would be preferred unless its licensing condition (uniqueness) is not met.

Thus the principles of Maximize Presupposition and Type Simplicity ensure that definiteness and determinacy, and indefiniteness and indeterminacy, coincide in English except in the case of \textit{only} NPs.

In summary, \citeauthor{cb2015}'s theory of definiteness has the following components:

\begin{itemize}
	\item The definite article encodes a weak uniqueness presupposition. The indefinite article is an identity function.
	\item Definite descriptions are fundamentally predicative and have type $\langle e, t \rangle$ before type-shifting.
	\item A DP receives its determinacy or indeterminacy through an \textsc{Iota} or \textsc{A} type shifts.
\end{itemize}

\citeauthor{cb2015}'s theory leaves something to be desired. Their proposal is essentially a wholesale reimagining of the structure of definiteness in English, and a redefinition of the definite and indefinite articles that removes most of their semantic content, not just with \textit{only} NPs but for all definites. The basis for this is quite limited: all of their counterexamples crucially involve \textit{only} NPs in the scope of negation.

The reason that \citeauthor{cb2015} offer such a wide-ranging proposal is that their counterexamples directly contradict the Fregean and Russellians accounts of definiteness, that is, accounts of definiteness with existence and uniqueness built-in to the definite article. As long as one maintains that \textit{only} NPs are regular definite descriptions, then indeterminate \textit{only} NPs are a flat contradiction to Frege and Russell.

Familiarity theories of definiteness, which are prominent alternatives to Frege's and Russell's account, are equally ill-equipped to deal with definites with DP-internal \textit{only}. Familiarity theories locate the fundamental difference between indefinites and definites in novelty and familiarity: the definite article is used when the referent is familiar to both speaker and listener, and the indefinite article is used when it is familiar only to the speaker \citep{heim82}.

In the exchange in (\ref{newcastle}), the referent of \textit{the only goal} is clearly not familiar to the speaker (hence why \textit{the goal} is not licensed), but it can nonetheless be used felicitously.

\begin{exe}
	\ex \label{newcastle} - What happened in the match this morning? \\
	- Not much. Newcastle scored the *(only) goal.
\end{exe}

However, as I have shown in section \ref{sec:my-theory}, if one abandons the assumption that \textit{only} NPs are proper definites, then no modifications to the theory of definiteness are necessary, and the semantic account of \textit{only} NPs can be formulated in a manner that is consistent with their limited scope of application.