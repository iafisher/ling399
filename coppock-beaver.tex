\section{\citet{cb2015} \label{coppock-beaver}}
The cornerstone of \citeauthor{cb2015}'s theory is that the definite and indefinite articles in English are identity functions, and the definite article carries an additional weak uniqueness presupposition. In the formulae below, $\partial$ stands for \citegen{beaver92} partial operator which is used to model presuppositions compositionally: $\partial(\phi)$ is true if $\phi$ is true and undefined otherwise.

\begin{exe}
	\ex $\textit{the} = \lambda P . \lambda x . [\partial(|P| \le 1) \land P(x)]$
	\ex $\textit{a(n)} = \lambda P . \lambda x . P(x)$
\end{exe}

A typical definite like \textit{the table} would be given the formula in (\ref{the-table}).

\begin{exe}
	\ex \label{the-table} $\textit{the table} = \lambda x . [ \partial(|\textsc{Table}| \le 1) \land \textsc{Table}(x) ]$
\end{exe}

A consequence of \citeauthor{cb2015}'s definition of the definite article is that definite descriptions are of type $\langle e, t \rangle$. Of course, definite descriptions commonly appear in argument positions where they are of type $e$. To allow for this, \citeauthor{cb2015} propose that the covert type shifters \textsc{Iota} and \textsc{A} from \citet{partee86} apply in English to yield determinate and indeterminate readings of DPs.

Definites are always determinate, except in the case of anti-uniqueness effects, and indefinites are always indeterminate, so \citeauthor{cb2015} need an account of why \textsc{Iota} can never apply to indefinites, and \textsc{A} can only apply to anti-unique definites. They account for this with the principles of Maximize Presupposition and Type Simplicity. Informally, Maximize Presupposition states that if there are two possible words whose meanings are identical, then the one with the greater presupposition must be chosen. Per \citeauthor{cb2015}, the indefinite and definite article have the same meaning, but the definite article has an extra presupposition of weak uniqueness, so in a situation where weak uniqueness is in the common ground, the definite article must be chosen.

Type Simplicity is the preference for simpler types, all else being equal. \textsc{Iota} has type $\langle et, e \rangle$ while \textsc{A} has type $\langle et, \langle et, t \rangle \rangle$, so \textsc{Iota} would be preferred unless its licensing condition (uniqueness) is not met.

Thus the principles of Maximize Presupposition and Type Simplicity ensure that definiteness and determinacy, and indefiniteness and indeterminacy, coincide in English except in the case of anti-uniqueness effects.

In summary, \citeauthor{cb2015}'s theory of definiteness has the following components:

\begin{itemize}
	\item The definite and indefinite articles are identity functions. The definite article additionally carries a weak uniqueness presupposition.
	\item Definite and indefinite are fundamentally predicative and have type $\langle e, t \rangle$ before type-shifting.
	\item DPs receive their determinacy or indeterminacy through the \textsc{Iota} and \textsc{A} covert type shifts.
\end{itemize}

The Russian data in this paper should pose no particular issues for \citeauthor{cb2015}'s analysis---its role was to establish a cross-linguistic understanding of anti-uniqueness effects. Nevertheless, \citeauthor{cb2015}'s theory leaves something to be desired. Their proposal is essentially a wholesale reimagining of the structure of definiteness in English, and a redefinition of the definite and indefinite articles that removes nearly all their semantic content. The basis for this is quite limited: all of their counterexamples crucially involve DP-internal \textit{only} or other exclusive adjectives in the scope of negation.

The reason that \citeauthor{cb2015} offer such an ambitious proposal is that their counterexamples directly contradict the Fregean and Russellians accounts of definiteness, that is, accounts of definiteness with existence and uniqueness as the meaning of the definite article. As long as one maintains that the phrases with DP-internal \textit{only} are regular definite descriptions, then anti-uniqueness effects are a flat contradiction that cannot be reconciled.

Familiarity theories of definiteness, which are prominent alternatives to Frege's and Russell's account, are equally ill-equipped to deal with definites with DP-internal \textit{only}. Familiarity theories locate the fundamental difference between indefinites and definites in novelty and familiarity: the definite article is used when the referent is familiar to both speaker and listener, and the indefinite article is used when it is familiar only to the speaker \citep{heim82}.

In the exchange in (\ref{newcastle}), the referent of \textit{the only goal} is clearly not familiar to the speaker (hence why \textit{the goal} is not licensed), but it can nonetheless be used felicitously.

\begin{exe}
	\ex \label{newcastle} - What happened in the match this morning? \\
	- Not much. Newcastle scored the *(only) goal.
\end{exe}

However, as I have shown in section \ref{sec:my-theory}, if one abandons the assumption that \textit{the only} phrases are proper definites, then no modifications to the theory of definiteness are necessary, and the semantic account of anti-uniqueness effects can be formulated in a manner that is consistent with their limited scope of application.