\section{Introduction \label{sec:intro}}
Suppose, counterfactually, that the novel \textit{Waverley} was written by a committee comprising Scott, Macfarlane, and Campbell. In such circumstances, one could truthfully utter (\ref{scott}).

\begin{exe}
	\ex \label{scott} Scott is not the only author of \textit{Waverley}.\\
	\hspace*{\fill} \citep{cb2015}  % Right-justify citation
\end{exe}

What does the definite description \textit{the only author of Waverley} in (\ref{scott}) denote? In conventional theories of definiteness,\footnote{That is, in theories of definiteness in the vein of \citet{frege} or \citet{strawson50}, in which the definite article presupposes uniqueness. Familiarity theories of definiteness, which describe the definite article's licensing conditions in different terms, will be addressed briefly in section \ref{sec:coppock-beaver}. See \citet[chap. 1]{schwarz09} and \citet{horn-abbott-2012} for general discussion of the principle contemporary theories of definiteness.} an expression of the form \textit{the P} can only be used when a single individual satisfies the predicate \textit{P}. But if Scott, Macfarlane and Campbell are all authors of \textit{Waverley}, then there is no single ``only author of \textit{Waverley},'' and the definite description should not be interpretable. Nonetheless, (\ref{scott}) is a grammatical sentence of English.

The main thrust of this paper is to argue for an account of (\ref{scott}) and similar examples in English and Russian wherein \textit{only} presupposes existence and asserts uniqueness, and \textit{the} is not the definite article at all, but a semantically vacuous determiner. The major advantage of this analysis compared to \citegen{cb2015} is that it is compatible with a Fregean theory of definiteness.

Here and throughout the term ``DP-internal \textit{only}'' (following \citet{mcnally08}) refers to the adjectival use of \textit{only} in phrases such as \textit{the only author of Waverley} in (\ref{scott}).

The paper is organized as follows. Section \ref{sec:only-nps-english} reviews the data presented in \citet{cb2015} on DP-internal \textit{only} and definiteness. Section \ref{sec:only-nps-russian} shows that Russian also evinces a surprising interaction between \textit{only} and definiteness. Section \ref{sec:my-theory} presents my analysis of the semantics of \textit{only} NPs in detail. Section \ref{sec:coppock-beaver} compares my analysis to \citegen{cb2015}. Section \ref{sec:edinstvennyj} describes the semantics of DP-internal \textit{only} in Russian in more depth. Section \ref{sec:two-onlys} examines the adverbial use of the word \textit{only}. Section \ref{sec:conclusion} concludes the paper.