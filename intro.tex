\section{Introduction \label{sec:intro}}
A definite description is conventionally assumed to presuppose existence and uniqueness. That is, an expression of the form \textit{the P} may only be used when the predicate \textit{P} has a unique satisfier in the domain of discourse. If \textit{P} has no satisfiers or if it has more than one, then the expression \textit{the P} cannot be used.

There exist exceptions to this generalization, however. In particular, there is a class of ``anti-unique'' definites, discussed by \citet{cb2015}, which lack an existence entailment. (\ref{scott}) is an example. The use of the definite description \textit{the only author of Waverley} ought to require that there be a single individual with the ``only author of \textit{Waverley}'' property. But this is plainly not the case: if Scott is not the only author of \textit{Waverley}, then there must be multiple authors, and thus no individual has the property of being the ``only author of \textit{Waverley}.''

\begin{exe}
	\ex \label{scott} Scott is not the only author of \textit{Waverley}.
\end{exe}

Observe that it is crucially the words \textit{only} and \textit{not} which instigate this state of affairs. When either is removed, as in (\ref{scott-wo-only}) and (\ref{scott-wo-not}), then the definite description retains a unique referent.

\begin{exe}
	\ex \label{scott-wo-only} Scott is not the author of \textit{Waverley}.
	\ex \label{scott-wo-not} Scott is the only author of \textit{Waverley}.
\end{exe}

The contrast between (\ref{scott}) and (\ref{scott-wo-only}-\ref{scott-wo-not}) illustrates the surprising interaction of DP-internal \textit{only}, negation and definiteness. The main thrust of this paper is to explore this interaction and its ramifications in Russian, and to offer an alternative proposal to \citegen{cb2015} of the semantics of anti-uniqueness effects.

The paper is organized as follows. Sections \ref{sec:anti-uniqueness-english} and \ref{sec:anti-uniqueness-russian} review the distribution of anti-uniqueness effects in English and Russian, respectively. Section \ref{sec:indeterminate-only} discusses the possibility of DP-internal \textit{only} phrases with indeterminate readings. Section \ref{sec:existence-and-uniqueness} argues for the existence presupposition and uniqueness assertion of DP-internal \textit{only}. Section \ref{sec:edinstvennyj} reviews the syntax and semantics of DP-internal \textit{only} in Russian. Section \ref{sec:two-onlys} compares the DP-internal and DP-external uses of the word \textit{word}. Section \ref{sec:conclusion} concludes the paper.