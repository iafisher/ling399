\section{Introduction \label{sec:intro}}
A definite description is conventionally assumed to presuppose existence and uniqueness. That is, an expression of the form \textit{the P} may only be used when the predicate \textit{P} has a unique\footnote{To account for plural definites like \textit{the tables}, the concept of uniqueness can be generalized to maximality: \textit{the tables} may only refer to a maximal grouping of tables in a given context. Further discussion of plural definites is beyond the scope of discussion of this paper.} satisfier in the domain of discourse. If \textit{P} has no satisfiers or if it has more than one, then the expression \textit{the P} cannot be used \citep{frege, russell, horn-abbott-2012}.
% TODO: Handle obvious objection to this statement: plural definites

There exist exceptions to this generalization, however. In particular, \textit{only} NPs in the scope of negation not only lack a uniqueness presupposition but can be used to deny uniqueness \citep{cb2015}. (\ref{scott}) is an example. The use of the definite description \textit{the only author of Waverley} ought to require that there be a single individual with the ``only author of \textit{Waverley}'' property. But this is plainly not the case: if Scott is not the only author of \textit{Waverley}, then there must be multiple authors, and thus uniqueness with respect to the set of authors of \textit{Waverley} is not upheld.

\begin{exe}
	\ex \label{scott} Scott is not the only author of \textit{Waverley}.
\end{exe}

Observe that it is crucially the words \textit{only} and \textit{not} which cancel the uniqueness presupposition. When either is removed, as in (\ref{scott-wo-only}) and (\ref{scott-wo-not}), then the definite description refers to a single, unique author.

\begin{exe}
	\ex \label{scott-wo-only} Scott is not the author of \textit{Waverley}.
	\ex \label{scott-wo-not} Scott is the only author of \textit{Waverley}.
\end{exe}

% TODO: Mention Russian data
The contrast between (\ref{scott}) and (\ref{scott-wo-only}-\ref{scott-wo-not}) illustrates the surprising interaction of DP-internal \textit{only}, negation and definiteness. The main thrust of this paper is to argue for an account of (\ref{scott}) and similar examples wherein \textit{only} presupposes existence and asserts existence, and \textit{the} is not the definite article at all, but a semantically vacuous determiner. The major advantage of this analysis compared to the one presented in \citet{cb2015} is that it does not require modification to Russell's or Frege's theories of definiteness, because it in essence implies that only \textit{NP}s are not definites at all.

The paper is organized as follows. Section \ref{sec:only-nps-english} expands on (\ref{scott}) to show how \textit{only} NPs pose a problem for definiteness in English. Section \ref{sec:edinstvennyj} gives background on determinacy and DP-internal \textit{only} in Russian, and section \ref{sec:only-nps-russian} shows that \textit{only} NPs are similarly recalcitrant for definiteness in Russian. Section \ref{sec:my-theory} presents my analysis of the semantics of \textit{only} NPs. Section \ref{sec:coppock-beaver} compares my analysis to that of \citet{cb2015}. Section \ref{sec:two-onlys} examines the DP-internal and DP-external uses of the word \textit{only}. Section \ref{sec:conclusion} concludes the paper.