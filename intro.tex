\section{Introduction \label{sec:intro}}
Suppose, counterfactually, that the novel \textit{Waverley} was written by a committee comprising Scott, Macfarlane, and Campbell. In such circumstances, one could truthfully utter (\ref{scott}).

\begin{exe}
	\ex \label{scott} Scott is not the only author of \textit{Waverley}.\\
	\hspace*{\fill} \citep{cb2015}
\end{exe}

What does the definite description \textit{the only author of Waverley} in (\ref{scott}) denote? If, as the sentence entails, \textit{Waverley} has multiple authors, then surely it does not denote anything at all. But in conventional theories of definiteness, an expression of the form \textit{the P} cannot be used unless a single individual satisfies the predicate \textit{P}.\footnote{I assume a theory of definiteness in the vein of \citet{frege} or \citet{strawson50}, in which the definite articles presupposes existence and uniqueness. \citet{russell}, another influential theory of definiteness, has existencen and uniqueness as assertions rather than presuppositions. Familiarity theories of definiteness, which describe the definite article's licensing conditions in different terms, will be addressed briefly in section \ref{sec:coppock-beaver}. See \citet[chap. 1]{schwarz09} and \citet{horn-abbott-2012} for overviews of the historical development of theories of definiteness.} (\ref{scott}) fails to meet this requirement, but is nonetheless a grammatical sentence of English.

The main thrust of this paper is to argue for an account of (\ref{scott}) and similar examples wherein \textit{only} presupposes existence and asserts uniqueness, and \textit{the} is not the definite article at all, but a semantically vacuous determiner. The major advantage of this analysis compared to \citegen{cb2015} is that it is compatible with Frege's theory of definiteness. I bolster my account of \textit{the} and \textit{only} with data from Russian.

The paper is organized as follows. Section \ref{sec:only-nps-english} reviews the data presented in \citet{cb2015} on DP-internal \textit{only} and definiteness. Section \ref{sec:only-nps-russian} shows that Russian patterns similarly to English with regard to \textit{only} and definiteness. Section \ref{sec:my-theory} presents my analysis of the semantics of \textit{only} NPs. Section \ref{sec:coppock-beaver} compares my analysis to that of \citet{cb2015}. Section \ref{sec:edinstvennyj} describes the semantics of DP-internal \textit{only} in Russian in more depth. Section \ref{sec:two-onlys} examines the adjectival and adverbial uses of the word \textit{only}. Section \ref{sec:conclusion} concludes the paper.