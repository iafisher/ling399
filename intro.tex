\section{Introduction \label{sec:intro}}
The definite article presupposes existence and uniqueness.\footnote{I assume a theory of definiteness in the vein of \citet{frege} or \citet{strawson50}. \citet{russell}, another influential theory of definiteness, has uniqueness as an assertion rather than a presupposition. Familiarity theories of definiteness describe the definite article's licensing conditions in different terms, and will be addressed briefly in section \ref{TODO}. See \citet[chap. 1]{schwarz09} and \citet{horn-abbott-2012} for an overview of the historical development of theories of definiteness.} That is, an expression of the form \textit{the P} may only be used when the predicate \textit{P} has a unique\footnote{To account for plural definites like \textit{the tables}, the concept of uniqueness can be generalized to maximality: \textit{the tables} may only refer to a maximal grouping of tables in a given context. Further discussion of plural definites is beyond the scope of discussion of this paper.} satisfier in the domain of discourse. If \textit{P} has no satisfiers or if it has more than one, then the expression \textit{the P} cannot be used.

There exist exceptions to this generalization, however. In particular, nominal phrases containing the exclusive adjective \textit{only} in the scope of negation not only lack a uniqueness presupposition but in fact entail non-uniqueness \citep{cb2015}. (\ref{scott}) is an example. The use of the definite description \textit{the only author of Waverley} ought to require that there be a single individual with the ``only author of \textit{Waverley}'' property. But this is plainly not the case: if Scott is not the only author of \textit{Waverley}, then there must be multiple authors, meaning that uniqueness cannot be supposed by the definite description in (\ref{scott}).

\begin{exe}
	\ex \label{scott} Scott is not the only author of \textit{Waverley}.
\end{exe}

The main thrust of this paper is to argue for an account of (\ref{scott}) and similar examples wherein \textit{only} presupposes existence and asserts existence, and \textit{the} is not the definite article at all, but a semantically vacuous determiner. The major advantage of this analysis compared to the one presented in \citet{cb2015} is that it does not require modification to Frege's theory of definiteness. I also present data from Russian for a broader perspective on the semantics of \textit{only} NPs.

% TODO: Haven't defined only NPs at this point

The paper is organized as follows. Section \ref{sec:only-nps-english} expands on (\ref{scott}) to show how \textit{only} NPs are problematic for Frege's theory of definiteness. Section \ref{sec:only-nps-russian} shows that Russian \textit{only} NPs pattern similarly to English ones with regard to definiteness. Section \ref{sec:my-theory} presents my analysis of the semantics of \textit{only} NPs. Section \ref{sec:coppock-beaver} compares my analysis to that of \citet{cb2015}. Section \ref{sec:edinstvennyj} describes the semantics of DP-internal \textit{only} in more depth. Section \ref{sec:two-onlys} examines the adjectival and adverbial uses of the word \textit{only}. Section \ref{sec:conclusion} concludes the paper.