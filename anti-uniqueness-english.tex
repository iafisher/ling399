\section{Anti-uniqueness effects in English \label{sec:anti-uniqueness-english}}
Anti-uniqueness effects occur in English when a definite description containing DP-internal \textit{only} or another exclusive adjective (such as \textit{sole} or \textit{single}) is in the scope of negation. The data presented in this section is thanks to \citet{cb2012b, cb2015}.

\subsection{Predicative anti-uniqueness in English}
(\ref{scott}), repeated below, is the canonical example of anti-uniqueness in the predicate position.

\begin{exe}
	\exr{scott} Scott is not the only author of \textit{Waverley}.
\end{exe}

Suppose a fictional context where the novel \textit{Waverley} was written by a committee comprising Scott, Macfarlane and Campbell, in which case (\ref{scott}) would be a true utterance.

In such a context, the phrases \textit{author of Waverley} and \textit{only author of Waverley} denote the sets (\ref{def-author}) and (\ref{def-only-author}). In other words, \textit{author of Waverley} is the set of individuals who are authors of \textit{Waverley}, and \textit{only author of Waverley} is the set of individuals who are ``only authors'' of \textit{Waverley}. Since in this context there are by definition no ``only authors'', this set is empty.

\begin{exe}
	\ex \label{def-author} $\textit{author of Waverley} = \lbrace Scott, Macfarlane, Campbell \rbrace$
	\ex \label{def-only-author} $\textit{only author of Waverley} = \emptyset$
\end{exe}

But if the set denoted by \textit{only author of Waverley} is empty, then what could \textit{the only author of Waverley} denote? In conventional theories of definiteness, the use of a phrase \textit{the P} is only possible if there is a single, unique \textit{P} \citep{horn-abbott-2012}, or in the terminology of sets, if \textit{P} denotes a singleton set.\footnote{Plural definites like \textit{the authors} are outside of the scope of discussion of this thesis.}

In particular, \citet{frege} and \citet{strawson50} have uniqueness as a presupposition for definite descriptions, \citet{russell} has it as an assertion, and \citet{horn-abbott-2012} have it as an implicature.

All these accounts share uniqueness as an essential part of the meaning of the definite article. What is so surprising about the ``anti-uniqueness effects'' that (\ref{scott}) evinces is the absence, and in fact denial, of uniqueness in a definite description.

\subsection{Argumental anti-uniqueness in English}
As \citet{strawson50} observed, definite descriptions may be used predicatively, as in (\ref{scott}) above and (\ref{napoleon}) below, where \textit{the greatest French soldier} is not used to mention an individual but to attribute a property to Napoleon.

\begin{exe}
	\ex \label{napoleon} Napoleon was the greatest French soldier.
\end{exe}

Anti-uniqueness effects are not unique to predicative definites, however. They are also possible with definites in argument positions, as in (\ref{only-goal}). On its most prominent reading, (\ref{only-goal}) entails that multiple goals were scored, including one by Anna, so just as in (\ref{scott}) the description \textit{the only goal} cannot have a referent.

\begin{exe}
	\ex \label{only-goal} Anna didn't score the only goal.
\end{exe}

There is a simple diagnostic is available to test the determinacy of an argumental definite (and thus presence or absence of an anti-uniqueness effect). If the definite description fails to denote an individual, then it is not able to serve as the antecedent of a pronoun in a subsequent sentence. The contrast between (\ref{the-goal}) and (\ref{only-goal-multiple}) therefore testifies to the presence of an anti-uniqueness effect in (\ref{only-goal}).

\begin{exe}
	\ex \label{the-goal} Anna didn't score [ the goal ]_1. It_1 was an excellent strike.
	\ex \label{only-goal-multiple} Anna didn't score [ the only goal ]_1. \#It_1 was an excellent strike.
\end{exe}

Note that (\ref{only-goal}) is actually ambiguous between a reading where one goal was scored by someone other than Anna and a reading where multiple goals were scored, including one by Anna. Under the one-goal reading, \textit{the only goal} does have a referent and therefore should be able to be a pronoun's antecedent, while it should not be under the multiple-goals reading. (\ref{only-goal-ambig-one}) and (\ref{only-goal-ambig-multiple}) tease apart the two readings with additional context and validate the predictions.

\begin{exe}
	\ex \label{only-goal-ambig-one} One-goal reading: Anna didn't score [ the only goal ]$_1$, Maria did. It$_1$ was an excellent strike.
	\ex \label{only-goal-ambig-multiple} Multiple-goals reading: Anna didn't score [ the only goal ]$_1$, Maria also scored. \#It$_1$ was an excellent strike.
\end{exe}

The two readings correspond to two different scopes of negation. In the one-goal reading, negation takes wide scope over the entire VP \textit{score the only goal}. In the multiple-goals reading, negation takes narrow scope over the argument \textit{the only goal}, yielding an anti-uniqueness effect. The narrow scope of negation in (\ref{only-goal-ambig-multiple}) is evident in the fact that the verb \textit{score} is not interpreted as negated---Anna did score something, on this reading.

Only verbs of creation can induce argumental anti-uniqueness effects in English. When \textit{see} is substituted for \textit{score}, as in (\ref{see-only-goal}), then the referential use of \textit{the only goal} is forced; (\ref{see-only-goal}) can only mean that there was a single goal.\footnote{The multiple-goals reading is still possible with heavy emphasis on \textit{only}, as in: \begin{exe} \ex Anna didn't see the ONLY goal. There was more than one. \end{exe} This example will be discussed further in section \ref{sec:my-theory}.}

\begin{exe}
	\ex \label{see-only-goal} Anna didn't see the only goal (\# Vera saw one, too).
\end{exe}

The term ``verbs of creation'' in this context must be understood in a broad sense, because verbs like \textit{bring} that do not exactly involve acts of creation nonetheless induce argumental anti-uniqueness:

\begin{exe}
	\ex Anna didn't bring the only cake (Vera brought one, too).
\end{exe}

\citeauthor{cb2015} note that only verbs with the inference pattern in (\ref{verb-of-creation}) may induce argumental anti-uniqueness. (\ref{not-verb-of-creation}) shows that the pattern fails for \textit{see}, which is why (\ref{see-only-goal}) does not have an anti-unique reading.

\begin{exe}
	\ex \label{verb-of-creation} There were ten cakes, and then Anna brought one, making eleven.
	\ex \label{not-verb-of-creation} \# There were ten goals, and then Anna saw one, making eleven.\footnote{Some of the strangeness of (\ref{not-verb-of-creation}) could be mitigated by replacing \textit{goal} with \textit{bird} and imagining a context where Anna was on a bird-spotting trip. Such a usage of \textit{saw} involves a semantic notion of bringing something into the realm of existence, and crucially would allow for anti-uniqueness: \begin{exe} \ex Anna didn't see the only bird. Vera saw one, too. \end{exe} }
	% TODO: Footnote about how "seeing a bird" could complicate this pattern?
\end{exe}