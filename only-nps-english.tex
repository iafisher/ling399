\section{\textit{Only} NPs in English \label{sec:only-nps-english}}
Here and throughout the term ``\textit{only} NPs'' refers to nominal phrases containing the adjective \textit{only}, such as \textit{the only table} and \textit{the only chair}. \textit{Only} NPs are always definite in English.\footnote{An exception is the idiom \textit{an only child}.} Although \textit{only} NPs are definites, they do not always presuppose uniqueness. Consider (\ref{scott}).

% TODO: Make it clear that these are C+B's examples.

\begin{exe}
	\exr{scott} Scott is not the only author of \textit{Waverley}.
\end{exe}

Suppose a fictional context where the novel \textit{Waverley} was written by a committee comprising Scott, Macfarlane and Campbell, in which case (\ref{scott}) would be a true utterance. In such a context, the predicate \textit{author of Waverley} would have multiple satisfiers, and uniqueness would not be satisfied. If \textit{the only author of Waverley} were a regular definite, then the failure of the uniqueness presupposition would cause the entire sentence to have an undefined truth value. But (\ref{scott}) is a true sentence in the given circumstances, so there must not be a uniqueness presupposition attached to \textit{the only author of Waverley}.

Observe that it is crucially the words \textit{only} and \textit{not} which cancel the uniqueness presupposition. When either is removed, as in (\ref{scott-wo-only}) and (\ref{scott-wo-not}), then the definite description refers to a single, unique author.

\begin{exe}
	\ex \label{scott-wo-only} Scott is not the author of \textit{Waverley}.
	\ex \label{scott-wo-not} Scott is the only author of \textit{Waverley}.
\end{exe}

\textit{Only} NPs without a uniquenss presupposition are not restricted to the predicate position. (\ref{only-goal}) is an example of an argumental \textit{only} NP that does not presuppose uniqueness.

\begin{exe}
	\ex \label{only-goal} Anna didn't score the only goal.
\end{exe}

In a context where Anna's team scored multiple goals, and she scored one, then (\ref{only-goal}) would be true. But the existence of multiple goals means that uniqueness is not satisfied relative to the predicate \textit{goal}. Another way of putting it is that \textit{the only goal} in (\ref{only-goal}) does not denote an individual and thus is not of type $e$.

Note that (\ref{only-goal}) may also be used to mean that there was only goal, but Anna didn't score it. Context makes this reading more accessible:

\begin{exe}
	\ex Anna didn't score the only goal. Maria did.
\end{exe}

This reading of (\ref{only-goal})

\textit{Only} NPs in the scope of negation with a verb of creation are ambiguous between readings where uniqueness is and is not satisfied. Other verbs do not have the same effect. When \textit{see} is substituted for \textit{score}, as in (\ref{see-only-goal}), then the referential use of \textit{the only goal} is forced; (\ref{see-only-goal}) can only mean that there was a single goal.\footnote{The multiple-goals reading is still possible with heavy emphasis on \textit{only}, as in: \begin{exe} \ex Anna didn't see the ONLY goal. There was more than one. \end{exe} This example will be discussed further in section \ref{sec:my-theory}.}

\begin{exe}
	\ex \label{see-only-goal} Anna didn't see the only goal. \#Vera saw one, too.
\end{exe}

The term ``verbs of creation'' in this context must be understood in a broad sense, because verbs like \textit{bring} that do not exactly involve acts of creation nonetheless lack an existence entailment:

\begin{exe}
	\ex Anna didn't bring the only cake. Vera brought one, too.
\end{exe}

% TODO: Better phrase than "the relevant verbs"
\citet{cb2015} note that the relevant verbs all follow the inference pattern in (\ref{verb-of-creation}). (\ref{not-verb-of-creation}) shows that the pattern fails for \textit{see}, which is why (\ref{see-only-goal}) does has only a referential reading of \textit{the only goal}.

\begin{exe}
	\ex \label{verb-of-creation} There were ten cakes, and then Anna brought one, making eleven.
	\ex \label{not-verb-of-creation} \# There were ten goals, and then Anna saw one, making eleven.\footnote{The ungrammaticality of (\ref{not-verb-of-creation}) could be mitigated by replacing \textit{goal} with \textit{bird} and imagining a context where Anna was on a bird-spotting trip. But such a usage of \textit{saw} involves a notion of bringing something into the realm of first-hand knowledge, and, as predicted, would allow for a non-referential reading of \textit{the only bird}: \begin{exe} \ex Anna didn't see the only bird. Vera saw one, too. \end{exe} }
	% TODO: Didn't realize this but C+B make this point in their paper (pp. 415-16)
\end{exe}

I have described (\ref{scott}) and (\ref{only-goal}) in terms of a lack of uniqueness presupposition with negated \textit{only} NPs. \citet{cb2015} describe it instead as a lack of an existence presupposition. These descriptions are equivalent. They simply take uniqueness and existence relative to different sets. If a predicate \textit{P} has multiple satisfiers, then \textit{only P} must be empty (i.e., non-existent), because the existence of multiple entities that satisfy \textit{P} entails that there is no ``only P.'' For example, if \textit{author} denotes the set in (\ref{author-set}), then no author has the \textit{only author} property, so the set of \textit{only authors} is empty. Uniqueness is not satisfied with regard to the \textit{author set}, and existence is not satisfied with regard to the \textit{only author} set.

\begin{exe}
	\ex \label{author-set} \textit{author}: $\lbrace \textsc{Scott}, \textsc{Macfarlane}, \textsc{Campbell} \rbrace$
	\ex \textit{only author}: $\emptyset$
\end{exe}

As I will later argue that \textit{only} and not \textit{the} is the locus of uniqueness and existence, I describe the missing presupposition in terms of the set that \textit{only} combines with.

Theories of definiteness in the tradition of Frege have existence and uniqueness as entailments of the definite article. So regardless of whether one speaks of the lack of an existence or a uniqueness presupposition, negated \textit{only} NPs pose a clear problem for such theories.