\section{\textit{Only} NPs in English \label{sec:only-nps-english}}
\textit{Only} NPs are always definite in English, with the exception of the idiom \textit{an only child}. Although \textit{only} NPs are definites, they do not always carry a uniqueness presupposition. (\ref{scott}), repeated below, is a typical example of an \textit{only} NP that does not presuppose uniqueness.

% TODO: Make it clear that these are C+B's examples.

\begin{exe}
	\exr{scott} Scott is not the only author of \textit{Waverley}.
\end{exe}

Suppose a fictional context where the novel \textit{Waverley} was written by a committee comprising Scott, Macfarlane and Campbell, in which case (\ref{scott}) would be a true utterance. In such a context, the predicate \textit{author of Waverley} would have multiple satisfiers and uniqueness would not be satisfied. A sentence without a false presupposition does not have a defined truth value, but (\ref{scott}) is a true sentence, so the \textit{only} NP must not have a uniqueness presupposition.

Another way of putting it is that \textit{the only author of Waverley} in (\ref{scott}) is indeterminate: it does not denote an individual and thus is not of type $e$.

Observe that it is crucially the words \textit{only} and \textit{not} which cancel the uniqueness presupposition. When either is removed, as in (\ref{scott-wo-only}) and (\ref{scott-wo-not}), then the definite description refers to a single, unique author.

\begin{exe}
	\ex \label{scott-wo-only} Scott is not the author of \textit{Waverley}.
	\ex \label{scott-wo-not} Scott is the only author of \textit{Waverley}.
\end{exe}

Indeterminate \textit{only} NPs are not restricted to the predicate position. (\ref{only-goal}) is an example of an argumental \textit{only} NPs lacking a uniqueness presupposition.

\begin{exe}
	\ex \label{only-goal} Anna didn't score the only goal.
\end{exe}

In a context where Anna's team scored multiple goals, and she scored one, then (\ref{only-goal}) would be true. But the existence of multiple goals means that uniqueness is not satisfied relative to the predicate \textit{goal}, and \textit{the only goal} is indeterminate.

Note that (\ref{only-goal}) is actually ambiguous between a reading where one goal was scored by someone other than Anna and a reading where multiple goals were scored, including one by Anna. Under the one-goal reading, \textit{the only goal} does have a referent and therefore should be able to be a pronoun's antecedent, while it should not be under the multiple-goals reading. (\ref{only-goal-ambig-one}) and (\ref{only-goal-ambig-multiple}) use additional context to tease apart the two readings and validate the predictions.

\begin{exe}
	\ex \label{only-goal-ambig-one} \textbf{One-goal reading} \\ Anna didn't score [ the only goal ]$_1$, Maria did. It$_1$ was an excellent strike.
	\ex \label{only-goal-ambig-multiple} \textbf{Multiple-goals reading} \\ Anna didn't score [ the only goal ]$_1$, Maria also scored. \#It$_1$ was an excellent strike.
\end{exe}

Only verbs of creation cause \textit{only} NPs to shed their uniqueness presupposition. When \textit{see} is substituted for \textit{score}, as in (\ref{see-only-goal}), then the referential use of \textit{the only goal} is forced; (\ref{see-only-goal}) can only mean that there was a single goal.\footnote{The multiple-goals reading is still possible with heavy emphasis on \textit{only}, as in: \begin{exe} \ex Anna didn't see the ONLY goal. There was more than one. \end{exe} This example will be discussed further in section \ref{sec:my-theory}.}

\begin{exe}
	\ex \label{see-only-goal} Anna didn't see the only goal. \# Vera saw one, too.
\end{exe}

The term ``verbs of creation'' in this context must be understood in a broad sense, because verbs like \textit{bring} that do not exactly involve acts of creation nonetheless lack an existence entailment:

\begin{exe}
	\ex Anna didn't bring the only cake. Vera brought one, too.
\end{exe}

% TODO: Better phrase than "the relevant verbs"
\citet{cb2015} note that the relevant verbs all follow the inference pattern in (\ref{verb-of-creation}.) (\ref{not-verb-of-creation}) shows that the pattern fails for \textit{see}, which is why (\ref{see-only-goal}) does has only a referential reading of \textit{the only goal}.

\begin{exe}
	\ex \label{verb-of-creation} There were ten cakes, and then Anna brought one, making eleven.
	\ex \label{not-verb-of-creation} \# There were ten goals, and then Anna saw one, making eleven.\footnote{Some of the strangeness of (\ref{not-verb-of-creation}) could be mitigated by replacing \textit{goal} with \textit{bird} and imagining a context where Anna was on a bird-spotting trip. But such a usage of \textit{saw} involves a notion of bringing something into the realm of first-hand knowledge, and, as predicted, would allow for a non-referential reading of \textit{the only bird}: \begin{exe} \ex Anna didn't see the only bird. Vera saw one, too. \end{exe} }
	% TODO: Didn't realize this but C+B make this point in their paper (pp. 415-16)
\end{exe}

I have spoken of a lack of uniqueness presupposition with negated \textit{only} NPs. \citet{cb2015} describe it differently, as a lack of an existence presupposition. These descriptions are equivalent because they take uniqueness and existence relative to different sets. If \textit{P} does not satisfy uniqueness, then \textit{only P} must be empty (i.e., non-existent), because the existence of multiple entities that satisfy \textit{P} entails that there is no ``only P.'' If there are multiple tables, then there is no ``only table.'' Section \ref{sec:my-theory} will make it clearer why I have chosen different terminology.

Fregean theories of definiteness have existence and uniqueness as entailments of the definite article. So regardless of whether one wishes to talk about the lack of an existence or a uniqueness presupposition, negated \textit{only} NPs pose a clear problem for such theories.