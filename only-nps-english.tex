\section{\textit{Only} NPs in English \label{sec:only-nps-english}}
Here and throughout the term ``\textit{only} NPs'' refers to nominal phrases containing the adjective \textit{only}, such as \textit{the only table} and \textit{the only chair}. \textit{Only} NPs are almost always definite in English.\footnote{An exception is the idiom \textit{an only child}.} Although \textit{only} NPs are definites, they do not always presuppose uniqueness. Consider again (\ref{scott}).

\begin{exe}
	\exr{scott} Scott is not the only author of \textit{Waverley}.
\end{exe}

Given that (\ref{scott}) means that there are multiple authors of \textit{Waverley},\footnote{Ignoring the irrelevant equative reading of ``Scott is not the same person as the only author of \textit{Waverley}.''} uniqueness is not satisfied in the definite description \textit{the only author of Waverley}. If it were a regular definite, then the failure of the uniqueness presupposition would cause the sentence to have an undefined truth value. But (\ref{scott}) is a true sentence in the given circumstances, so \textit{the only author of Waverley} must not have a uniqueness presupposition.

Observe that it is \textit{only} and \textit{not} together which cancel the uniqueness presupposition. When either is removed, as in (\ref{scott-wo-only}) and (\ref{scott-wo-not}), then the definite description refers to a single, unique author.

\begin{exe}
	\ex \label{scott-wo-only} Scott is not the author of \textit{Waverley}.
	\ex \label{scott-wo-not} Scott is the only author of \textit{Waverley}.
\end{exe}

\textit{Only} NPs without a uniqueness presupposition are not restricted to the predicate position. In (\ref{only-goal}), also from \citet{cb2015}, an \textit{only} NP in the argument position similarly does not presuppose uniqueness. In a context where Anna's team scored multiple goals, and she scored one, then (\ref{only-goal}) would be true. But the existence of multiple goals means that uniqueness is not satisfied relative to the predicate \textit{goal}.

\begin{exe}
	\ex \label{only-goal} Anna didn't score the only goal.
\end{exe}

Another way of describing the phenomenon in (\ref{scott}) and (\ref{only-goal}) is that the \textit{only} NP does not denote an individual and is thus not of type $e$.

Note that (\ref{only-goal}) may also be used to mean that there was only one goal, which Anna didn't score it. Context makes this reading more accessible:

\begin{exe}
	\ex \label{only-goal2} Anna didn't score the only goal. Maria did.
\end{exe}

This referential reading of (\ref{only-goal}) is roughly the same as the one that arises when \textit{the goal} is substituted for \textit{the only goal}.

(\ref{only-goal}) and (\ref{only-goal2}) show that argumental \textit{only} NPs in the scope of negation are ambiguous between readings where uniqueness is and is not satisfied. Only verbs of creation induce this ambiguity. When \textit{see} is substituted for \textit{score}, as in (\ref{see-only-goal}), then the referential use of \textit{the only goal} is forced; (\ref{see-only-goal}) can only mean that there was a single goal.

\begin{exe}
	\ex \label{see-only-goal} Anna didn't see the only goal. \#Vera saw one, too.
\end{exe}

In fact, verbs of creation is an overly narrow characterization, because verbs like \textit{bring} that do not exactly involve acts of creation nonetheless lack an existence entailment:

\begin{exe}
	\ex Anna didn't bring the only cake. Vera brought one, too.
\end{exe}

\citet{cb2015} note that the verbs which cause \textit{only} NPs to entail non-uniqueness all follow the inference pattern in (\ref{verb-of-creation}).

\begin{exe}
	\ex \label{verb-of-creation} There were ten cakes, and then Anna brought one, making eleven.
\end{exe}
	
(\ref{not-verb-of-creation}) shows that the pattern fails for \textit{see}, which is why (\ref{see-only-goal}) does has only a referential reading of \textit{the only goal}.

\begin{exe}
	\ex \label{not-verb-of-creation} \# There were ten goals, and then Anna saw one, making eleven.\footnote{\citeauthor{cb2015} note that the ungrammaticality of (\ref{not-verb-of-creation}) could be mitigated by replacing \textit{goal} with \textit{bird} and imagining a context where Anna was on a bird-spotting trip. But such a usage of \textit{saw} involves a notion of bringing something into the realm of first-hand knowledge, and, as predicted, would allow for a non-referential reading of \textit{the only bird}: \begin{exe} \ex Anna didn't see the only bird. Vera saw one, too. \end{exe} }
\end{exe}

In summary, although \textit{only} NPs appear to be regular definites, they lack a uniqueness presupposition in negated sentences. Argumental \textit{only} NPs are ambiguous between a definite and indefinite (i.e., non-unique) reading with verbs of creation. With other verbs, they do not have an indefinite reading.

What I describe as a lack of a uniqueness presupposition is described instead as a lack of an existence presupposition by \citet{cb2015}. These descriptions are equivalent. They simply take uniqueness and existence relative to different sets. If a predicate \textit{P} has multiple satisfiers, then \textit{only P} must be empty (i.e., non-existent), because the existence of multiple entities that satisfy \textit{P} entails that there is no ``only P.'' For example, if \textit{author} denotes the set in (\ref{author-set}), then no author has the \textit{only author} property, so the set of \textit{only authors} is empty. Uniqueness is not satisfied with regard to the \textit{author set}, and existence is not satisfied with regard to the \textit{only author} set.

\begin{exe}
	\ex \label{author-set} \textit{author}: $\lbrace \textsc{Scott}, \textsc{Macfarlane}, \textsc{Campbell} \rbrace$
	\ex \textit{only author}: $\emptyset$
\end{exe}

As I will later argue that \textit{only} and not \textit{the} is the locus of uniqueness and existence, I describe the missing presupposition relative to the set that \textit{only} combines with.