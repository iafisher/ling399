\section{\textit{Only} NPs in English \label{sec:only-nps-english}}
Although \textit{only} NPs (that is, nominal phrases containing \textit{only}) appear to be definites,\footnote{With the exception of the idiom \textit{an only child}. The use of the indefinite article with \textit{only} is not productive: \begin{exe} \ex[*] {He is an only friend.} \end{exe}} they do not always meet the licensing conditions of the definite article. Consider (\ref{scott}) again.

\begin{exe}
	\exr{scott} Scott is not the only author of \textit{Waverley}.
\end{exe}

The terms \textit{author of Waverley} and \textit{only author of Waverley} denote the following sets in the scenario from the previous section:

\begin{exe}
	\ex \textit{author of Waverley}: $\lbrace \textsc{Scott}, \textsc{Macfarlane}, \textsc{Campbell} \rbrace$
	\ex \label{only-author-set} \textit{only author of Waverley}: $\emptyset$
\end{exe}

Someone has the property of being an ``only author \textit{Waverley}'' if they are an author of \textit{Waverley} and no one else is. The existence of multiple authors of \textit{Waverley} entails that there are no ``only authors,'' so the set in (\ref{only-author-set}) is empty. Hence the uniqueness presupposition of the definite article is not satisfied with regard to \textit{only author of Waverley},\footnote{\citet{cb2015} divide the presupposition of the definite article into two parts: existence and uniqueness. In those terms, it is the existence presupposition that is not satisfied in (\ref{scott}) because the set that directly combines with \textit{the} is empty.} which ought to denote a set of a single element if the definite article is to be used felicitously.

Observe that it is \textit{only} and \textit{not} together which cancel the uniqueness entailment. When either is removed, as in (\ref{scott-wo-only}) and (\ref{scott-wo-not}), then \textit{the (only) author} refers to a single, unique author.

\begin{exe}
	\ex \label{scott-wo-only} Scott is not the author of \textit{Waverley}.
	\ex \label{scott-wo-not} Scott is the only author of \textit{Waverley}.
\end{exe}

\textit{Only} NPs without a uniqueness entailment are not restricted to the predicate position. (\ref{only-goal}), also from \citet{cb2015}, shows such an \textit{only} NP in an argument position. In a context where Anna's team scored multiple goals, and she scored one, then (\ref{only-goal}) would be true. But the existence of multiple goals means that uniqueness is not satisfied.

\begin{exe}
	\ex \label{only-goal} Anna didn't score the only goal.
\end{exe}

Both (\ref{scott}) and (\ref{only-goal}) have secondary readings in which the \textit{only} NP does satisfy uniqueness. These readings can be made more prominent with context:

\begin{exe}
	\ex \label{scott2} Scott isn't the only author of \textit{Waverley}. Macfarlane is.
	\ex \label{only-goal2} Anna didn't score the only goal. Maria did.
\end{exe}

These referential readings are roughly the same as the ones that arise when \textit{the} is substituted for \textit{the only}:

\begin{exe}
	\ex Scott isn't the author of \textit{Waverley}. Macfarlane is.
	\ex Anna didn't score the goal. Maria did.
\end{exe}

(\ref{scott}) and (\ref{scott2}) and (\ref{only-goal}) and (\ref{only-goal2}) show that \textit{only} NPs in the scope of negation are ambiguous between readings where uniqueness is and is not satisfied. Only verbs of creation induce this ambiguity. When \textit{see} is substituted for \textit{score}, as in (\ref{see-only-goal}), then the unique reading of \textit{the only goal} is forced; (\ref{see-only-goal}) can only mean that there was a single goal.

\begin{exe}
	\ex \label{see-only-goal} Anna didn't see the only goal. \#Vera saw one, too.
\end{exe}

In fact, verbs of creation is an overly narrow characterization, because verbs like \textit{bring} that do not exactly involve acts of creation nonetheless lack a uniqueness entailment:

\begin{exe}
	\ex Anna didn't bring the only cake. Vera brought one, too.
\end{exe}

\citet{cb2015} note that the verbs which cause \textit{only} NPs to entail non-uniqueness all follow the inference pattern in (\ref{verb-of-creation}).

\begin{exe}
	\ex \label{verb-of-creation} There were ten cakes, and then Anna brought one, making eleven.
\end{exe}
	
(\ref{not-verb-of-creation}) shows that the pattern fails for \textit{see}, which is why (\ref{see-only-goal}) has only a referential reading of \textit{the only goal}.

\begin{exe}
	\ex \label{not-verb-of-creation} \#There were ten goals, and then Anna saw one, making eleven.\footnote{The ungrammaticality of (\ref{not-verb-of-creation}) could be mitigated by replacing \textit{goal} with \textit{bird} and imagining a context where Anna was on a bird-spotting trip. But such a usage of \textit{saw} involves a notion of bringing something into the realm of first-hand knowledge, and, as predicted, would allow for a non-unique reading of \textit{the only bird}: \begin{exe} \ex Anna didn't see the only bird. Vera saw one, too. \end{exe} }
\end{exe}

A sentence whose presuppositions are not met should not have a defined truth value. But since (\ref{scott}) and (\ref{only-goal}) are true in the given circumstances, \textit{only} NPs must lack the uniqueness presupposition of regular definites.