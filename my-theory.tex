\section{A theory of \textit{only} NPs \label{sec:my-theory}}
In this section, I will present a theory to account for the data in sections \ref{sec:only-nps-english} and \ref{sec:only-nps-russian}, in which DP-internal \textit{only} presupposes existence and asserts uniqueness, and the lexical item \textit{the} that combines with \textit{only} in English is a semantically vacuous determiner.

\subsection{Existence presupposition}
Since the definite article\footnote{Although later I will argue that \textit{the} in \textit{the only} is not actually the definite article, here I assume a skeptical reader who has not yet been convinced of the point.} carries an existence presupposition itself, it is difficult to tease apart whether it is \textit{only} or \textit{the} which contributes the existence presupposition in examples where both words are present. Regardless, it is clear from the shared entailment of Scott's authorship in (\ref{exist-presup1}) and its negated counterpart (\ref{exist-presup2}) that definite descriptions with DP-internal \textit{only} have an existence presupposition.

\begin{exe}
	\ex \label{exist-presup1} Scott is the only author of \textit{Waverley}.
		\begin{xlist}
			\ex Entailment: Scott is an author of \textit{Waverley}.
			\ex Entailment: There are no other authors of \textit{Waverley}.
		\end{xlist}
	\ex \label{exist-presup2} Scott is not the only author of \textit{Waverley}.
		\begin{xlist}
			\ex Entailment: Scott is an author of \textit{Waverley}.
			\ex Entailment: There are other authors of \textit{Waverley}.
		\end{xlist}
\end{exe}

To say that DP-internal \textit{only} has an existence presupposition is a bit of a stipulation as it by definition cannot be separated from the definite article. There is stronger evidence that DP-internal \textit{only} has a uniqueness assertion, and if that is the case then the definite article would no longer be compatible with DP-internal \textit{only} since its uniqueness presupposition would clash with \textit{only}'s uniqueness assertion. If the definite article is out, then \textit{only} is the only word left which could plausibly carry the existence presupposition.

\subsection{Uniqueness assertion}
DP-internal \textit{only} asserts uniqueness. This fact is demonstrated first of all by the \textit{only} NPs in sections \ref{sec:only-nps-english} and \ref{sec:only-nps-russian} which do not entail uniqueness. Since uniqueness can be cancelled by negation in these examples, it cannot be a presupposition.

Additional evidence supports the point. In (\ref{green-roof-the}), the second speaker cannot felicitously challenge the uniqueness of \textit{house with a green roof}. In (\ref{green-roof-the-only}), the same exchange but with \textit{the only} instead of \textit{the}, the second speaker is free to challenge the uniqueness of the phrase's referent, indicating that uniqueness is at-issue and thus a semantic assertion.

\begin{exe}
	\ex \label{green-roof-the} Is it true that John lives in the house with a green roof? \\
	    - No, he lives next door. \\
	    - \#No, there are two houses with a green roof.
	\ex \label{green-roof-the-only} Is it true that John lives in the only house with a green roof? \\
	    - No, he lives next door. \\
	    - No, there are two houses with a green roof.
\end{exe}

Similar evidence comes directly from \citet{cb2015}:

% TODO: Think some more about Courtney's comments.
\begin{exe}
	\ex \#He's not the ambassador to Spain---there are two.
	\ex He's not the only ambassador to Spain---there are two.
\end{exe}

Only the uniqueness of the phrase with \textit{only} may be negated---a clear indication that \textit{the only ambassador to Spain} lacks a uniqueness presupposition.

\subsection{Contribution of \textit{the}}
In fact, the existence presupposition and uniqueness assertion are already presented in \citegen{cb2015} proposed logical form for \textit{only}, given below, although they do not identify the entailments in these terms in their paper.

\begin{exe}
	\ex \label{only-lf} \textit{only}: $ \lambda P . \lambda x . [ \partial(P(x)) \land \forall y [ x \ne y \to \neg P(y) ] ] $
\end{exe}

\citeauthor{cb2015} use \citegen{beaver92} partial operator $\partial$ to model presuppositions compositionally: $\partial(\phi)$ is true if $\phi$ is true and undefined otherwise. Notice that presupposing $P(x)$ amounts to presupposing existence, because at the end of derivation $x$ will be substituted for some concrete entity, say $a$, and presupposing $P(a)$ logically entails $\exists x . P(x)$. And the second conjunct is an assertion of the uniqueness of $x$ relative to the predicate $P$, so existence and uniqueness are already built in to \citeauthor{cb2015}'s definition and redundantly encoded in their theory by the \textsc{Iota} type-shift.

Where my proposal diverges from \citegen{cb2015} is in the role of \textit{the} in \textit{only} NPs. \citeauthor{cb2015} assert that it is just the definite article. This is why they must substantially change its semantics to fit the evidence. In my proposal, \textit{the} in \textit{only} NPs is not the definite article, but a semantically vacuous determiner.

This is not merely an \textit{ad hoc} stipulation. There is independent evidence that \textit{the} may be used in non-definite contexts. For example, the phrase \textit{the unicorn} in (\ref{unicorn}) has a kind reading rather than a definite one.

\begin{exe}
	\ex \label{unicorn} The unicorn is a rare beast.
\end{exe}

Certain expressions like \textit{read the newspaper} and \textit{take the bus} also do not seem to involve proper definites, as the referents of \textit{the newspaper} and \textit{the bus} need not be familiar or unique. They are essentially identical to saying \textit{read a newspaper} and \textit{take a bus}.

\citet{schwarz09} presents ``donkey sentences'' like (\ref{donkey}) in which the definite \textit{the donkey} does not refer to a particular donkey but instead co-varies with the indefinite NP in the antecedent clause.

\begin{exe}
	\ex \label{donkey} If a farmer owns a donkey, he beats the donkey.
\end{exe}

The existence of other examples where the word \textit{the} does not have the semantics of the definite article supports my hypothesis that \textit{the} in \textit{only} NPs is likewise not the definite article.

An important consequence of this claim is that \textit{only} NPs are not of type $e$. Since uniqueness is not a presupposition, \textit{the only P}, unlike \textit{the P}, is licensed even when \textit{P} has multiple elements. But what would \textit{the only P} denote in such circumstances? It cannot be the undefined individual, because such sentences have defined truth values.

In fact, it is natural to model the semantics of \textit{only} NPs with an existential quantifier. The formula in (\ref{scored-only-goal}) expresses that there is a goal which Anna scored and which is the unique goal in the situation (the universal quantification asserts that all other individuals in the domain of discourse are not goals, and therefore $x$ is the only goal).

\begin{exe}
	\ex \label{scored-only-goal} \textit{Anna scored the only goal}: \\ $\exists x . \textsc{Goal}(x) \land \textsc{Scored}(a, x) \land \forall y [ x \ne y \to \neg \textsc{Goal}(x) ] $
\end{exe}

(\ref{scored-only-goal}) is much closer to the semantics of \textit{a goal} than that of \textit{the goal}:

\begin{exe}
	\ex \label{a-goal} \textit{Anna scored a goal}: $\exists x . \textsc{Goal}(x) \land \textsc{Scored}(a, x)$
	\ex \label{the-goal} \textit{Anna scored the goal}: $\textsc{Scored}(a, \iota x . \textsc{Goal}(x))$
\end{exe}

So the theoretical claim that \textit{only} NPs are not of type $e$ agrees with the actual semantics of such phrases---\textit{the only P} is similar to \textit{a P} but with one additional condition, embodied by the last clause in (\ref{scored-only-goal}): uniqueness.

% TODO: May move this down one subsection?
An implication is that, in a compositional derivation, argumental \textit{only} NPs must undergo some kind of raising operation akin to that of indefinites.

\subsection{Empirical coverage}
Sections \ref{sec:only-nps-english} and \ref{sec:only-nps-russian} reached four main empirical conclusions:

\begin{exe}
	\ex \label{empirical1} \textit{Only} NPs lack a uniqueness presupposition in English.
	\ex \label{empirical2} Argumental \textit{only} NPs (with a verb of creation) are ambiguous in English between a determinate and indeterminate reading.
	\ex \label{empirical3} Argumental \textit{edinstvennyj} NPs must be indefinite when co-occurring with constituent negation, and must be definite otherwise (without extra semantic help).
	
	% TODO: This one could be clearer
	\ex \label{empirical4} Argumental \textit{edinstvennyj} NPs are unambiguous in Russian
\end{exe}

Let us see how my theory accounts for this data. (\ref{empirical1}) follows directly from my claim that \textit{only} NPs assert uniqueness.

\subsection{Compositional semantics}
% TODO: Better intro

For concreteness, I adopt the \citegen{cb2015} formula for \textit{only} (see (\ref{only-lf})) and the following formula for \textit{the}:

\begin{exe}
	\ex \textit{the} (when combining with \textit{only}): $\lambda P . P$ (type $\langle et, et \rangle$)
\end{exe}

The derivation in (\ref{author-lf})-(\ref{not-only-author-lf}) shows how the semantics of (\ref{scott}) are correctly accounted for under my theory.

\begin{exe}
	\ex \label{author-lf} \textit{author of Waverley}: $\lambda x . \textsc{Author}(x)$
	\ex \textit{only author of Waverley}: $\lambda x . \partial(\textsc{Author}(x)) \land \forall y [ x \ne y \to \neg \textsc{Author}(y) ]$
	\ex \textit{the only author of Waverley}: $\lambda x . \partial(\textsc{Author}(x)) \land \forall y [ x \ne y \to \neg \textsc{Author}(y) ]$
	\ex \textit{not the only author of Waverley}: \\ $\lambda x . \partial(\textsc{Author}(x)) \land \neg  \forall y [ x \ne y \to \neg \textsc{Author}(y) ]$
	\ex \textit{is not the only author of Waverley}: \\ $\lambda x . \partial(\textsc{Author}(x)) \land \neg  \forall y [ x \ne y \to \neg \textsc{Author}(y) ]$
	\ex \label{not-only-author-lf} \textit{Scott is not the only author of Waverley}: \\ $\partial(\textsc{Author}(s)) \land \neg \forall y [ s \ne y \to \neg \textsc{Author}(y) ]$
\end{exe}

% TODO: Necessary? Could be confusing because I talk about a different existential quantifier later
The universal quantifier in (\ref{not-only-author-lf}) can be re-written as an existential quantifier, to make more clear the assertion that there is another author of \textit{Waverley}:

\begin{exe}
	% TODO: Make sure this follows logically.
	\ex $\neg \forall y [ s \ne y \to \neg \textsc{Author}(y) ] \equiv \exists y [ s \ne y \land \textsc{Author}(y) ]$
\end{exe}

A compositional account of argumental \textit{only} NPs is somewhat more difficult. Since \textit{only} NPs are not of type $e$, they cannot compose directly with verbs in the way that regular definites and proper nouns do. Instead, they must undergo some kind of raising operation, akin to that of indefinites.

% TODO: Should unify this example with the only goals example.

Let's see how an \textit{only} NP in the argument position like in (\ref{yusuf}) would be interpreted.

\begin{exe}
	\ex \label{yusuf} Yusuf ate the only cookie.
\end{exe}

The meaning of (\ref{yusuf}) is roughly captured by the formula in (\ref{yusuf-lf}).

\begin{exe}
	% TODO: Think about whether the presupposition is correct here.
	\ex \label{yusuf-lf} $\exists x . \textsc{Ate}(y, x) \land \partial(\textsc{Cookie}(x)) \land \forall y [ x \ne y \to \neg \textsc{Cookie}(x) ]$
\end{exe}

In order for (\ref{yusuf-lf}) to be derivable from (\ref{yusuf}), \textit{the only cookie} must undergo some kind of raising operation---analogous to that postulated for indefinite NPs in argument positions. The raising operation would have to bring \textit{the only cookie} outside of the scope of negation, so that the negated version of the sentence in (\ref{yusuf2}) would receive the interpretation in (\ref{yusuf2-lf}) in which only $\textsc{Ate}(x, y)$ is negated, and not the presupposition or the uniqueness assertion.

\begin{exe}
	\ex \label{yusuf2} Yusuf didn't eat the only cookie.
	\ex \label{yusuf2-lf} $\exists x . \neg\textsc{Ate}(y, x) \land \partial(\textsc{Cookie}(x)) \land \forall y [ x \ne y \to \neg \textsc{Cookie}(x) ]$
\end{exe}

Raising could also account for the ambiguity between the two readings of argumental \textit{only} NPs in examples like (\ref{only-goal}), whose two readings are listed below.

\begin{exe}
	\ex \label{only-goal-ambig-one} \textbf{One-goal reading} \\ Anna didn't score the only goal. (Maria did.)
	\ex \label{only-goal-ambig-multiple} \textbf{Multiple-goals reading} \\ Anna didn't score the only goal. (Maria also scored.)
\end{exe}

(\ref{only-goal-ambig-one}) arises from a derivation where raising brings \textit{the only goal} and its existential quantifier out of the scope of negation:

\begin{exe}
	\ex $\exists x . \neg \textsc{Score}(a, x) \land \partial(\textsc{Goal}(x)) \land \forall y [ x \ne y \to \neg \textsc{Goal}(x) ]$
\end{exe}

(\ref{only-goal-ambig-multiple}) arises from a derivation where negation takes scope over the existential quantifier:

\begin{exe}
	\ex $\neg \exists x . $
\end{exe}

The necessity of a raising operation is not surprising given the evidence that such phrases cannot be of type $e$. \citet{cb2015} also posited a raising operation for the interpretation of argumental \textit{only} NPs, so the caveat is not even unique to my theory.

\subsection{Application to Russian}
Recall from section \ref{sec:only-nps-russian} that Russian shows a pattern of determinate readings for \textit{edinstvennyj} NPs in affirmative sentences and indeterminate readings in equivalent negated sentences.

Since my analysis of \textit{the} in English is irrelevant for article-less Russian, the primary test of my theory is whether or not \textit{edinstvennyj} NPs have an existence presupposition and a uniqueness assertion.

% TODO: Why does Russian force indeterminacy even with verbs of reation