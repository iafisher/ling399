\section{A theory of \textit{only} NPs \label{sec:my-theory}}
Two approaches are possible to account for the absence of a uniqueness presupposition in (\ref{scott}) and (\ref{only-goal}): to loosen the uniqueness requirement for all definite descriptions, or to declare that the phrases with DP-internal \textit{only} are not definites at all. \citeauthor{cb2015} adopt the former approach. In this section, I will present a theory along the latter lines, in which DP-internal \textit{only} presupposes existence and asserts uniqueness, and the lexical item \textit{the} that combines with \textit{only} is a semantically vacuous determiner.

\subsection{Existence presupposition}
Since the definite article\footnote{Although later I will argue that \textit{the} in \textit{the only} is not actually the definite article, here I assume a skeptical reader who has not yet been convinced of the point.} carries an existence presupposition itself, it is difficult to tease apart whether it is \textit{only} or \textit{the} which contributes the existence presupposition in examples where both words are present. Regardless, it is clear from the shared entailment of Scott's authorship in (\ref{exist-presup1}) and its negated counterpart (\ref{exist-presup2}) that definite descriptions with DP-internal \textit{only} have an existence presupposition.

\begin{exe}
	\ex \label{exist-presup1} Scott is the only author of \textit{Waverley}.
		\begin{xlist}
			\ex Entailment: Scott is an author of \textit{Waverley}.
			\ex Entailment: There are no other authors of \textit{Waverley}.
		\end{xlist}
	\ex \label{exist-presup2} Scott is not the only author of \textit{Waverley}.
		\begin{xlist}
			\ex Entailment: Scott is an author of \textit{Waverley}.
			\ex Entailment: There are other authors of \textit{Waverley}.
		\end{xlist}
\end{exe}

To say that DP-internal \textit{only} has an existence presupposition is a bit of a stipulation as it by definition cannot be separated from the definite article. There is stronger evidence that DP-internal \textit{only} has a uniqueness assertion, and if that is the case then the definite article would no longer be compatible with DP-internal \textit{only} since its uniqueness presupposition would clash with \textit{only}'s uniqueness assertion. If the definite article is out, then \textit{only} is the only word left which could plausibly carry the existence presupposition.

\subsection{Uniqueness assertion}
DP-internal \textit{only} asserts uniqueness. This fact is demonstrated first of all by the \textit{only} NPs in sections \ref{sec:only-nps-english} and \ref{sec:only-nps-russian} which do not entail uniqueness. Since uniqueness can be cancelled by negation in these examples, it cannot be a presupposition.

Additional evidence supports the point. In (\ref{green-roof-the}), the second speaker cannot felicitously challenge the uniqueness of \textit{house with a green roof}. In (\ref{green-roof-the-only}), the same exchange but with \textit{the only} instead of \textit{the}, the second speaker is free to challenge the uniqueness of the phrase's referent, indicating that uniqueness is at-issue and thus a semantic assertion.

\begin{exe}
	\ex \label{green-roof-the} Is it true that John lives in the house with a green roof? \\
	    - No, he lives next door. \\
	    - \#No, there are two houses with a green roof.
	\ex \label{green-roof-the-only} Is it true that John lives in the only house with a green roof? \\
	    - No, he lives next door. \\
	    - No, there are two houses with a green roof.
\end{exe}

Similar evidence comes directly from \citet{cb2015}:

% TODO: Think some more about Courtney's comments.
\begin{exe}
	\ex \#He's not the ambassador to Spain---there are two.
	\ex He's not the only ambassador to Spain---there are two.
\end{exe}

Only the uniqueness of the phrase with \textit{only} may be negated---a clear indication that \textit{the only ambassador to Spain} lacks a uniqueness presupposition.

\subsection{Contribution of \textit{the}}
In fact, the existence presupposition and uniqueness assertion are already presented in \citegen{cb2015} proposed logical form for \textit{only}, given below, although they do not identify the entailments in these terms in their paper.

\begin{exe}
	\ex \textit{only}: $ \lambda P . \lambda x . [ \partial(P(x)) \land \forall y [ x \ne y \to \neg P(y) ] ] $
\end{exe}

\citeauthor{cb2015} use \citegen{beaver92} partial operator $\partial$ to model presuppositions compositionally: $\partial(\phi)$ is true if $\phi$ is true and undefined otherwise. Notice that presupposing $P(x)$ amounts to presupposing existence, because at the end of derivation $x$ will be substituted for some concrete entity, say $a$, and presupposing $P(a)$ logically entails $\exists x . P(x)$. And the second conjunct is an assertion of the uniqueness of $x$ relative to the predicate $P$, so existence and uniqueness are already built in to \citeauthor{cb2015}'s definition and redundantly encoded in their theory by the \textsc{Iota} type-shift.

Where my proposal diverges from \citegen{cb2015} is in the role of \textit{the} in \textit{only} NPs. \citeauthor{cb2015} assert that it is just the definite article. This is why they must substantially change its semantics to fit the evidence. In my proposal, \textit{the} in \textit{only} NPs is not the definite article, but a semantically vacuous determiner.

This is not merely an \textit{ad hoc} stipulation. There is independent evidence that \textit{the} may be used in non-definite contexts. For example, the phrase \textit{the unicorn} in (\ref{unicorn}) has a kind reading rather than a definite one.

\begin{exe}
	\ex \label{unicorn} The unicorn is a rare beast.
\end{exe}

Certain expressions like \textit{read the newspaper} and \textit{take the bus} also do not seem to involve proper definites, as the referents of \textit{the newspaper} and \textit{the bus} need not be familiar or unique. They are essentially identical to saying \textit{read a newspaper} and \textit{take a bus}.

The existence of other examples where the word \textit{the} does not have the semantics of the definite article supports my hypothesis that \textit{the} in \textit{only} NPs is likewise not the definite article.

\subsection{Examples}
% TODO: Show derivation of predicative "the only"
For concreteness, I adopt the following formulae for \textit{only} and \textit{the}:

\begin{exe}
	% TODO: I've already given this formula.
	% Repeated in two-onlys.tex
	\ex \label{only-lf} \textit{only}: $ \lambda P . \lambda x . [ \partial(P(x)) \land \forall y [ x \ne y \to \neg P(y) ] ] $ (type $\langle et, et \rangle$)
	\ex \textit{the} (when combining with \textit{only}): $\lambda P . P$ (type $\langle et, et \rangle$)
\end{exe}

The derivation in (\ref{author-lf})-(\ref{only-author-lf}) shows how the semantics of (\ref{scott}) are correctly accounted for under my theory.

\begin{exe}
	\ex \label{author-lf} \textit{author of Waverley}: $\lambda x . \textsc{Author}(x)$
	\ex \textit{only author of Waverley}: $\lambda x . \partial(\textsc{Author}(x)) \land \forall y [ x \ne y \to \neg \textsc{Author}(y) ]$
	\ex \textit{the only author of Waverley}: $\lambda x . \partial(\textsc{Author}(x)) \land \forall y [ x \ne y \to \neg \textsc{Author}(y) ]$
	\ex \textit{is the only author of Waverley}: $\lambda x . \partial(\textsc{Author}(x)) \land \forall y [ x \ne y \to \neg \textsc{Author}(y) ]$
	\ex \label{only-author-lf} \textit{Scott is the only author of Waverley}: \\ $\partial(\textsc{Author}(s)) \land \forall y [ s \ne y \to \neg \textsc{Author}(y) ]$
\end{exe}

The derivation of \textit{the only author of Waverley} is as before.

\begin{exe}
	\ex \textit{not the only author of Waverley}: \\ $\lambda x . \partial(\textsc{Author}(x)) \land \neg  \forall y [ x \ne y \to \neg \textsc{Author}(y) ]$
	\ex \textit{is not the only author of Waverley}: \\ $\lambda x . \partial(\textsc{Author}(x)) \land \neg  \forall y [ x \ne y \to \neg \textsc{Author}(y) ]$
	\ex \label{not-only-author-lf} \textit{Scott is not the only author of Waverley}: \\ $\partial(\textsc{Author}(s)) \land \neg \forall y [ s \ne y \to \neg \textsc{Author}(y) ]$
\end{exe}

The universal quantifier in (\ref{not-only-author-lf}) can be re-written as an existential quantifier, to make more clear the assertion that there is another author of \textit{Waverley}:

\begin{exe}
	% TODO: Make sure this follows logically.
	\ex $\neg \forall y [ s \ne y \to \neg \textsc{Author}(y) ] \equiv \exists y [ s \ne y \land \textsc{Author}(y) ]$
\end{exe}

A theory in which \textit{only} does not presuppose uniqueness implies that \textit{only} NPs behave more similarly to indefinites than definites. In particular, an expression like \textit{the only cookie} in (\ref{yusuf}) cannot be of type $e$, because the absence of the presupposition uniqueness means that

\begin{exe}
	\ex \label{yusuf} Yusuf ate the only cookie.
\end{exe}

The interpretation of (\ref{yusuf}) would be along the lines of:

\begin{exe}
	% TODO: Think about whether the presupposition is correct here.
	\ex \label{yusuf-lf} $\exists x . \textsc{Ate}(y, x) \land \partial(\textsc{Cookie}(x)) \land \forall y [ x \ne y \to \neg \textsc{Cookie}(x) ]$
\end{exe}

In order for (\ref{yusuf-lf}) to be derivable from (\ref{yusuf}), \textit{the only cookie} must undergo some kind of raising operation---analogous to that postulated for indefinite NPs in argument positions. The raising operation would have to bring \textit{the only cookie} outside of the scope of negation, so that the negated version of the sentence in (\ref{yusuf2}) would receive the interpretation in (\ref{yusuf2-lf}) in which only $\textsc{Ate}(x, y)$ is negated, and not the uniqueness assertion.

\begin{exe}
	\ex \label{yusuf2} Yusuf didn't eat the only cookie.
	\ex \label{yusuf2-lf} $\exists x . \neg\textsc{Ate}(y, x) \land \partial(\textsc{Cookie}(x)) \land \forall y [ x \ne y \to \neg \textsc{Cookie}(x) ]$
\end{exe}

% TODO: Haven't stated that only NPs cannot be of type e.
The necessity of a raising operation is not surprising given the evidence that such phrases cannot be of type $e$.

% TODO: Do I have anything to say about verbs of creation?

% TODO: Need to say something about how my theory handles Russian