\section{A theory of \textit{only} NPs \label{sec:my-theory}}
In this section, I present a theoretical account of the data in sections \ref{sec:only-nps-english} and \ref{sec:only-nps-russian}. I argue that DP-internal \textit{only} presupposes existence and asserts uniqueness, and the lexical item \textit{the} in \textit{only} NPs in English is semantically vacuous.

Before I do so, I will briefly show that \citegen{heim82} familiarity theory of definiteness, a prominent alternative to uniqueness theories, also has trouble with the semantics of \textit{only} NPs. Familiarity theories locate the fundamental difference between indefinites and definites in novelty and familiarity: the definite article is used when the referent is familiar to both speaker and listener, and the indefinite article is used when it is familiar only to the speaker.

Does a familiarity theory of definiteness account for the behavior of \textit{only} NPs more successfully? In the exchange in (\ref{newcastle}), the referent of \textit{the only goal} is clearly not familiar to the speaker (hence why \textit{the goal} is not licensed), but it can nonetheless be used felicitously.

\begin{exe}
	\ex \label{newcastle}
	- What happened in the match this morning? \\
	- Not much. Newcastle scored the *(only) goal.
\end{exe}

While the implications of \textit{only} NPs on familiarity theories of definiteness deserve a fuller treatment than I can give in this paper, (\ref{newcastle}) indicates that they would likewise have difficulty handling the semantics of \textit{only} NPs.

\subsection{Existence presupposition}
Since the definite article\footnote{Although later I will argue that \textit{the} in \textit{the only} is not actually the definite article, here I assume a skeptical reader who has not yet been convinced of the point.} itself presupposes uniqueness and thus existence (a weaker condition than uniqueness), it is difficult to tease apart whether it is \textit{only} or \textit{the} which contributes the existence presupposition to \textit{only} NPs. Regardless, the shared entailment of Scott's authorship in (\ref{exist-presup1}) and its negated counterpart (\ref{exist-presup2}) shows that \textit{only} NPs do presuppose existence, because both sentences entail that there is at least one author of \textit{Waverley}.

\begin{exe}
	\ex \label{exist-presup1} Scott is the only author of \textit{Waverley}.
		\begin{xlist}
			\ex Entailment: Scott is an author of \textit{Waverley}.
			\ex Entailment: There are no other authors of \textit{Waverley}.
		\end{xlist}
	\ex \label{exist-presup2} Scott is not the only author of \textit{Waverley}.
		\begin{xlist}
			\ex Entailment: Scott is an author of \textit{Waverley}.
			\ex Entailment: There are other authors of \textit{Waverley}.
		\end{xlist}
\end{exe}

Besides preservation under negation, another characteristic property of presuppositions is that failure to satisfy them causes the sentence as a whole to have an undefined truth value. Indeed, it would be strange (and not merely false) to utter either (\ref{exist-presup1}) or (\ref{exist-presup2}) if there were no authors of \textit{Waverley}. Along the same lines, (\ref{macron}) is no better with \textit{the only} than with just \textit{the}.

\begin{exe}
	\ex \label{macron} \#Macron met with the (only) King of France today.
\end{exe}

The evidence that DP-internal \textit{only} alone has an existence presupposition without the help of \textit{the} is not conclusive. There is stronger evidence that DP-internal \textit{only} has a uniqueness assertion, and if that is the case then the definite article would no longer be compatible with \textit{only} since \textit{the}'s uniqueness presupposition would clash with \textit{only}'s uniqueness assertion. If the definite article is out, then \textit{only} is the only word left to carry the existence presupposition.

\subsection{Uniqueness assertion}
DP-internal \textit{only} asserts uniqueness. Affirmative sentences with \textit{only} NPs entail the uniqueness of the set that \textit{only} combines with, as shown by the second entailment of (\ref{exist-presup1}) that Scott is a unique author of \textit{Waverley}. Indeed, the primary purpose of using DP-internal \textit{only} is to underscore uniqueness: one would use (\ref{feckless-boris}) rather than (\ref{feckless-boris2}) in order to emphasize that there is only one computer in the situation.

\begin{exe}
	\ex \label{feckless-boris} Boris broke the only computer.
	\ex \label{feckless-boris2} Boris broke the computer.
\end{exe}

That the uniqueness entailment is an assertion and not a presupposition is demonstrated first of all by the \textit{only} NPs in sections \ref{sec:only-nps-english} and \ref{sec:only-nps-russian} which do not entail uniqueness. Since uniqueness is not preserved under negation in these examples, it must not be a presupposition.

Additional evidence supports the point. Consider the exchanges in (\ref{green-roof-the}) and (\ref{green-roof-the-only}).

\begin{exe}
	\ex \label{green-roof-the} Is it true that John lives in the house with a green roof? \\
	    - No, he lives next door. \\
	    - \#No, there are two houses with a green roof.
	\ex \label{green-roof-the-only} Is it true that John lives in the only house with a green roof? \\
	    - No, he lives next door. \\
	    - No, there are two houses with a green roof.
\end{exe}

In (\ref{green-roof-the}), the second speaker cannot felicitously challenge the uniqueness of \textit{house with a green roof}. In (\ref{green-roof-the-only}), the same exchange but with \textit{the only} instead of \textit{the}, the second speaker is free to do so, indicating that uniqueness is at-issue and thus a semantic assertion.

Similar evidence comes directly from \citet{cb2015}:

\begin{exe}
	\ex \#He's not the ambassador to Spain---there are two.
	\ex He's not the only ambassador to Spain---there are two.
\end{exe}

Only the uniqueness of the phrase with \textit{only} may be negated---a clear sign that \textit{the only ambassador to Spain} asserts uniqueness instead of presupposing it.

The existence presupposition and uniqueness assertion are already present in \citegen{cb2015} proposed logical form for \textit{only}, given below.

\begin{exe}
	\ex \label{only-lf} \textit{only}: $ \lambda P . \lambda x . [ \partial(P(x)) \land \forall y [ x \ne y \to \neg P(y) ] ] $
\end{exe}

\citeauthor{cb2015} use \citegen{beaver92} partial operator $\partial$ to model presuppositions compositionally: $\partial(\phi)$ is true if $\phi$ is true and undefined otherwise. Notice that presupposing $P(x)$ amounts to presupposing the existence of a satisfier of the predicate $P$. And the second conjunct is an assertion of the uniqueness of $x$ relative to the predicate $P$, so existence and uniqueness are already built in to (\ref{only-lf}).

\subsection{Contribution of the determiner \textit{the}}
Where my proposal diverges from \citegen{cb2015} is in the role of \textit{the} in \textit{only} NPs. \citeauthor{cb2015} assume that it is the same definite article as in any other definite description. In my proposal, \textit{the} in \textit{only} NPs is a determiner with no semantic content. Its purpose is solely to satisfy the syntactic requirement in English that singular count nouns must have a determiner.

There is independent evidence that \textit{the} may be used in non-definite contexts, specifically with kind readings, covariation with an indefinite, and pseudo-indefinite readings.

An example of a kind reading with \textit{the} is in (\ref{snow-leopard}). The phrase \textit{the snow leopard} does not refer to a specific snow leopard, but to snow leopards as a species.

\begin{exe}
	\ex \label{snow-leopard} The snow leopard is a solitary creature.
\end{exe}

\citet{schwarz09} presents ``donkey sentences'' like (\ref{donkey}) in which \textit{the donkey} does not refer to a particular donkey but instead co-varies with the indefinite NP in the antecedent clause.

\begin{exe}
	\ex \label{donkey} If a farmer owns a donkey, he beats the donkey.
\end{exe}

If \textit{the} in (\ref{donkey}) were the definite article, the \textit{the donkey} would have to denote a single donkey, contrary to the meaning of the sentence.

Certain expressions like \textit{read the newspaper} and \textit{take the bus} also do not seem to involve proper definites, as the referents of \textit{the newspaper} and \textit{the bus} need not be unique (or familiar). They are essentially identical to the expressions \textit{read a newspaper} and \textit{take a bus}.

The existence of examples besides negated \textit{only} NPs in which the word \textit{the} does not have the semantics of the definite article lends credence to my hypothesis that \textit{the} in \textit{only} NPs is semantically vacuous.

An important consequence of this claim is that \textit{only} NPs are never of type $e$. Since uniqueness is not a presupposition, \textit{the only P}, unlike \textit{the P}, is licensed even when \textit{P} has multiple elements. But what would \textit{the only P} denote in such circumstances? It cannot be the undefined individual, because as we have seen in (\ref{scott}) and (\ref{only-goal}) such sentences have defined truth values. Rather, it must be something similar to the quantificational definite descriptions proposed by \citet{russell}, who modelled the definite article with an existential quantifier:\footnote{Russell's theory is subject to the well-known objection in \citet{strawson50} that it predicts that failure to satisfy the uniqueness condition of the definite article results in sentences that are false, not undefined as is actually the case.}

\begin{exe}
	\ex \textit{Bertie found the solution}: \\
	$\exists x . \textsc{Solution}(x) \land \forall y . [\textsc{Solution(y)} \to y = x] \land \textsc{Found}(b, x)$ \\ \\
	\hspace*{\fill} \citep[ex. 1$'$]{horn-abbott-2012}  % Right-justify citation
\end{exe}

In fact, it is natural to model the semantics of \textit{only} NPs as well with an existential quantifier. (\ref{only-goal}) can be translated into the formula in (\ref{scored-only-goal}).

\begin{exe}
	\ex \label{scored-only-goal} \textit{Anna scored the only goal}: \\ $\exists x . \partial(\textsc{Goal}(x)) \land \textsc{Scored}(a, x) \land \forall y [ x \ne y \to \neg \textsc{Goal}(x) ] $
\end{exe}

(\ref{scored-only-goal}) expresses that there is a goal, scored by Anna, which is the unique goal in the situation (the universal quantification term asserts that all other entities in the domain of discourse are not goals, and therefore $x$ is the only goal).

(\ref{scored-only-goal}) is much closer to the semantics of \textit{a goal} than that of \textit{the goal}:

\begin{exe}
	\ex \label{a-goal} \textit{Anna scored a goal}: $\exists x . \textsc{Goal}(x) \land \textsc{Scored}(a, x)$
	\ex \label{the-goal} \textit{Anna scored the goal}: $\textsc{Scored}(a, \iota x . \textsc{Goal}(x))$
\end{exe}

So the theoretical claim that \textit{only} NPs are not of type $e$ agrees with the actual semantics of such phrases---\textit{the only P} is similar in meaning to \textit{a P} but with an additional uniqueness condition, embodied by the last clause in (\ref{scored-only-goal}).

\subsection{Empirical coverage}
The evidence in sections \ref{sec:only-nps-english} and \ref{sec:only-nps-russian} can be distilled into four main empirical observations:

\begin{exe}
	\ex \label{empirical1} \textit{Only} NPs do not presuppose uniqueness in English.
	\ex \label{empirical2} Argumental \textit{only} NPs (with verbs of creation) are ambiguous in English between unique and non-unique readings.
	\ex \label{empirical3} Argumental \textit{edinstvennyj} NPs must have non-unique readings when co-occurring with constituent negation, and must have unique readings otherwise (without extra semantic help---see section \ref{sec:edinstvennyj}).
	\ex \label{empirical4} Argumental \textit{edinstvennyj} NPs are never ambiguous in Russian.
\end{exe}

(\ref{empirical1}) follows directly from my claim that \textit{only} NPs assert uniqueness. Accounting for (\ref{empirical2}) takes a bit more work. Recall that the generalization arises from sentences like (\ref{only-goal}), repeated below.

\begin{exe}
	\exr{only-goal} Anna didn't score the only goal.
\end{exe}

(\ref{only-goal}) is ambiguous between a reading consistent with a single goal and a reading consistent with multiple goals. (\ref{only-goal-ambig-one}) and (\ref{only-goal-ambig-multiple}) tease apart the two readings with additional context, and spell out their truth conditions.

\begin{exe}
	\ex \label{only-goal-ambig-one} \textbf{One-goal reading} \\ Anna didn't score the only goal. Maria did.
	\begin{xlist}
		\ex Anna scored a goal.
		\ex There was only one goal.
	\end{xlist}

	\ex \label{only-goal-ambig-multiple} \textbf{Multiple-goals reading} \\ Anna didn't score the only goal. Maria also scored.
	\begin{xlist}
		\ex Anna scored a goal.
		\ex There was not only one goal.
	\end{xlist}
\end{exe}

From (\ref{only-goal-ambig-one}) and (\ref{only-goal-ambig-multiple}), it is clear that the two sentences differ only in whether the uniqueness assertion of \textit{the only goal} is negated.

An implication of the quantificational nature of \textit{only} NPs noted in the previous section is that, in a compositional derivation, argumental \textit{only} NPs must undergo some kind of raising operation akin to that of argumental indefinites, since they cannot compose directly with verbs that expect an argument of type $e$.

If the \textit{only} NPs in (\ref{only-goal-ambig-one}) and (\ref{only-goal-ambig-multiple}) both must undergo raising, then the semantic ambiguity may be traced to a choice of different raising destinations, one outside of the scope of negation and the other inside. In (\ref{only-goal-ambig-one}), \textit{the only goal} has evidently raised outside of the scope of negation, so that its uniqueness assertion is not negated. In (\ref{only-goal-ambig-multiple}), \textit{the only goal} has raised to some lower position (perhaps adjoining to VP) in the scope of negation, so that its uniqueness assertion is negated. Thus the ambiguity of argumental \textit{only} NPs in English is a consequence of their quantificational nature which necessitates raising out of argument positions.

Predicative \textit{only} NPs like in (\ref{scott}) may also be ambiguous between a unique and non-unique reading. But unlike with argumental \textit{only} NPs, a similar ambiguity arises even with predicative definites that do not contain \textit{only}.

\begin{exe}
	\ex \label{napoleon} Napoleon is not the greatest French soldier.
\end{exe}

(\ref{napoleon}) may be used to mean either that Napoleon does not have the property of being the greatest French soldier, or that Napoleon is not the same individual as the one designated by the moniker ``the greatest French soldier.'' The former reading corresponds to the non-unique reading of \textit{only} NPs. The latter reading corresponds to the unique reading of \textit{only} NPs, in that \textit{the only P} is being used as a designator and not a predicate.

A raising explanation likewise holds for the ambiguity in Russian. Only argumental \textit{edinstvennyj} NPs under constituent negation are forced to be non-unique. Since the constituent negation particle is syntactically attached to the \textit{only} NP itself, it cannot raise out of negation; wherever it raises to, it will take the negation particle with it. It is not clear why no ambiguity arises with sentential negation in Russian, but whatever (presumably syntactic) reason there is that \textit{edinstvennyj} NPs must raise outside of the scope of sentential negation would also account for the observation in (\ref{empirical4}) that argumental \textit{edinstvennyj} NPs are never ambiguous as to a unique or non-unique reading. The ambiguity in English is due to raising possibilities that are apparently unavailable in Russian.

It was mentioned in section \ref{sec:only-nps-russian} that some idiolects of Russian allow reference to a \textit{ne edinstvennyj} NP. That is to say, some Russian speakers find the combination in (\ref{no-anti-unique}) grammatical, and similarly for (\ref{maria3}), (\ref{boris3}), (\ref{lena3}) and (\ref{olga3}).

\begin{exe}
	\ex \label{no-anti-unique} \begin{xlist}
		\ex \gll Anna posetila ne edinstvennuju lekciju, kotoruju pro\v{c}ital Xomskij, kogda byl v na\v{s}em universitete.\\
		Anna attended not only lecture which gave Chomsky when was at our university\\
		\glt `Anna went to one of the lectures Chomsky gave at our university.'
		
		\ex \gll Ona byla o lingvistike.\\
		it was about linguistics\\
		\glt `It was about linguistics.'
	\end{xlist}
\end{exe}

The issue is that \textit{edinstvennuju lekciju} `only lecture' still entails uniqueness in (\ref{no-anti-unique}), despite occurring under constituent negation, which cancels the uniqueness assertion for most speakers.

That it nonetheless can indicates that perhaps \textit{ne edinstvennuju lekciju} has undergone \citegen{partee86} \textsc{A} shift to become indefinite. In that case, \textit{ne edinstvennuju} would function as a complex determiner that asserts non-uniqueness, but the existence entailment would derive from \textsc{A} rather than \textit{edinstvennyj}. Since \textsc{A} would apply to the entire phrase, including \textit{ne}, it would be above the scope of negation and consequently would allow subsequent reference like any other bare indefinite nominal in Russian.

It is still mysterious why \textsc{A} would be available in (\ref{no-anti-unique}) for some speakers but not others. The speakers who accepted (\ref{no-anti-unique}) had much greater exposure to English than those who rejected them. There is no immediately clear reason why this correlation would hold, given that no equivalent construction to (\ref{no-anti-unique}) is possible in English. Further research is necessary to better understand this idiolect of Russian and its prevalence.

\subsection{Compositional semantics}
While a full compositional analysis of \textit{only} NPs is beyond the scope of this paper, I want to demonstrate that my theory is not inconsistent with a compositional semantics approach by showing the derivation of a simple sentence.

For concreteness, I adopt \citegen{cb2015} formula for \textit{only} (leaving out the presupposition for simplicity), and the following formula for \textit{the} when it combines with \textit{only}:

\begin{exe}
	\ex \textit{only}: $ \lambda P . \lambda x . \forall y [ x \ne y \to \neg P(y) ] $
	\ex \textit{the}: $\lambda P . \lambda x . P(x)$
\end{exe}

(\ref{author-lf})-(\ref{not-only-author-lf}) shows the derivation of (\ref{scott}).

\begin{exe}
	\ex \label{author-lf} \textit{author of Waverley}: $\lambda x . \textsc{Author}(x)$
	\ex \textit{only author of Waverley}: $\lambda x . \forall y [ x \ne y \to \neg \textsc{Author}(y) ]$
	\ex \textit{the only author of Waverley}: $\lambda x . \forall y [ x \ne y \to \neg \textsc{Author}(y) ]$
	\ex \textit{not the only author of Waverley}: $\lambda x . \neg  \forall y [ x \ne y \to \neg \textsc{Author}(y) ]$
	\ex \textit{is not the only author of Waverley}: $\lambda x . \neg  \forall y [ x \ne y \to \neg \textsc{Author}(y) ]$
	\ex \label{not-only-author-lf} \textit{Scott is not the only author of Waverley}: $\neg \forall y [ s \ne y \to \neg \textsc{Author}(y) ]$
\end{exe}

The second conjunct of (\ref{not-only-author-lf}) asserts that it is not the case that all individuals other than Scott are not authors of \textit{Waverley}, i.e. that there is an author of \textit{Waverley} who is not Scott. Together with the presupposition that Scott is an author of \textit{Waverley}, this formula correctly capture the semantics of (\ref{scott}) compositionally.