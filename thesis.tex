\documentclass{article}

\usepackage{verbatim}
\usepackage{natbib}
\bibliographystyle{linquiry2}
\usepackage{gb4e}

% Cite an author, e.g. Davidson's (1967)
\newcommand{\citegen}[1]{\citeauthor{#1}'s~(\citeyear{#1})}

\title{DP-internal \textit{only} and definiteness in English and Russian}
\author{Ian Fisher}
\date{October 26, 2018}

\begin{document}
\maketitle


\begin{abstract}
This thesis draft reviews several issues related to DP-internal \textit{only} and definiteness. It extends the work on definiteness done by \citet{cb2015, cb2012a, cb2012b} and tests some of their predictions against Russian. I show that anti-uniqueness effects occur in predicative constructions in Russian, but I fail to confirm the existence of argumental anti-uniqueness effects. I sketch an account of anti-uniqueness effects based on scalar implicature. I also take inventory of the syntactic and semantic properties of \textit{edinstvennyj} `(the) only' in comparison to a battery of similar words in English.
\end{abstract}


\section{Background}
Most treatments of definite descriptions hold that they entail the uniqueness of the entity they refer to as either a presupposition or as an assertion. There exist definite descriptions, however, which not only do not require uniqueness but are in fact incompatible with it. \citet{cb2015} observe that the definite description \textit{the only author of Waverley} in (\ref{scott}) is just such an anti-unique definite: the sentence entails that \textit{Waverley} was written by multiple authors, and consequently there can be no individual who fits the description \textit{the only author of Waverley}.

\begin{exe}
	\ex \label{scott} Scott is not the only author of \textit{Waverley}.
\end{exe}

Observe that it is crucially the word \textit{only} which instigates the anti-uniqueness effect. When \textit{only} is removed, as in (\ref{the-author}), then \textit{the author of Waverley} retains its referential status and the sentence becomes a simple negation of the identity of Scott and the author of \textit{Waverley}.

\begin{exe}
	\ex \label{the-author} Scott is not the author of \textit{Waverley}.
\end{exe}

DP-internal \textit{only} in the scope of negation is capable of stripping definite descriptions of their uniqueness presupposition. Following \citeauthor{cb2015}, this phenomenon will be termed an ``anti-uniqueness effect.''

The contrast between (\ref{scott}) and (\ref{the-author}) illustrates the surprising interaction of DP-internal \textit{only} and definiteness. The main thrust of this paper is to explore this interaction and its ramifications in Russian, and more broadly to account for the semantic properties of DP-internal \textit{only} in both English and Russian.

Before continuing, some basic typological facts about Russian should be established. Per \citet{chierchia98}, NPs in Russian may denote either predicates (type $\langle e, t \rangle$) or entities (type $e$), the same typological classification as English. However, since Russian lacks articles, bare count nominals are permitted in argument positions, provided they undergo covert type-shifting to acquire definite or indefinite import \`{a} la \citet{partee86}. This analysis is quite similar to \citeauthor{cb2015}'s analysis of English, and as such an exploration of the relationship between definiteness and DP-internal \textit{only} in Russian should prove especially fruitful.

\citeauthor{cb2015} distinguish between two related concepts: definiteness, a morphological feature which they argue signals a weak uniqueness presupposition in English; and determinacy, the property of denoting an individual (i.e., having type $e$). I adopt their terminology of ``determinate'' and ``definite'' for my analysis.

The paper is organized as follows: section \ref{sec:anti-uniqueness} presents the evidence for anti-uniqueness effects in Russian. Section \ref{sec:scalar} sketches an alternative account to \citeauthor{cb2015}'s of anti-uniqueness effects. Section \ref{sec:which-edin} presents an inventory of \textit{edinstvennyj}'s properties in comparison with an array of similar words in English, and tests \citeauthor{cb2015}'s prediction that indeterminate readings should be more widely available in Russian. Section \ref{sec:conclusion} concludes the paper.

% TODO: Write a footnote with my reservations about the term "adverbial only"



\section{Anti-uniqueness effects \label{sec:anti-uniqueness}}
\subsection{Anti-uniqueness effects in English}
% TODO: Some duplicate material needs to be removed from the introduction

Anti-uniqueness effects should not be confused with \citegen{strawson50} observation that definite descriptions need not be universally unique. (\ref{table-books}), for instance, surely does not require that only a single table exists in the world.

\begin{exe}
	\ex \label{table-books} The table is covered with books.
\end{exe}

(\ref{table-books}) requires that \textit{the table} be unique within a certain context, though it is compatible with a universe with multiple tables. (\ref{scott}), by contrast, is inherently incompatible with a situation with a unique author of \textit{Waverley}.

The difference, then, is between normal uses of definite descriptions, for which uniqueness is contextually limited, and anti-unique definite descriptions, for which uniqueness is outright forbidden.

\subsection{Anti-uniqueness effects in Russian}
\citeauthor{cb2015}'s first example of an anti-uniqueness effect was the predicative construction in (\ref{scott}). Definite descriptions can be predicative in Russian, as (\ref{pred-def}) shows.

\begin{exe}
	\ex \label{pred-def} \gll Dmitrij --- vysokij, simpatichnyj, i (samyj umnyj student vo vsyom universitete / *Boris).\\
	Dmitri {} tall cute and most smart student in all university {} Boris\\
	\glt `Dmitri is tall, cute and (the smartest student in the whole university / Boris).'
\end{exe}

Assuming that adjectives are of type $\langle e, t \rangle$ and proper names are of type $e$ in Russian, and that conjuncts must have the same semantic type, then the possibility of a definite description in (\ref{pred-def}) conjoining with an adjective, and the impossibility of a proper name doing so, indicates that definites can have type $\langle e, t \rangle$. The equivalent sentence without conjunction is grammatical (see (\ref{dmitri-boris})), so it is crucially the adjectival conjunction that renders the sentence with \textit{Boris} ungrammatical.\footnote{That is, the sentence is ungrammatical on an equative reading where \textit{Boris} has type $e$. It does have a grammatical reading where \textit{Boris} is taken to denote a set of properties associated with ``Boris-ness'', similarly to the usage below in English: \begin{exe} \ex He's such a Boris.\end{exe} Russian speakers may find the reading more accessible with a name like \textit{Putin} that is more easily given a property reading. Since this property-denoting interpretation of \textit{Boris} plausibly has type $\langle e, t \rangle$, its grammaticality supports my assertion.} Note that the superlative \textit{samyj umnyj student} `smartest student' was used to force the definite interpretation, since superlatives cannot be indefinite but regular bare nominals can be.

\begin{exe}
	\ex \label{dmitri-boris} \gll Dmitrij --- (samyj umnyj student vo vsyom universitete / Boris).\\
	Dmitri {} most smart student in all university {} Boris\\
	\glt `Dmitri is (the smartest student in the whole university / Boris).'
\end{exe}

It has therefore been established that definite descriptions can be predicates in Russian. Do Russian predicative definites exhibit anti-uniqueness effects? (\ref{tolstoy}) indicates that they do.

\begin{exe}
	\ex \label{tolstoy} \gll Tolstoj ne edinstvennyj avtor \textit{Vojny i mira}\\
	Tolstoy not only author \textit{War and Peace}\\
	\glt `Tolstoy is not the only author of \textit{War and Peace}.'
\end{exe}

(\ref{tolstoy}) has the same meaning as its English translation. It presupposes that Tolstoy is an author of \textit{War and Peace} and entails that one or more others are also authors. Therefore, \textit{edinstvennyj avtor Vojny i mira} `the only author of \textit{War and Peace}' fails to refer to an individual, just as in English, and an anti-uniqueness effect arises.

Anti-uniqueness effects are evident with argumental as well as predicative definites in English. For example, (\ref{only-goal-multiple}) is only compatible (on the most prominent reading) with a situation in which multiple goals were scored, including one scored by Anna. In such a situation, the description \textit{the only goal} does not refer to anything, since the existence of multiple goals precludes any one goal being denoted as the only goal.

The second sentence of (\ref{the-goal})-(\ref{only-goal-one}) is a diagnosis for anti-uniqueness effects: if a definite description does not refer, it should not be able to serve as the antecedent of a pronoun. Therefore, when \textit{It was an excellent strike} is not a felicitous continuation, an anti-uniqueness effect must be operative in the initial sentence. (\ref{the-goal}) shows that regular definite descriptions do not exhibit anti-uniqueness effects, as expected.

\begin{exe}
	\ex \label{the-goal} Anna didn't score the goal. It was an excellent strike.
	\ex \label{only-goal-multiple} Anna didn't score the only goal. *It was an excellent strike. (multiple goals)
	\ex \label{only-goal-one} Anna didn't score the only goal. It was an excellent strike. (one goal)
\end{exe}

Note however that the first sentence of (\ref{only-goal-multiple}) is ambiguous between negation taking narrow or wide scope. (\ref{only-goal-multiple}) is the narrow-scope reading, in which negation only applies to \textit{the only goal}, to mean approximately ``Anna scored not the only goal.'' A wide scope reading, in which negation applies to the entire verb phrase, is also available, meaning ``It wasn't Anna that scored the only goal.'' This reading, shown in (\ref{only-goal-one}), requires a situation in which one goal was scored (by someone other than Anna), which means that anti-uniqueness should not be in force since there is indeed an \textit{only goal}. As predicted, the second sentence in (\ref{only-goal-one}) is grammatical under the wide-scope reading. The felicity of (\ref{only-goal-one}) becomes more apparent when one appends ``Maria did'' to the first sentence, to force wide scope.

In Russian, the two ambiguous readings of the English sentence correspond to two distinct sentences that differ as to the placement of the negation particle \textit{ne}. In (\ref{only-goal-one-ru}), the negation particle precedes the VP and thus takes wide scope, yielding the reading under which a single goal was scored. As expected, the continuation \textit{On byl otlichnyj udar} `It was an excellent strike' is grammatical.

\begin{exe}
	\ex \label{only-goal-one-ru} \gll Anna \textbf{ne} zabila edinstvennyj gol. On byl otlichnyj udar.\\
	Anna \textbf{not} scored only goal it was excellent strike\\
	\glt `Anna didn't score the only goal. It was an excellent strike.' (one goal)
	\ex \label{only-goal-multiple-ru} \gll Anna zabila \textbf{ne} edinstvennyj gol. On byl otlichnyj udar.\\
	Anna scored \textbf{not} only goal it was excellent strike\\
	\glt `Anna didn't score the only goal. It was an excellent strike.' (multiple goals)
\end{exe}

In (\ref{only-goal-multiple-ru}), the negation particle precedes the object and thus takes narrow scope, yielding the reading under which multiple goals were scored. In contrast to English, though, this reading does not result in an anti-uniqueness effect---the second sentence of (\ref{only-goal-multiple-ru}) is perfectly grammatical in Russian, with the pronoun \textit{on} `it' referring to the goal that Anna scored.

% TODO: Probably need to reorganize this section.

A similar pattern emerges for a number of other verbs of creation. \textit{Prochitat'} `to read' and \textit{prinesti} `to bring' compose with \textit{ne} and \textit{edinstvennyj} in the exact same way as \textit{zabit'} `to score' to yield two sentences corresponding to the ambiguous readings of a single English sentence, as (\ref{prochitat}) and (\ref{prinesti}) show.

\begin{exe}
	\ex \label{prochitat} \begin{xlist}
		\ex \gll Anna ne prochitala edinstvennuju lektsiju.\\
		Anna not read only lecture\\
		\glt `Anna didn't give the only talk.' (one talk)
		\ex Anna prochitala ne edinstvennuju lektsiju.\\
		Anna read not only lecture
		\glt `Anna didn't give the only talk.' (multiple talks)
	\end{xlist}
	\ex \label{prinesti} \begin{xlist}
		\ex \gll Denis ne prinjos edinstvennyj tort.\\
		Denis not brought only cake\\
		\glt `Denis didn't bring the only cake.'
		\ex Denis prinjos ne edinstvennyj tort.\\
		Denis brought not only cake
		\glt `Denis brought one of the cakes.'
	\end{xlist}
\end{exe}

\citeauthor{cb2015} note that analogous sentences without verbs of creation lack ambiguous readings of \textit{the only} phrases. (\ref{see-the-only}) entails the existence of a single talk, which Anna didn't see. The reading ``There were multiple talks, of which Anna saw one'' is not available.

\begin{exe}
	\ex \label{see-the-only} Anna didn't see the only talk.
\end{exe}

In Russian, the multiple-talks reading, unavailable in (\ref{see-the-only}) in English, can be expressed in the same way as with verbs of creation:

\begin{exe}
	\ex \begin{xlist}
		\ex \gll Anna ne posmotrela edinstvennuju lektsiju.\\
		Anna not watched only lecture\\
		\glt `Anna didn't see the only talk.'
		\ex Anna posmotrela ne edinstvennuju lektsiju.\\
		Anna watched not only lecture
		\glt `Anna saw one of the talks.'
	\end{xlist}
\end{exe}

Some speakers prefer \textit{eto} `this' to \textit{on} `it' in the second sentences of (\ref{only-goal-one-ru}) and (\ref{only-goal-multiple-ru}). \textit{Eto}, unlike \textit{on}, may refer to a semantic event \citep{davidson67}:

\begin{exe}
	\ex \label{davidson-ru} \gll Dzhon namazal maslo na khleb. On (eto / *ego) sdelal medlenno.\\
	John put butter on bread he this {} it did slowly\\
	\glt `John buttered the toast. He did it slowly.'
	\ex \label{davidson-ru-2} \gll Babushka umerla. (Eto sluchilos' / *on sluchilsja / *ono sluchilos') vchera.\\
	grandmother died this happened {} he happened {} it happened yesterday \\
	\glt `My grandmother died. It happened yesterday.'
\end{exe}

In (\ref{davidson-ru}) and (\ref{davidson-ru-2}), \textit{eto} but not \textit{on} or \textit{ono} `it' (and their shared accusative form \textit{ego}) may refer to the event variable introduced in the first sentence.

The diagnostic for anti-uniqueness effects relied on the inability of the definite description to be an antecedent, but it only works when no other antecedent is available (hence the use of the feminine \textit{Anna} in the first sentences of (\ref{only-goal-one-ru}) and (\ref{only-goal-multiple-ru}). Since \textit{eto} may refer to a semantic event, it is possible that it is the previous sentence's event variable and not the definite description that \textit{eto} in the continuation refers to, in which case an anti-uniqueness effect may still be operative.

However, note that, as the gloss of (\ref{davidson-ru}) suggests, \textit{it} may refer to semantic events in English. Even so, (\ref{only-goal-multiple}) is still ungrammatical; the first sentence apparently does not introduce an event variable, or if it does it is not felicitous to describe it as ``an excellent strike.'' It is plausible that (\ref{only-goal-multiple-ru}), the direct Russian translation of (\ref{only-goal-multiple}) similarly does not have a reading where \textit{eto} refers to a semantic event.

Regardless, the possibility of \textit{on} in (\ref{only-goal-one-ru}) and (\ref{only-goal-multiple-ru}) in at least some Russian idiolects is sufficient to support the assertion that certain constructions do not show anti-uniqueness effects when they ought to.

% TODO: Perhaps add something about how the inability of 'on' to refer to a semantic event allows us to make a stronger generalization than in English (or at least a more useful diagnostic).

The surprising lack of an anti-uniqueness effect in (\ref{only-goal-multiple-ru}) may be accounted for if the negation particle is actually inside the DP. In other words, the string \textit{ne edinstvennyj gol} is a DP headed by a null definite determiner, referring to the goal that Anna scored and entailing that other goals were scored as well. Since this DP has a referent, it is able to serve as the antecedent for the pronoun in the next sentence.

This account is not without issues. \textit{Edinstvennyj gol} presumably denotes a singleton set. The negation particle \textit{ne} normally encodes the set complement operation, which means that \textit{ne edinstvennyj gol} ought to denote a set of more than one element. But a non-singleton set should not be able to undergo the \textsc{Iota} shift to become determinate.

The difficulty is not so much a matter of an inexpressive theory as a true peculiarity in the facts of the Russian language in this instance. It is quite unusual that \textit{ne edinstvennyj gol} `the not-only goal' can simultaneously entail a multiplicity of goals while denoting a single one. What would be required in the semantics is some way of picking out one of the multiplicity of goals to refer to. This selection operation, which extracts a singular referent from a multiplicity, is constrained by the form of the rest of the sentence: `on' (or `eto' as the case may be) in (\ref{only-goal-multiple-ru}) may not refer to just any goal that was scored, but only the goal that Anna scored.

It is conceivable that there is some division of semantic meaning between presupposition and assertion which retains the multiplicity entailment while allowing \textit{ne edinstvennyj gol} to refer to a single goal. That is, the multiplicity entailment would be confined to the presupposition, and the assertive content would therefore contain only a reference to a singular goal. Unfortunately, the presuppositive component of (\ref{only-goal-multiple-ru}) cannot be isolated by negation, because the negated version of the sentence is independently ungrammatical:\footnote{Note that (\ref{only-goal-multiple-ru}) is an affirmative sentence, as the negation particle \textit{ne} does not have sentential scope.}

\begin{exe}
	\ex[*] { \label{double-neg}
		\gll Anna ne zabila ne edinstvennyj gol.\\
		Anna not scored not only goal\\
		\glt Intended: `Anna didn't score the not-only goal.'
	}
\end{exe}

The presumed meaning of (\ref{double-neg}) would be, as the rough English gloss suggests, that multiple goals were scored, and Anna didn't score any of them. In that case, the presupposition and assertion of (\ref{only-goal-multiple-ru}) would be as follows:

\begin{exe}
	\ex Anna zabila ne edinstvennyj gol. \begin{xlist}
		\ex Presupposition: Multiple goals were scored.
		\ex Assertion: Anna scored a goal.
	\end{xlist}
\end{exe}

The ability of \textit{ne edinstvennyj gol} to serve as a singular antecedent would be made less mysterious. % TODO: Finish this thought

A somewhat similar phenomenon is complement anaphora. Complement anaphora occurs when a pronoun has as its antecedent the complement set of some entity \citep{nouwen03, schwarz09}. In (\ref{kennedy}), the antecedent of \textit{they} in the second sentence is the complement of \textit{few congressmen}, i.e. few congressmen admire Kennedy so the majority of them think he is incompetent.

\begin{exe}
	\ex \label{kennedy} Few congressmen admire Kennedy. They think he's incompetent.
\end{exe}

Complement anaphora illustrate that the antecedent of a pronoun need not always be explicitly present in the semantics, as long as it can be derived from some entity that is present. Complement anaphora involve the set complement relationship; the data from Russian suggests that some individual-selection operation on sets may also be available.


\section{Scalar implicature account of anti-uniqueness effects \label{sec:scalar}}
The account of anti-uniqueness effects in \citet{cb2015, cb2012a} constitutes an expansive reimagining of definiteness in English. Three main ideas underlie their theory of the semantics of definiteness:

\begin{itemize}
	\item The definite and indefinite articles are identity functions, and the definite article carries a weak uniqueness presupposition.\footnote{Weak uniqueness is defined as uniqueness or non-existence, i.e. if uniqueness means $|P| = 1$ then weak uniqueness means $|P| \le 1$.}
	\item The determinacy of a given expression is established by covert type-shifting operations.
	\item Indeterminate interpretations of definites and determinate interpretations of indefinites are generally blocked by the principles of Maximize Presupposition and Type Simplicity.
\end{itemize}

However, anti-uniqueness effects are only ever observed with \textit{the only} phrases in the scope of negation in English, and their existence in Russian seems to be further limited to predicative contexts. It would make for a more perspicuous theory if the existence of anti-uniqueness effects could be attributed to some special properties of \textit{the only}, thus leaving the wider theory of definiteness unchanged.

One candidate for such a theory relates anti-uniqueness effects to the scalar properties of DP-internal \textit{only}. \citet{fauconnier75} proposed the idea of pragmatic scales to relate similar properties of superlatives and \textit{any}.

% TODO: Transition

Consider (\ref{scott-pos}) and (\ref{scott-neg}).

\begin{exe}
	\ex \label{scott-pos} Scott is the only author of \textit{Waverley}.
	\ex \label{scott-neg} Scott is not the only author of \textit{Waverley}.
\end{exe}

(\ref{scott-pos}) entails that the set denoted by \textit{author of Waverley} has a cardinality of exactly one. (\ref{scott-neg}) entails that the set has a cardinality greater than one. These sentences may be taken to induce a scale of $n$ authors:

\begin{itemize}
	\item 0 authors
	\item 1 author
	\item 2 authors
	\item etc.
\end{itemize}


In terms of this $n$ authors scale, (\ref{scott-pos}) restricts the scale to a single point and (\ref{scott-neg}) flips it to contain all greater points.

Note that the logical negation of $|\textsc{only author of Waverley}| = 1$ is
$|\textsc{only author of Waverley}| \ne 1$, or equivalently $|\textsc{only author of Waverley}| = 0 \lor |\textsc{only author of Waverley}| > 1$. But (\ref{scott-neg}) is not compatible with the cardinality equaling zero, so the zero case must be excluded somehow.

Here's how. Both (\ref{scott-pos}) and (\ref{scott-neg}) entail that Scott is an author of \textit{Waverley}, so that proposition must be a semantic presupposition. From this presupposition, it follows logically that the cardinality of the \textit{author of Waverley} set cannot be zero, since it must contain at least Scott. Thus, the proper scalar entailment is derived correctly after all.

What a scalar implicature account of anti-uniqueness effects, and indeed any account that preserves the conventional theory of definiteness, would require is that \textit{the only} is a semantic atom rather than the composition of \textit{the} and \textit{only}. If \textit{the only P} is derived compositionally as \textit{the} and \textit{only P}, then by principle the entire phrase is a definite description, since it is headed by a definite article, and its lack of a uniqueness presupposition would truly be evidence that the definite article cannot categorically have a uniqueness presupposition. On the other hand, \textit{the only} is a semantic atom, then \textit{the only P} need not be considered a definite description under all circumstances.

There is independent evidence that DP-internal \textit{only} may be different from adverbial \textit{only}, in which case the non-compositional analysis of \textit{the only} would be more plausible. We have seen that DP-internal \textit{only} is translated as \textit{edinstvennyj} in Russian. Adverbial \textit{only} corresponds to a different lexical item, \textit{tol'ko}:

\begin{exe}
	\ex \label{only-tolko} \gll \textbf{Tol'ko} studenty pri\v{s}li.\\
	\textbf{only} students came\\
	\glt `Only the students came.'
	\ex \label{only-edin} \gll Marija vzjala \textbf{edinstvennuju} knigu.\\
	Maria took \textbf{only} book.\\
	\glt `Maria took the only book.'
\end{exe}

The same lexical distinction is made in Spanish and German \citep{mcnally08} and Chinese (Shizhe Huang, p.c.). The contrast in Russian and the other languages raises the possibility that adverbial \textit{only} and DP-internal \textit{only}, though phonetically identical in English, are lexically distinct.

A lexical distinction between DP-internal and adverbial \textit{only} would make an atomic analysis of \textit{the only} more palatable. If both uses of \textit{only} were indeed the same lexical item, than to propose that \textit{the only} is semantically atomic would miss the generalization. % TODO: Clarify this idea

It is also possible that \textit{edinstvennyj} in fact corresponds to the entire complex determiner \textit{the only}, i.e. \textit{edinstvennyj} itself is the locus of the determinate import, rather than receiving its determinacy through \citegen{partee86} \textsc{Iota} type-shift, which is how bare nominals typically become determinate in Russian.

% TODO: Talk about only's other scalar properties



\section{The semantics of \textit{edinstvennyj} \label{sec:which-edin}}
A related but somewhat separate issue is the actual semantic meaning of \textit{edinstvennyj}, which clearly must be understood thoroughly if we are to have any hope of

A first step towards isolating its meaning is distinguishing which nominal modifier in English \textit{edinstvennyj} most exactly corresponds to, out of several that have a similar exclusive meaning: \textit{sole}, \textit{single}, and \textit{one}. (\ref{osso}), for instance, has the same meaning regardless of the choice of adjective.

\begin{exe}
	\ex \label{osso} The (only/sole/single/one) person to come was Ahmed.
\end{exe}

However, the various words evince distinct semantic and syntactic properties in other circumstances. \citet{cb2012b} catalog the inventory of properties thoroughly. Their conclusions are summarized in the table below.\\

\begin{tabular}{ l | l l l l l }
	& indefinite article & superlative & plural & NPIs & DP negation \\
	\hline
	\textit{only} & no & no & yes & yes & no \\
	\textit{sole} & yes & yes & yes & yes & yes \\
	\textit{single} & yes & yes & no & marginal & yes \\
	\textit{one} & no & marginal & no & yes & no \\
\end{tabular}

% Awkward way to force more space below table.
\ \\

The judgments in the table are shown by the sentences (\ref{osso-indef})-(\ref{osso-dp-neg}). In (\ref{osso-indef}), the exclusive adjectives are placed in a DP headed by an indefinite article. In (\ref{osso-super}), they combine with a superlative NP. In (\ref{osso-pl}), they combine with a plural NP. In (\ref{osso-npi}), they license or fail to license negative polarity items. In (\ref{osso-dp-neg}), they undergo DP negation.

\begin{exe}
	\ex \label{osso-indef} This company has a(n) (*only/sole/single/*one) director.
	\ex \label{osso-super} The oil spill was the (*only/sole/single/?one) worst environmental disaster in the state's history.
	\ex \label{osso-pl} They are the (only/sole/*single/*one) people we can trust.
	\ex \label{osso-npi} The (only/sole/??single/one) pick-up truck he ever owned was a Chevrolet.
	\ex \label{osso-dp-neg} Not a(n) (*only/sole/single/*one) person came.
\end{exe}

\subsection{Indeterminate \textit{edinstvennyj}}
% TODO: Move this intro to the end and replace with more appropriate intro.
\citet{cb2015} predict that indeterminate readings should be more widely available in article-less languages than in English. Preliminary evidence from Russian shows that NPs beginning with \textit{edinstvennyj} resist indeterminate interpretation. In (\ref{sole-director}), for instance, the preferred translation of the English sentence with the indefinite phrase \textit{a sole director} uses the regular cardinal number \textit{odin} `one' rather than \textit{edinstvennyj}, which is marginal.

\begin{exe}
	\ex \label{sole-director} \gll U etoj kompanii --- (odin/??edinstvennyj) direktor\\
	At this company {} (one/only) director\\
	\glt `This company has a sole director.'
\end{exe}

% TODO: Add or remove this
%Note that \textit{odin} is morphologically related to \textit{edinstvennyj}, whose root is \textit{edin}. Similarly, \textit{one} in English may be morphologically related to \textit{only}.

In (\ref{not-a-sole}), another example of indefinite \textit{sole} in English, the translation with \textit{edinstvennyj} is outright ungrammatical. \textit{Odin} must be used.

\begin{exe}
	\ex \label{not-a-sole} \gll Ni (odin/*edinstvennyj) chelovek ne pri\v{s}yol.\\
	Not one/only person not came\\
	\glt `Not a sole person came.'\footnote{Russian \textit{ni} is a negative concordance particle rather than double negation.}
\end{exe}

(\ref{sole-director}) and (\ref{not-a-sole}) indicate that indeterminate readings for \textit{edinstvennyj} are dispreferred if not outright impossible. There are at least two exceptions to this generalization, however. The first is that an indeterminate reading for \textit{edinstvennyj} can be achieved in the compound expression \textit{odin-edinstvennyj}:

\begin{exe}
	\ex \label{odin-edinstvennyj} \gll Vrachi rekomendovali odin-edinstvennyj podkhod.\\
	doctors recommended one-only approach\\
	`The doctors recommended one single approach.'
\end{exe}

The second is that \textit{edinstvennyj} can combine with \textit{rebyonok} `child' to mean `an only child' (i.e., a child with no siblings):

\begin{exe}
	\ex \label{only-child-ru} Marija --- edinstvennyj rebyonok.\\
	Maria {} only child
	\glt `Maria is an only child.'\footnote{As is generally the case with bare nominals in Russian, \textit{edinstvennyj rebyonok} also has a determinate reading, meaning `Maria is the only child.'}
\end{exe}

It is consistent with the other evidence to conclude that the source of the indefinite import of the NP in (\ref{odin-edinstvennyj}) is the numeral \textit{odin} rather than \textit{edinstvennyj}.

(\ref{only-child-ru}) is more problematic, as there is no other candidate for licensing the indeterminate reading. However, it is also true that \textit{an only child} but DP-internal \textit{only} cannot generally be indeterminate:

% TODO: Note though that 'edinstvennyj rebyonok' is only indeterminate insofar as other predicative DPs are indeterminate.

\begin{exe}
	\ex Examples (32)-(34) from \citet{cb2012a} \begin{xlist}
		\ex If the business is owned by a(n) sole/*only owner, only the owner is eligible to be the managing officer.
		\ex This company has a(n) sole/*only director.
		\ex There was a(n) sole/*only piece of cake left.
	\end{xlist}
\end{exe}

In English, \textit{only} can be indeterminate only when it combines with the noun \textit{child} (and derived nouns like \textit{grandchild}). (\ref{sole-director}) shows that it is the same case in Russian. Since DP-internal \textit{only} does not productively allow indeterminate readings, \textit{an only child} may be considered idiomatic in both languages and not indicative of the general properties of DP-internal \textit{only}.\footnote{It is nonetheless curious that the same idiom should surface in both languages. I have no comment on this coincidence at the moment.}

Thus, despite the two objections, the generalization remains that \textit{edinstvennyj} does not independently allow an indeterminate reading.

\subsection{Licensing of negative polarity items}
Both adverbial and DP-internal \textit{only} license negative polarity items in English:

\begin{exe}
	\ex *(Only) Khalid ever goes to the movies.
	\ex The *(only) poem I ever read in high school was ``The Raven.''
\end{exe}

DP-internal \textit{only} cannot license NPIs outside of its DP:

\begin{exe}
	\ex The only team that I had heard of (*ever) won the World Cup.
\end{exe}

The facts in Russian are more complicated. \textit{Chto-nibud'} `anything' is cited as an NPI by \citet{russneg}, and it is licensed by \textit{tol'ko}:

\begin{exe}
	\ex \gll Tol'ko Boris chto-nibud' delaet.\\
	only Boris anything does\\
	\glt `Only Boris is doing anything.'
\end{exe}

However, it is also licensed without \textit{tol'ko}, in a sentence in which \textit{anybody} would be ungrammatical in English:

\begin{exe}
	\ex \label{nibud-no-polarity} \gll Boris chto-nibud' delaet.\\
	Boris anything does\\
	\glt `Boris is doing something.' % TODO: This might not be the best translation of the sentence.
\end{exe}

\textit{Edinstvennyj} seems to prefer \textit{kto-libo} to \textit{kto-nibud'} (both of which are conventionally translated as anybody). (\ref{libo-vs-nibud}) is better with \textit{kto-libo}, though not completely ungrammatical with \textit{kto-nibud'}.

\begin{exe}
	\ex \label{libo-vs-nibud} \gll Ivan vzyal edinstvennuju knigu, kotoruju (kto-libo / ?kto-nibud') khotel.\\
	Ivan took only book which anybody {} anybody wanted\\
	\glt `Ivan took the only book that anybody wanted.'
\end{exe}

However, \textit{edinstvennyj} cannot license \textit{kto-nibud'} in a position outside of its DP:

\begin{exe}
	\ex \label{nibud-out-of-dp} \gll Edinstvennyj uchitel' vybral (kogo-to / *kogo-nibud').\\
	only teacher picked someone {} anyone\\
	\glt `The only teacher picked someone.'
\end{exe}

In (\ref{nibud-out-of-dp}), \textit{kto-to}	is a positive polarity item that is subject to Principle C of the Binding Theory \citep{russneg}.\footnote{The morphemes \textit{to}, \textit{nibud'} and \textit{libo} are suffixes which may attach to a number of pronouns, including \textit{chto} `what' (\textit{chto-to}, \textit{chto-nibud'}, \textit{chto-libo}) and \textit{kto} `who' (\textit{kto-to}, \textit{kto-nibud'}, \textit{kto-libo}). Only the underlying pronoun takes case endings, hence forms like \textit{kogo-to}, the genitive and accusative declension of \textit{kto-to}.}

The ungrammaticality of (\ref{nibud-out-of-dp}) suggests that, despite (\ref{nibud-no-polarity}), \textit{nibud'} phrases have some kind of polarity requirements that may not neatly fit the paradigm of positive versus negative polarity.

In summary, \textit{edinstvennyj} licenses \textit{libo} phrases, but it is not yet clear whether such phrases are negative polarity items or not.

\subsection{Other properties of \textit{edinstvennyj}}
The remaining properties of \textit{edinstvennyj} to be pinned down are its ability to combine with superlative NPs and plural NPs, and to undergo DP negation. (\ref{not-a-sole}) already showed that DP negation is impossible for \textit{edinstvennyj}. (\ref{plural-edin}) shows that \textit{edinstvennyj} may modify a plural NP.

\begin{exe}
	\ex \label{plural-edin} \gll Oni --- edinstvennye lyudi, kotorym ya doveryayu.\\
	they {} only people which I trust\\
	\glt `They are the only people that I trust.'
\end{exe}

The ungrammaticality of (\ref{super-edin}) demonstrates that \textit{edinstvennyj} cannot modify a superlative NP.

\begin{exe}
	\ex[*] { \label{super-edin} \gll
		Eto edinstvennyj samyj vysokiy neboskryob v Chikago.\\
		this only most tall skyscraper in Chicago\\
		\glt Intended: `It is the single tallest skyscraper in Chicago.'
	}
\end{exe}

\subsection{Summary}
The relevant semantic and syntactic properties of \textit{edinstvennyj} and its potential counterparts are thus:\\

\begin{tabular}{ l | l l l l l }
	& indefinite article & superlative & plural & NPIs & DP negation \\
	\hline
	\textit{only} & no & no & yes & yes & no \\
	\textit{sole} & yes & yes & yes & yes & yes \\
	\textit{single} & yes & yes & no & marginal & yes \\
	\textit{one} & no & marginal & no & yes & no \\
	\textit{edinstvennyj} & no & no & yes & unclear & no \\
\end{tabular}

% Awkward way to force more space below table.
\ \\

The properties of \textit{edinstvennyj} are most similar to those of DP-internal \textit{only}, and indeed may be identical, at least within the matrix under consideration, if \textit{edinstvennyj} turns out to be an NPI-licenser. The near-identity of \textit{only} and \textit{edinstvennyj} in a range of circumstances adds support to my comparison between the two words with regards to anti-uniqueness effects.



\section{Summary and further research \label{sec:conclusion}}
The Russian data reviewed in this paper argues for three main positions: that there is a possibility of adverbial and DP-internal \textit{only} being different lexical items in English; that anti-uniqueness effects exist in Russian, but not in (all) argumental positions; and that \textit{edinstvennyj} corresponds most closely to DP-internal \textit{only} out of the array of exclusive adjectives in English.

I have shown that a scalar implicature account of \textit{the only} constructions goes a little ways towards clarifying the issue of anti-uniqueness effects. The scalar approach remains promising but will require much further elucidation. It seems to me that an explanation that situates DP-internal \textit{only} as the locus of the anti-uniqueness effects is more desirable than one that requires a wholesale reevaluation of definiteness in English, as \citeauthor{cb2015}'s theory does.



\section*{Acknowledgements}
I am grateful to Alexandr Trubetskoy and Maxim Sonin for their help with the Russian data.



\bibliography{thesis}

\end{document}
