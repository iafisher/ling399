\documentclass{article}

\usepackage{verbatim}
\usepackage{natbib}
\bibliographystyle{linquiry2}
\usepackage{gb4e}

% Cite an author, e.g. Davidson's (1967)
\newcommand{\citegen}[1]{\citeauthor{#1}'s~(\citeyear{#1})}

\title{DP-internal \textit{only} and definiteness in English and Russian}
\author{Ian Fisher}
\date{October 26, 2018}

\begin{document}
\maketitle


\begin{abstract}
% TODO: Update this for Draft 1.

This thesis draft reviews several issues related to DP-internal \textit{only} and definiteness. It extends the work on definiteness done by \citeauthor{cb2015} (\citeyear{cb2012a}, \citeyear{cb2012b}, \citeyear{cb2015}) and tests some of their predictions against Russian. I show that anti-uniqueness effects occur in predicative constructions in Russian, but I fail to confirm the existence of argumental anti-uniqueness effects. I also argue that \textit{edinstvennyj} `(the) only' resists indeterminate interpretation.
\end{abstract}


\section{Background}
Most treatments of definite descriptions hold that they entail the uniqueness of the entity they refer to, either as a presupposition or as an assertion. There exist definite descriptions, however, which do not refer to any individual at all. \citet{cb2015} observe that the definite description \textit{the only author of Waverley} in (\ref{scott}) is just such a non-referential definite: the sentence entails that \textit{Waverley} was written by multiple authors, and consequently there can be no individual matching the description \textit{the only author of Waverley}.

\begin{exe}
	\ex \label{scott} Scott is not the only author of \textit{Waverley}.
\end{exe}

\citeauthor{cb2015} term instances where definite descriptions fail to refer as ``anti-uniqueness effects,'' and they develop a substantial new theory of definiteness to account for the existence of such effects.

Observe that it is crucially the word \textit{only} which instigates the anti-uniqueness effect. When \textit{only} is removed, as in (\ref{the-author}), then \textit{the author of Waverley} retains its referential status and the sentence becomes a simple negation of the identity of Scott and the author of \textit{Waverley}.

\begin{exe}
	\ex \label{the-author} Scott is not the author of \textit{Waverley}.
\end{exe}

(\ref{scott}) and (\ref{the-author}) illustrate the surprising interaction of DP-internal \textit{only} and definiteness. The main thrust of this paper is to explore this interaction and its ramifications in Russian, and more broadly to account for the semantic properties of DP-internal \textit{only} in both English and Russian.

Before continuing, some basic typological facts about Russian should be established. Per \citet{chierchia98}, NPs in Russian may denote either predicates (type $\langle e, t \rangle$) or entities (type $e$), the same typological classification as English. However, since Russian lacks articles, bare count nominals are permitted in argument positions, provided they undergo covert type-shifting to acquire definite or indefinite import \`{a} la \citet{partee86}. This analysis is quite similar to \citeauthor{cb2015}'s analysis of English, and as such an exploration of the relationship between definiteness and DP-internal \textit{only} in Russian should prove especially fruitful.

\citeauthor{cb2015} distinguish between two related concepts: definiteness, a morphological feature which they argue signals a weak uniqueness presupposition in English; and determinacy, the property of denoting an individual (i.e., having type $e$). I adopt their terminology of ``determinate'' and ``definite'' for my analysis.

The paper is organized as follows: section \ref{sec:t-e} reviews the two words in Russian whose English translation is \textit{only}. Section \ref{sec:anti-uniqueness} presents the evidence for anti-uniqueness effects in Russian. Section \ref{sec:indet-e} tests \citeauthor{cb2015}'s prediction that indeterminate readings should be more widely available in Russian. Section \ref{sec:conclusion} concludes the paper.

% TODO: Write a footnote with my reservations about the term "adverbial only"


\section{\textit{Tol'ko} and \textit{edinstvennyj} \label{sec:t-e}}
In English, the word \textit{only} may be used as an adverb or as an exclusive adjective inside a determiner phrase. In Russian, these two uses correspond to distinct lexical items: adverbial \textit{only} is translated by \textit{tol'ko} while DP-internal \textit{only} is translated by \textit{edinstvennyj}, as shown in (\ref{only-tolko}) and (\ref{only-edin}).

\begin{exe}
	\ex \label{only-tolko} \gll \textbf{Tol'ko} studenty pri\v{s}li.\\
	\textbf{only} students came\\
	\glt `Only the students came.'
	\ex \label{only-edin} \gll Marija vzjala \textbf{edinstvennuju} knigu.\\
	Maria took \textbf{only} book.\\
	\glt `Maria took the only book.'
\end{exe}

The same lexical distinction is made in Spanish and German \citep{mcnally08} and Chinese (Shizhe Huang, p.c.). The contrast in Russian and the other languages raises the possibility that adverbial \textit{only} and DP-internal \textit{only} are lexically distinct, though phonetically identical, in English. Another possibility is that \textit{edinstvennyj} corresponds not to \textit{only} but to another similar word like \textit{sole} or \textit{single}. \citet{cb2012a} show that \textit{only}, \textit{sole} and \textit{single} have similar but not identical syntactic and semantic properties.

It is also possible that \textit{edinstvennyj} in fact corresponds to the entire complex determiner \textit{the only}, i.e. \textit{edinstvennyj} itself is the locus of the determinate import, rather than receiving its determinacy through \citegen{partee86} \textsc{Iota} type-shift, which is how bare nominals typically become determinate in Russian.

% TODO: Finish this thought
% A similar issue arises in English as to whether \textit{the only} is a complex determiner or the semantic composition of \textit{the} and \textit{only}.


\section{Anti-uniqueness effects in Russian \label{sec:anti-uniqueness}}
\citeauthor{cb2015}'s first example of an anti-uniqueness effect was the predicative construction in (\ref{scott}). Definite descriptions can be predicative in Russian, as (\ref{pred-def}) shows.

\begin{exe}
	\ex \label{pred-def} \gll Dmitrij --- vysokij, simpatichnyj, i (samyj umnyj student vo vsyom universitete / *Boris).\\
	Dmitri {} tall cute and most smart student in all university {} Boris\\
	\glt `Dmitri is tall, cute and (the smartest student in the whole university / Boris).'
\end{exe}

Assuming that adjectives are of type $\langle e, t \rangle$ and proper names are of type $e$ in Russian, and that conjuncts must have the same semantic type, then the possibility of a definite description in (\ref{pred-def}) conjoining with an adjective, and the impossibility of a proper name doing so, indicates that definites can have type $\langle e, t \rangle$. The equivalent sentence without conjunction is grammatical (see (\ref{dmitri-boris})), so it is crucially the adjectival conjunction that renders the sentence with \textit{Boris} ungrammatical.\footnote{That is, the sentence is ungrammatical on an equative reading where \textit{Boris} has type $e$. It does have a grammatical reading where \textit{Boris} is taken to denote a set of properties associated with ``Boris-ness'', similarly to the usage below in English: \begin{exe} \ex He's such a Boris.\end{exe} Russian speakers may find the reading more accessible with a name like \textit{Putin} that is more easily given a property reading. Since this property-denoting interpretation of \textit{Boris} plausibly has type $\langle e, t \rangle$, its grammaticality supports my assertion.} Note that the superlative \textit{samyj umnyj student} `smartest student' was used to force the definite interpretation, since superlatives cannot be indefinite but regular bare nominals can be.

\begin{exe}
	\ex \label{dmitri-boris} \gll Dmitrij --- (samyj umnyj student vo vsyom universitete / Boris).\\
	Dmitri {} most smart student in all university {} Boris\\
	\glt `Dmitri is (the smartest student in the whole university / Boris).'
\end{exe}

It has therefore been established that definite descriptions can be predicates in Russian. Do Russian predicative definites exhibit anti-uniqueness effects? (\ref{tolstoy}) indicates that they do.

\begin{exe}
	\ex \label{tolstoy} \gll Tolstoj ne edinstvennyj avtor \textit{Vojny i mira}\\
	Tolstoy not only author \textit{War and Peace}\\
	\glt `Tolstoy is not the only author of \textit{War and Peace}.'
\end{exe}

(\ref{tolstoy}) has the same meaning as its English translation. It presupposes that Tolstoy is an author of \textit{War and Peace} and entails that one or more others are also authors. Therefore, \textit{edinstvennyj avtor Vojny i mira} `the only author of \textit{War and Peace}' fails to refer to an individual, just as in English, and an anti-uniqueness effect arises.

Anti-uniqueness effects are evident with argumental as well as predicative definites in English. For example, (\ref{only-goal-multiple}) is only compatible (on the most prominent reading) with a situation in which multiple goals were scored, including one scored by Anna. In such a situation, the description \textit{the only goal} does not refer to anything, since the existence of multiple goals precludes any one goal being denoted as the only goal.

The second sentence of (\ref{the-goal})-(\ref{only-goal-one}) is a diagnosis for anti-uniqueness effects: if a definite description does not refer, it should not be able to serve as the antecedent of a pronoun. Therefore, when \textit{It was an excellent strike} is not a felicitous continuation, an anti-uniqueness effect must be operative in the initial sentence. (\ref{the-goal}) shows that regular definite descriptions do not exhibit anti-uniqueness effects, as expected.

\begin{exe}
	\ex \label{the-goal} Anna didn't score the goal. It was an excellent strike.
	\ex \label{only-goal-multiple} Anna didn't score the only goal. *It was an excellent strike. (multiple goals)
	\ex \label{only-goal-one} Anna didn't score the only goal. It was an excellent strike. (one goal)
\end{exe}

Note however that the first sentence of (\ref{only-goal-multiple}) is ambiguous between negation taking narrow or wide scope. (\ref{only-goal-multiple}) is the narrow-scope reading, in which negation only applies to \textit{the only goal}, to mean approximately ``Anna scored not the only goal.'' A wide scope reading, in which negation applies to the entire verb phrase, is also available, meaning ``It wasn't Anna that scored the only goal.'' This reading, shown in (\ref{only-goal-one}), requires a situation in which one goal was scored (by someone other than Anna), which means that anti-uniqueness should not be in force since there is indeed an \textit{only goal}. As predicted, the second sentence in (\ref{only-goal-one}) is grammatical under the wide-scope reading. The felicity of (\ref{only-goal-one}) becomes more apparent when one appends ``Maria did'' to the first sentence, to force wide scope.

In Russian, the two ambiguous readings of the English sentence correspond to two distinct sentences that differ as to the placement of the negation particle \textit{ne}. In (\ref{only-goal-one-ru}), the negation particle precedes the VP and thus takes wide scope, yielding the reading under which a single goal was scored. As expected, the continuation \textit{On byl otlichnyj udar} `It was an excellent strike' is grammatical.

\begin{exe}
	\ex \label{only-goal-one-ru} \gll Anna \textbf{ne} zabila edinstvennyj gol. On byl otlichnyj udar.\\
	Anna \textbf{not} scored only goal it was excellent strike\\
	\glt `Anna didn't score the only goal. It was an excellent strike.' (one goal)
	\ex \label{only-goal-multiple-ru} \gll Anna zabila \textbf{ne} edinstvennyj gol. On byl otlichnyj udar.\\
	Anna scored \textbf{not} only goal it was excellent strike\\
	\glt `Anna didn't score the only goal. It was an excellent strike.' (multiple goals)
\end{exe}

In (\ref{only-goal-multiple-ru}), the negation particle precedes the object and thus takes narrow scope, yielding the reading under which multiple goals were scored. In contrast to English, though, this reading does not result in an anti-uniqueness effect---the second sentence of (\ref{only-goal-multiple-ru}) is perfectly grammatical in Russian, with the pronoun \textit{on} `it' referring to the goal that Anna scored.

% TODO: Tie these examples in.

\begin{exe}
	\ex \begin{xlist}
		\ex \gll Anna ne prochitala edinstvennuju lektsiju.\\
		Anna not read only lecture\\
		\glt `Anna didn't give the only talk.' (one talk)
		\ex Anna prochitala ne edinstvennuju lektsiju.\\
		Anna read not only lecture
		\glt `Anna didn't give the only talk.' (multiple talks)
	\end{xlist}
	\ex \begin{xlist}
		\ex \gll Anna ne posmotrela edinstvennuju lektsiju.\\
		Anna not watched only lecture\\
		\glt `Anna didn't see the only talk.'
		\ex Anna posmotrela ne edinstvennuju lektsiju.\\
		Anna watched not only lecture
		\glt `Anna saw one of the talks.'
	\end{xlist}
	\ex \begin{xlist}
		\ex \gll Denis ne prinjos edinstvennyj tort.\\
		Denis not brought only cake\\
		\glt `Denis didn't bring the only cake.'
		\ex Denis prinjos ne edinstvennyj tort.\\
		Denis brought not only cake
		\glt `Denis brought one of the cakes.'
	\end{xlist}
\end{exe}

Some speakers prefer \textit{eto} `this' to \textit{on} `it' in the second sentences of (\ref{only-goal-one-ru}) and (\ref{only-goal-multiple-ru}). \textit{Eto}, unlike \textit{on}, may refer to a semantic event \citep{davidson67}:

\begin{exe}
	\ex \label{davidson-ru} \gll Dzhon namazal maslo na khleb. On (eto / *ego) sdelal medlenno.\\
	John put butter on bread he this {} it did slowly\\
	\glt `John buttered the toast. He did it slowly.'
\end{exe}

In (\ref{davidson-ru}), \textit{eto} but not \textit{ego}, the accusative form of \textit{on},\footnote{Also the accusative form of \textit{ono}, the third-person neuter pronoun, so the ungrammaticality is not due to a gender mismatch.} may refer to the event variable introduced in the first sentence.

The diagnostic for anti-uniqueness effects relied on the inability of the definite description to be an antecedent, but it only works when no other antecedent is available (hence the use of the feminine \textit{Anna} in the first sentence). Since \textit{eto} may refer to a semantic event, it is possible that it is the previous sentence's event variable and not the definite description that \textit{eto} in the continuation refers to, in which case an anti-uniqueness effect may still be operative.

However, note that, as the gloss of (\ref{davidson-ru}) suggests, \textit{it} may refer to semantic events in English. Even so, (\ref{only-goal-multiple}) is still ungrammatical; the first sentence apparently does not introduce an event variable, or if it does it is not felicitous to describe it as ``an excellent strike.'' It is plausible that (\ref{only-goal-multiple-ru}) similarly does not have a reading where \textit{eto} refers to a semantic event.

Regardless, the possibility of \textit{on} in (\ref{only-goal-one-ru}) and (\ref{only-goal-multiple-ru}) in at least some Russian idiolects is sufficient to support the assertion that certain constructions do not show anti-uniqueness effects when they ought to.

The surprising lack of an anti-uniqueness effect in (\ref{only-goal-multiple-ru}) may be accounted for if the negation particle is actually inside the DP. In other words, the string \textit{ne edinstvennyj gol} is a DP headed by a null definite determiner, referring to the goal that Anna scored and entailing that other goals were scored as well. Since this DP has a referent, it is able to serve as the antecedent for the pronoun in the next sentence.

\section{Indeterminate \textit{edinstvennyj} \label{sec:indet-e}}
\citet{cb2015} predict that indeterminate readings should be more widely available in article-less languages than in English. Preliminary evidence from Russian shows that NPs beginning with \textit{edinstvennyj} resist indeterminate interpretation. In (\ref{sole-director}), for instance, the preferred translation of the English sentence with the indefinite phrase \textit{a sole director} uses the regular cardinal number \textit{odin} `one' rather than \textit{edinstvennyj}, which is marginal.

\begin{exe}
	\ex \label{sole-director} \gll U etoj kompanii --- (odin/??edinstvennyj) direktor\\
	At this company {} (one/only) director\\
	\glt `This company has a sole director.'
\end{exe}

% TODO: Add or remove this
%Note that \textit{odin} is morphologically related to \textit{edinstvennyj}, whose root is \textit{edin}. Similarly, \textit{one} in English may be morphologically related to \textit{only}.

In (\ref{not-a-sole}), another example of indefinite \textit{sole} in English, the translation with \textit{edinstvennyj} is outright ungrammatical. \textit{Odin} must be used.

\begin{exe}
	\ex \label{not-a-sole} \gll Ni (odin/*edinstvennyj) chelovek ne pri\v{s}yol.\\
	Not one/only person not came\\
	\glt `Not a sole person came.'\footnote{Russian \textit{ni} is a negative concordance particle rather than double negation.}
\end{exe}

(\ref{sole-director}) and (\ref{not-a-sole}) indicate that indeterminate readings for \textit{edinstvennyj} are dispreferred if not outright impossible. There is at least one exception to this generalization, however: an indeterminate reading for \textit{edinstvennyj} can be achieved in the compound expression \textit{odin-edinstvennyj}:

\begin{exe}
	\ex \label{odin-edinstvennyj} \gll Vrachi rekomendovali odin-edinstvennyj podkhod.\\
	doctors recommended one-only approach\\
	`The doctors recommended one single approach.'
\end{exe}

It is consistent with the other evidence to conclude that the source of the indefinite import of the NP in (\ref{odin-edinstvennyj}) is the numeral \textit{odin} rather than \textit{edinstvennyj}, and thus the generalization remains that \textit{edinstvennyj} does not independently allow an indeterminate reading.

\section{Summary and further research \label{sec:conclusion}}
The Russian data reviewed in this paper argues for three main positions: that there is a possibility of adverbial and DP-internal \textit{only} being different lexical items in English; that anti-uniqueness effects exist in Russian, but not in (all) argumental positions; and that \textit{edinstvennyj} requires a determinate interpretation.

Section \ref{sec:t-e} raised the possibility that \textit{edinstvennyj} corresponds not to \textit{only} but to \textit{sole} or \textit{single}. This possibility warrants further attention, as does the idea that adverbial \textit{only} and DP-internal \textit{only} may be distinct in English. In addition, a more exhaustive exploration of argumental anti-uniqueness effects and indeterminate readings of \textit{edinstvennyj} should be undertaken. Finally, both adverbial and DP-internal \textit{only} license NPIs in English:

\begin{exe}
	\ex *(Only) John ever comes.
	\ex That is the *(only) book that I ever read.
\end{exe}

The ability or inability of \textit{tol'ko} and \textit{edinstvennyj} to license NPIs in Russian deserves investigation.

I will collect more Russian data on each of these issues in order to further clarify the semantics of DP-internal \textit{only} and its relationship to definiteness.

\section*{Acknowledgements}
I am grateful to Alexandr Trubetskoy and Maxim Sonin for their help with the Russian data.


\bibliography{thesis}

\end{document}
