\documentclass{article}

\usepackage{verbatim}
\usepackage{natbib}
\bibliographystyle{linquiry2}
\usepackage{gb4e}

% Cite an author, e.g. Davidson's (1967)
\newcommand{\citegen}[1]{\citeauthor{#1}'s~(\citeyear{#1})}

\title{DP-internal \textit{only} and definiteness in English and Russian}
\author{Ian Fisher}
\date{October 29, 2018}

\begin{document}
\maketitle


\begin{abstract}
This thesis draft reviews several issues related to DP-internal \textit{only} and definiteness. It extends the work on definiteness done by \citet{cb2015, cb2012a, cb2012b} and tests some of their predictions against Russian. I show that anti-uniqueness effects occur in predicative constructions in Russian, but I fail to confirm the existence of argumental anti-uniqueness effects. I sketch an account of anti-uniqueness effects based on the scalar properties of DP-internal \textit{only}. I also take inventory of the syntactic and semantic properties of \textit{edinstvennyj} `(the) only' in comparison to a battery of similar words in English.
\end{abstract}


\section{Introduction}
The use of a definite description is conventionally assumed to entail that the description is unique to the entity that it denotes. There exist definite descriptions, however, which not only do not require uniqueness but are in fact incompatible with it. \citet{cb2015} observe that \textit{the only author of Waverley} in (\ref{scott}) is just such an anti-unique definite: the sentence entails that \textit{Waverley} was written by multiple authors, and consequently there can be no individual who fits the description \textit{the only author of Waverley}.

\begin{exe}
	\ex \label{scott} Scott is not the only author of \textit{Waverley}.
\end{exe}

Observe that it is crucially the word \textit{only} which instigates the anti-uniqueness effect. When \textit{only} is removed, as in (\ref{the-author}), then \textit{the author of Waverley} retains its referential status and the sentence becomes a simple negation of the identity of Scott and the author of \textit{Waverley}.

\begin{exe}
	\ex \label{the-author} Scott is not the author of \textit{Waverley}.
\end{exe}

The contrast between (\ref{scott}) and (\ref{the-author}) illustrates the surprising interaction of DP-internal \textit{only}, negation and definiteness. The main thrust of this paper is to explore this interaction and its ramifications in Russian, and more broadly to account for the semantic properties of DP-internal \textit{only} in both English and Russian.

In the following discussion several terms from \citet{cb2015} are used: ``definiteness'' in the special sense of a morphological feature which signals a weak uniqueness presupposition in English; ``determinacy,'' the property of denoting an individual (i.e., having type $e$); and ``anti-uniqueness effect,'' to refer to the phenomenon of a definite description lacking a uniqueness entailment, i.e. a definite which is indeterminate.

The paper is organized as follows: section \ref{sec:anti-uniqueness-en} reviews the distribution of anti-uniqueness effects in English. Section \ref{sec:anti-uniqueness-ru} discusses their distribution in Russian. Section \ref{sec:scalar} sketches an alternative account to \citeauthor{cb2015}'s of anti-uniqueness effects. Section \ref{sec:which-edin} presents an inventory of \textit{edinstvennyj}'s properties in comparison with an array of similar words in English, and tests \citeauthor{cb2015}'s prediction that indeterminate readings should be more widely available in Russian.

% TODO: Write a footnote with my reservations about the term "adverbial only"



\section{Anti-uniqueness effects in English \label{sec:anti-uniqueness-en}}
Anti-uniqueness effects occur with two kinds of definites in English: predicative definites, as in (\ref{scott}), and argumental definites, as in (\ref{only-goal}).

\begin{exe}
	\exr{scott} Scott is not the author of \textit{Waverley}.
	\ex \label{only-goal} Anna didn't score the only goal.
\end{exe}

A simple diagnostic is available to test the presence or absence of anti-uniqueness with an argumental definite. If the definite description indeed fails to denote an individual, then it should not be able to serve as the antecedent of a pronoun in a subsequent sentence. The contrast between (\ref{the-goal}) and (\ref{only-goal-multiple}) thus testifies to the presence of an anti-uniqueness effect in (\ref{only-goal}).

\begin{exe}
	\ex \label{the-goal} Anna didn't score [ the goal ]_1. It_1 was an excellent strike.
	\ex \label{only-goal-multiple} Anna didn't score [ the only goal ]_1. \#It_1 was an excellent strike.
\end{exe}

Note that (\ref{only-goal}) is actually ambiguous between a reading where one goal was scored by someone other than Anna and a reading where multiple goals were scored, including one by Anna. Under the one-goal reading, \textit{the only goal} does have a referent and therefore should be able to be a pronoun's antecedent, while it should not be under the multiple-goals reading. (\ref{only-goal-ambig-one}) and (\ref{only-goal-ambig-multiple}) tease apart the two readings with additional context and validate the two predictions.

\begin{exe}
	\ex \label{only-goal-ambig-one} One-goal reading: Anna didn't score [ the only goal ]$_1$, Maria did. It$_1$ was an excellent strike.
	\ex \label{only-goal-ambig-multiple} Multiple-goals reading: Anna didn't score [ the only goal ]$_1$, Maria also scored. \#It$_1$ was an excellent strike.
\end{exe}

The two readings correspond to two different scopes of negation. In the one-goal reading, negation takes wide scope over the entire VP \textit{score the only goal}. In the multiple-goals reading, negation takes narrow scope over the argument \textit{the only goal}, yielding an anti-uniqueness effect.

Only verbs of creation can induce argumental anti-uniqueness effects in English. When \textit{see} is substituted for \textit{score}, as in (\ref{see-only-goal}), then the referential use of \textit{the only goal} is forced; (\ref{see-only-goal}) can only mean that there was a single goal.\footnote{The multiple-goals reading is still possible with heavy emphasis on \textit{only}, as in: \begin{exe} \ex Anna didn't see the ONLY goal. There was more than one. \end{exe} I have no explanation for this possibility.}

\begin{exe}
	\ex \label{see-only-goal} Anna didn't see the only goal.
\end{exe}

In summary, anti-uniqueness effects are evident in predicative definites and argumental definites selected by verbs of creation, although only when the definite contains DP-internal \textit{only} in the scope of negation.

\begin{comment}
Anti-uniqueness effects bear a superficial resemblance to \citegen{strawson50} observation that definite descriptions need not be universally unique. (\ref{table-books}), for instance, surely does not require that only a single table exists in the world.

\begin{exe}
	\ex \label{table-books} The table is covered with books.
\end{exe}

Instead, (\ref{table-books}) requires that \textit{the table} be unique within a certain context, though it is compatible with a universe with multiple tables. (\ref{scott}), by contrast, is inherently incompatible with a situation with a unique author of \textit{Waverley}.

The difference, then, is between normal uses of definite descriptions, for which uniqueness is contextually limited, and anti-unique definite descriptions, for which uniqueness is outright forbidden.
\end{comment}

% TODO: Maybe talk about definites that fail to refer (e.g., "the King of France") as something distinct from anti-uniqueness effects
% TODO: Discuss Coppock and Beaver's account, maybe



\section{Anti-uniqueness effects in Russian \label{sec:anti-uniqueness-ru}}
Per \citet{chierchia98}, NPs in Russian may denote either predicates (type $\langle e, t \rangle$) or entities (type $e$), the same typological classification as English. However, since Russian lacks articles, bare count nominals are permitted in argument positions, provided they undergo covert type-shifting to acquire determinate or indeterminate import \`{a} la \citet{partee86}.

The upshot is that definiteness in the limited sense of \citeauthor{cb2015} is absent from Russian---indeterminate NPs are morphologically indistinguishable from determinate ones. Nevertheless, the use of certain adjectives like DP-internal \textit{only} and superlatives may be taken as a proxy for definiteness since their semantics more or less preclude indeterminate import (and indeed it will be shown in section \ref{sec:which-edin} that DP-internal \textit{only} in Russian cannot be indeterminate).

In what follows, I show that Russian evinces predicative but not argumental anti-uniqueness effects.

\subsection{Predicative anti-uniqueness in Russian}
Definite descriptions can be predicative in Russian, as (\ref{pred-def}) shows.

\begin{exe}
	\ex \label{pred-def} \gll Dmitrij --- vysokij, simpati\v{c}nyj, i (samyj umnyj student vo vs\"{e}m universitete / *Boris).\\
	Dmitri {} tall cute and most smart student in all university {} Boris\\
	\glt `Dmitri is tall, cute and (the smartest student in the whole university / Boris).'
\end{exe}

Assuming that (a) adjectives are of type $\langle e, t \rangle$ and proper names are of type $e$ in Russian, and (b) conjuncts must have the same semantic type, then the ability of a definite description in (\ref{pred-def}) to conjoin with an adjective, and the inability of a proper name to do so, indicates that definites can have type $\langle e, t \rangle$. The equivalent sentence without conjunction is grammatical (see (\ref{dmitri-boris})), so it is crucially the adjectival conjunction that renders the sentence with \textit{Boris} ungrammatical.\footnote{That is, the sentence is ungrammatical on an equative reading where \textit{Boris} has type $e$. It does have a grammatical reading where \textit{Boris} is taken to denote a set of properties associated with ``Boris-ness'', similarly to \textit{such-a} phrases in English: \begin{exe} \ex He's such a Boris.\end{exe} Russian speakers may find the reading more accessible with a name like \textit{Putin} that is more easily given a property reading. Since this property-denoting interpretation of \textit{Boris} plausibly has type $\langle e, t \rangle$, its grammaticality supports my assertion.} Note that the superlative \textit{samyj umnyj student} `smartest student' was used to force the determinate interpretation, since superlatives cannot be indeterminate but regular bare nominals can be.

\begin{exe}
	\ex \label{dmitri-boris} \gll Dmitrij --- (samyj umnyj student vo vs\"{e}m universitete / Boris).\\
	Dmitri {} most smart student in all university {} Boris\\
	\glt `Dmitri is (the smartest student in the whole university / Boris).'
\end{exe}

It has therefore been established that definite descriptions can be predicates in Russian. Do Russian predicative definites exhibit anti-uniqueness effects? (\ref{tolstoy}) indicates that they do.

\begin{exe}
	\ex \label{tolstoy} \gll Tolstoj ne edinstvennyj avtor \textit{Vojny i mira}\\
	Tolstoy not only author \textit{War and Peace}\\
	\glt `Tolstoy is not the only author of \textit{War and Peace}.'
\end{exe}

(\ref{tolstoy}) has the same meaning as its English translation. It presupposes that Tolstoy is an author of \textit{War and Peace} and entails that one or more others are also authors. Therefore, \textit{edinstvennyj avtor Vojny i mira} `the only author of \textit{War and Peace}' fails to refer to an individual, just as in English, and an anti-uniqueness effect arises.

\subsection{Argumental anti-uniqueness in Russian}
A prototypical example of an argumental anti-uniqueness effect was (\ref{only-goal}), repeated below.

\begin{exe}
	\ex Anna didn't score the only goal.
\end{exe}

In section \ref{sec:anti-uniqueness-en} it was shown that (\ref{only-goal}) is ambiguous between two readings, one evincing an anti-uniqueness effect and the other not.

In Russian, the two ambiguous readings of the English sentence correspond to two distinct sentences that differ in the placement of the negation particle \textit{ne}. In (\ref{only-goal-one-ru}), the negation particle precedes the VP and thus takes wide scope, yielding the reading under which a single goal was scored. As expected, the continuation \textit{On byl otli\v{c}nyj udar} `It was an excellent strike' is grammatical.

\begin{exe}
	\ex \label{only-goal-one-ru} \gll Anna \textbf{ne} zabila edinstvennyj gol. On byl otli\v{c}nyj udar.\\
	Anna \textbf{not} scored only goal it was excellent strike\\
	\glt `Anna didn't score the only goal. It was an excellent strike.' (one goal)
\end{exe}

In (\ref{only-goal-multiple-ru}), the negation particle precedes the object and thus takes narrow scope, yielding the reading under which multiple goals were scored. In contrast to English, though, this reading does not result in an anti-uniqueness effect---the second sentence of (\ref{only-goal-multiple-ru}) is perfectly grammatical in Russian, with the pronoun \textit{on} `it' referring to the goal that Anna scored.

\begin{exe}
	\ex \label{only-goal-multiple-ru} \gll Anna zabila \textbf{ne} edinstvennyj gol. On byl otli\v{c}nyj udar.\\
	Anna scored \textbf{not} only goal it was excellent strike\\
	\glt `Anna didn't score the only goal. It was an excellent strike.' (multiple goals)
\end{exe}

A similar pattern emerges for a number of other verbs of creation. \textit{Pro\v{c}itat'} `to read' and \textit{prinesti} `to bring' compose with \textit{ne} and \textit{edinstvennyj} in the exact same way as \textit{zabit'} `to score' to yield two sentences corresponding to the ambiguous readings of a single English sentence, as (\ref{prochitat}) and (\ref{prinesti}) show. As with (\ref{only-goal-multiple-ru}), (\ref{prochitat}) and (\ref{prinesti}) surprisingly lack the anti-uniqueness effects that are present in their English counterparts.

\begin{exe}
	\ex \label{prochitat} \begin{xlist}
		\ex \gll Anna ne pro\v{c}itala edinstvennuju lekciju.\\
		Anna not read only lecture\\
		\glt `Anna didn't give the only talk.' (one talk)
		\ex Anna pro\v{c}itala ne edinstvennuju lekciju.\\
		Anna read not only lecture
		\glt `Anna didn't give the only talk.' (multiple talks)
	\end{xlist}
	\ex \label{prinesti} \begin{xlist}
		\ex \gll Denis ne prin\"{e}s edinstvennyj tort.\\
		Denis not brought only cake\\
		\glt `Denis didn't bring the only cake.' (one cake)
		\ex Denis prin\"{e}s ne edinstvennyj tort.\\
		Denis brought not only cake
		\glt `Denis didn't bring the only cake.' (multiple cakes)
	\end{xlist}
\end{exe}

In another departure from the data in English, \textit{edinstvennyj} and negation can produce a multiplicity reading for argumental definites of verbs of non-creation, although again no anti-uniqueness effect occurs.

\begin{exe}
	\ex \label{see-only-goal-ru} \begin{xlist}
		\ex \gll Anna ne posmotrela edinstvennuju lekciju.\\
		Anna not watched only lecture\\
		\glt `Anna didn't see the only talk.'
		\ex Anna posmotrela ne edinstvennuju lekciju.\\
		Anna watched not only lecture
		\glt `Anna saw one of the talks.'
	\end{xlist}
\end{exe}

The Russian data may be summarized as showing that anti-uniqueness effects are not possible with any kind of argumental definite.

A minor issue must be dealt without before an account of the Russian data may be given. Some speakers prefer \textit{\`{e}to} `this' to \textit{on} `it' in the second sentences of (\ref{only-goal-one-ru}) and (\ref{only-goal-multiple-ru}). \textit{\`{E}to}, unlike \textit{on}, may refer to a semantic event \citep{davidson67}:

\begin{exe}
	\ex \label{davidson-ru} \gll D\v{z}on namazal maslo na xleb. On (\`{e}to / *ego) sdelal medlenno.\\
	John put butter on bread he this {} it did slowly\\
	\glt `John buttered the toast. He did it slowly.'
	\ex \label{davidson-ru-2} \gll Babu\v{s}ka umerla. (\`{E}to slu\v{c}ilos' / *on slu\v{c}ilsja / *ono slu\v{c}ilos') v\v{c}era.\\
	grandmother died this happened {} he happened {} it happened yesterday \\
	\glt `My grandmother died. It happened yesterday.'
\end{exe}

In (\ref{davidson-ru}) and (\ref{davidson-ru-2}), \textit{\`{e}to} but not \textit{on} or \textit{ono} `it' (and their shared accusative form \textit{ego}) may refer to the event variable introduced in the first sentence.

The diagnostic for anti-uniqueness effects relied on the inability of the definite description to be an antecedent, but it only works when no other antecedent is available (hence the use of the feminine \textit{Anna} in the first sentences of (\ref{only-goal-one-ru}) and (\ref{only-goal-multiple-ru})). Since \textit{\`{e}to} may refer to a semantic event, it is possible that it is the previous sentence's event variable and not the definite description that \textit{\`{e}to} in the continuation refers to, in which case an anti-uniqueness effect may still be operative.

However, note that, as the gloss of (\ref{davidson-ru}) suggests, \textit{it} may also refer to semantic events in English. Even so, (\ref{only-goal-multiple}) is still ungrammatical; the first sentence apparently does not introduce an event variable, or if it does it is not felicitous to describe it as ``an excellent strike.'' It is plausible that (\ref{only-goal-multiple-ru}), the direct Russian translation of (\ref{only-goal-multiple}), similarly does not have a reading where \textit{\`{e}to} refers to a semantic event.

Regardless, the possibility of \textit{on} in (\ref{only-goal-multiple-ru}) in at least some Russian idiolects is sufficient to support the assertion that certain constructions do not show anti-uniqueness effects when they ought to.

\subsection{The interpretation of \textit{ne edinstvennyj gol}}
How can it be that (\ref{only-goal-multiple-ru}) entails a multiplicity of goals and yet allows for a singular reference? The first step towards answering this question is to clarify the syntactic composition of \textit{ne edinstvennyj gol} `not (the) only goal.' The constituent negation particle \textit{ne} must either be outside the DP, as in (\ref{external-ne}), or internal to it, as in (\ref{internal-ne}).

\begin{exe}
	\ex \label{external-ne} Anna zabila ne [_{DP} edinstvennyj gol].
	\ex \label{internal-ne} Anna zabila [_{DP} ne edinstvennyj gol].
\end{exe}

% TODO: What does the literature on constituent negation say?
% TODO: In fact I think the facts might be easier to explain if ne is external to the DP.

The presence of \textit{edinstvennyj} is crucial to the availability of constituent negation: (\ref{ne-no-edin}), the same sentence without \textit{edinstvennyj}, is ungrammatical.

\begin{exe}
	\ex[*] { \label{ne-no-edin}
		\gll Anna zabila ne gol.\\
		Anna scored not goal\\
	}
\end{exe}

The meaning of (\ref{only-goal-multiple-ru}) indicates that \textit{edinstvennyj} is being negated, since \textit{edinstvennyj} entails singularity while the matrix sentence entails multiplicity. It is not obvious how this semantic relationship and the ungrammaticality of (\ref{ne-no-edin}) could be explained if \textit{ne} where external to the DP. It would appear that the syntactic and semantic properties of \textit{ne edinstvennyj gol} are most easily explained if \textit{ne} and \textit{edinstvennyj} compose directly, implying that \textit{ne} must be internal to the DP since otherwise the phonologically null head of DP would intervene between the two words.

In that case, it must be the DP \textit{ne edinstvennyj gol} that refers to the goal that Anna scored, out of the multiple goals that were scored.

This account raises a serious issue. \textit{Edinstvennyj gol} presumably denotes a singleton set. The negation particle \textit{ne} normally functions as the set complement operation, which means that \textit{ne edinstvennyj gol} ought to denote the complement of a singleton set, which should have more than one element. But a set with more than one element should not be able to be interpreted as determinate.

The difficulty is not so much a matter of an inexpressive theory as a true peculiarity in the facts of the Russian language in this instance. It is quite unusual that \textit{ne edinstvennyj gol} `the not-only goal' can simultaneously entail a multiplicity of goals while denoting a single one.

I suggest two approaches: one which makes use of a selection operation on sets to pick out a singular referent, and one which postulates a multiplicity-singularity division between presupposition and assertion.

What would be required in the semantics to model this peculiarity is some operation to pick out one of the multiplicity of goals. This selection operation, which would extract a singular referent from a multiplicity, must be constrained by the form of the rest of the sentence: \textit{on} (or \textit{\`{e}to}  as the case may be) in (\ref{only-goal-multiple-ru}) may not refer to just any goal that was scored, but specifically the goal that Anna scored.

The idea of a phrase yielding an antecedent other than the set it denotes is not unprecedented in the literature: complement anaphora are just such a phenomenon. A complement anaphor is a pronoun that has as its antecedent the complement of some set previously denoted \citep{nouwen03, schwarz09}. In (\ref{kennedy}), the antecedent of \textit{they} in the second sentence is the complement of \textit{few congressmen}, i.e. few congressmen admire Kennedy so the majority of them think he is incompetent.

\begin{exe}
	\ex \label{kennedy} Few congressmen admire Kennedy. They think he's incompetent.
\end{exe}

Complement anaphora illustrate that the antecedent of a pronoun need not always be explicitly present in the semantics, as long as it can be derived from some entity that is present. Complement anaphora involve the set complement relationship; the data from Russian suggests that some individual-selection operation on sets may also be available.

The second possibility is that there is some division of semantic meaning between presupposition and assertion which retains the multiplicity entailment while allowing \textit{ne edinstvennyj gol} to refer to a single goal. That is, the multiplicity entailment would be confined to the presupposition, and the assertive content would therefore contain only a reference to a singular goal. Unfortunately, the presuppositive component of (\ref{only-goal-multiple-ru}) cannot be isolated by negation, because the negated version of the sentence is independently ungrammatical:\footnote{Note that (\ref{only-goal-multiple-ru}) is an affirmative sentence, as the negation particle \textit{ne} does not have sentential scope.}

\begin{exe}
	\ex[*] { \label{double-neg}
		\gll Anna ne zabila ne edinstvennyj gol.\\
		Anna not scored not only goal\\
		\glt Intended: `Anna didn't score any of the goals.'
	}
\end{exe}

The presumed meaning of (\ref{double-neg}) would be, as the English gloss suggests, that multiple goals were scored, and Anna didn't score any of them. In that case, the presupposition and assertion of (\ref{only-goal-multiple-ru}) would be as follows:

\begin{exe}
	\ex Anna zabila ne edinstvennyj gol. \begin{xlist}
		\ex Presupposition: Multiple goals were scored.
		\ex Assertion: Anna scored a goal.
	\end{xlist}
\end{exe}

It would seem then that the reference of \textit{ne edinstvennyj gol} is relative to the assertion, while the multiplicity entailment is actually a presupposition.

The two approaches I have proposed are not incompatible. It could be that the set selection operation is sensitive to the assertion-presupposition distinction. In any case, I hope to have made the ability of \textit{ne edinstvennyj gol} to imply multiple goals but denote a single one a little less puzzling.



\section{Scalar account of anti-uniqueness effects \label{sec:scalar}}
The account of anti-uniqueness effects in \citet{cb2015, cb2012a} constitutes an expansive reimagining of definiteness in English. Three main ideas underlie their theory of the semantics of definiteness:

\begin{itemize}
	\item The definite and indefinite articles are identity functions, and the definite article carries a weak uniqueness presupposition.\footnote{Weak uniqueness is defined as uniqueness or non-existence, i.e. if uniqueness means $|P| = 1$ then weak uniqueness means $|P| \le 1$.}
	\item The determinacy of an expression is established by covert type-shifting.
	\item Indeterminate interpretations of definites and determinate interpretations of indefinites are generally blocked by the principles of Maximize Presupposition and Type Simplicity.
\end{itemize}

However, anti-uniqueness effects are only ever observed with \textit{the only} phrases in the scope of negation in English, and their existence in Russian seems to be further limited to predicative contexts. It would make for a more perspicuous theory if the existence of anti-uniqueness effects could be attributed to some special properties of \textit{the only}, leaving the theory of definiteness unchanged.

One candidate for such a theory relates anti-uniqueness effects to the scalar properties of DP-internal \textit{only}. A pragmatic scale is a ranking of terms $x_1$ through $x_n$ such that if $P$ is true of $x_i$, then $P$ is true of all terms $x_j$ where $j \le i$ \citep{fauconnier75}.

The scalar properties of DP-internal \textit{only} are illustrated by (\ref{scott-pos}) and (\ref{scott-neg}).

\begin{exe}
	\ex \label{scott-pos} Scott is the only author of \textit{Waverley}.
	\ex \label{scott-neg} Scott is not the only author of \textit{Waverley}.
\end{exe}

(\ref{scott-pos}) entails that the set denoted by \textit{author of Waverley} has a cardinality of exactly one. (\ref{scott-neg}) entails that the set has a cardinality greater than one. These sentences may be taken to induce a scale along the dimension of $n$ authors:

\begin{itemize}
	\item 0 authors
	\item 1 author
	\item 2 authors
	\item etc.
\end{itemize}

In terms of this $n$ authors scale, (\ref{scott-pos}) restricts the scale to a single point and (\ref{scott-neg}) flips it to contain all greater points.

Note that the logical negation of $|\textsc{only author of Waverley}| = 1$ is
$|\textsc{only author of Waverley}| \ne 1$, or equivalently $|\textsc{only author of Waverley}| = 0 \lor |\textsc{only author of Waverley}| > 1$. But (\ref{scott-neg}) is not compatible with $|\textsc{only author of Waverley}| = 0$, so the zero case must be excluded somehow.

Here's how. Both (\ref{scott-pos}) and (\ref{scott-neg}) entail that Scott is an author of \textit{Waverley}, so that proposition must be a semantic presupposition. From this presupposition, it follows logically that the cardinality of the \textit{author of Waverley} set cannot be zero, since it must contain at least Scott. Thus, the proper scalar entailment is derived correctly after all.

What a scalar account of anti-uniqueness effects, and indeed any account that preserves the conventional theory of definiteness, would require is that \textit{the only} is a semantic atom rather than the composition of \textit{the} and \textit{only}. If \textit{the only P} is derived compositionally as \textit{the} and \textit{only P}, then the entire phrase must be a definite description, since it is headed by a definite article, and its lack of a uniqueness presupposition would truly be evidence that the definite article cannot categorically have a uniqueness presupposition. On the other hand, \textit{the only} is a semantic atom, then \textit{the only P} need not be considered a definite description under all circumstances.

There is independent evidence that DP-internal \textit{only} may be different from adverbial \textit{only}, in which case the non-compositional analysis of \textit{the only} would be more plausible. We have seen that DP-internal \textit{only} is translated as \textit{edinstvennyj} in Russian. Adverbial \textit{only} corresponds to a different lexical item, \textit{tol'ko}:

\begin{exe}
	\ex \label{only-tolko} \gll \textbf{Tol'ko} studenty pri\v{s}li.\\
	\textbf{only} students came\\
	\glt `Only the students came.'
	\ex \label{only-edin} \gll Marija vzjala \textbf{edinstvennuju} knigu.\\
	Maria took \textbf{only} book.\\
	\glt `Maria took the only book.'
\end{exe}

The same lexical distinction is made in Spanish and German \citep{mcnally08} and Chinese (Shizhe Huang, p.c.). The contrast in Russian and the other languages raises the possibility that adverbial \textit{only} and DP-internal \textit{only}, though phonetically identical in English, are semantically distinct.

If the two \textit{only}'s had the same semantics, then a compositional analysis of \textit{the only} would be preferred, all else being equal, to capture the semantic similarity in terms of a shared lexical entry. If their semantics were distinct, then there would be no particular reason to favor the compositional analysis over the atomic one.

To summarize the line of reasoning thus far: a scalar account of anti-uniqueness effects is attractive, but requires that \textit{the only} be semantically atomic. Evidence from Russian and other languages indicates that DP-internal \textit{only} may be different from adverbial \textit{only}, which tends to support the atomic analysis of \textit{the only}.

% TODO: Talk about only's other scalar properties



\section{The semantics of \textit{edinstvennyj} \label{sec:which-edin}}
The preceding discussion has more or less assumed that \textit{edinstvennyj} in Russian corresponds to DP-internal \textit{only} in English. In this section, I wish to flesh out that assumption by comparing \textit{edinstvennyj} with a number of exclusive nominal modifiers in English. There are several with a similar meaning, in addition to \textit{only}: \textit{sole}, \textit{single}, and \textit{one}. (\ref{osso}), for instance, has the same meaning regardless of the choice of adjective.

\begin{exe}
	\ex \label{osso} The (only/sole/single/one) person to come was Ahmed.
\end{exe}

However, the various words evince distinct semantic and syntactic properties in other circumstances. \citet{cb2012b} catalog the inventory of properties thoroughly. Their conclusions are summarized in the table below.\\

\begin{tabular}{ l | l l l l l }
	& indefinite article & superlative & plural & NPIs & DP negation \\
	\hline
	\textit{only} & no & no & yes & yes & no \\
	\textit{sole} & yes & yes & yes & yes & yes \\
	\textit{single} & yes & yes & no & marginal & yes \\
	\textit{one} & no & marginal & no & yes & no \\
\end{tabular}

% Awkward way to force more space below table.
\ \\

The judgments in the table are shown by the sentences (\ref{osso-indef})-(\ref{osso-dp-neg}). In (\ref{osso-indef}), the exclusive adjectives are placed in a DP headed by an indefinite article. In (\ref{osso-super}), they combine with a superlative NP. In (\ref{osso-pl}), they combine with a plural NP. In (\ref{osso-npi}), they license or fail to license negative polarity items. In (\ref{osso-dp-neg}), they undergo DP negation.

\begin{exe}
	\ex \label{osso-indef} This company has a(n) (*only/sole/single/*one) director.
	\ex \label{osso-super} The oil spill was the (*only/sole/single/?one) worst environmental disaster in the state's history.
	\ex \label{osso-pl} They are the (only/sole/*single/*one) people we can trust.
	\ex \label{osso-npi} The (only/sole/??single/one) pick-up truck he ever owned was a Chevrolet.
	\ex \label{osso-dp-neg} Not a(n) (*only/sole/single/*one) person came.
\end{exe}

The remainder of this section will test the properties of \textit{edinstvennyj} against this matrix.

\subsection{Indeterminate \textit{edinstvennyj}}
% TODO: Move this intro to the end and replace with more appropriate intro.
Russian lacks an indefinite article, but instances where English uses an indefinite article with the exclusive adjectives listed above generally cannot be translated with \textit{edinstvennyj} in Russian. In (\ref{sole-director}), for instance, the preferred translation of the English sentence with the indefinite phrase \textit{a sole director} uses the regular cardinal number \textit{odin} `one' rather than \textit{edinstvennyj}, which is marginal.

\begin{exe}
	\ex \label{sole-director} \gll U \`{e}toj kompanii --- (odin/??edinstvennyj) direktor.\\
	At this company {} (one/only) director\\
	\glt `This company has a sole director.'
\end{exe}

% TODO: Add or remove this
%Note that \textit{odin} is morphologically related to \textit{edinstvennyj}, whose root is \textit{edin}. Similarly, \textit{one} in English may be morphologically related to \textit{only}.

In (\ref{not-a-sole}), another example of indefinite \textit{sole} in English, the translation with \textit{edinstvennyj} is outright ungrammatical. \textit{Odin} must be used.

\begin{exe}
	\ex \label{not-a-sole} \gll Ni (odin/*edinstvennyj) \v{c}elovek ne pri\v{s}\"{e}l.\\
	Not one/only person not came\\
	\glt `Not a sole person came.'\footnote{Russian \textit{ni} is a negative concordance particle in this case, rather than double negation.}
\end{exe}

(\ref{sole-director}) and (\ref{not-a-sole}) indicate that indeterminate readings for \textit{edinstvennyj} are dispreferred if not outright impossible. There are at least two exceptions to this generalization, however. The first is that an indeterminate reading for \textit{edinstvennyj} can be achieved in the compound expression \textit{odin-edinstvennyj}:

\begin{exe}
	\ex \label{odin-edinstvennyj} \gll Vra\v{c}i rekomendovali odin-edinstvennyj podxod.\\
	doctors recommended one-only approach\\
	`The doctors recommended one single approach.'
\end{exe}

The second is that \textit{edinstvennyj} can combine with \textit{reb\"{e}nok} `child' to mean `an only child' (i.e., a child with no siblings):

\begin{exe}
	\ex \label{only-child-ru} Marija --- edinstvennyj reb\"{e}nok.\\
	Maria {} only child
	\glt `Maria is an only child.'\footnote{As is generally the case with bare nominals in Russian, \textit{edinstvennyj reb\"{e}nok} also has a determinate reading, meaning `Maria is the only child.'}
\end{exe}

It is consistent with the other evidence to conclude that the source of the indeterminate import of the NP in (\ref{odin-edinstvennyj}) is the numeral \textit{odin} rather than \textit{edinstvennyj}, so (\ref{odin-edinstvennyj}) is not a true counterexample.

(\ref{only-child-ru}) is more problematic, as there is no other candidate for licensing the indeterminate reading. However, it is also true that DP-internal \textit{only} cannot generally be indeterminate in English:

% TODO: Note though that 'edinstvennyj rebyonok' is only indeterminate insofar as other predicative DPs are indeterminate.

\begin{exe}
	\ex Examples (32)-(34) from \citet{cb2012a} \begin{xlist}
		\ex If the business is owned by a(n) sole/*only owner, only the owner is eligible to be the managing officer.
		\ex This company has a(n) sole/*only director.
		\ex There was a(n) sole/*only piece of cake left.
	\end{xlist}
\end{exe}

In English, \textit{only} can be indeterminate only when it combines with the noun \textit{child} (and derived nouns like \textit{grandchild}). (\ref{sole-director}) shows that it is the same case in Russian. Since DP-internal \textit{only} does not productively allow indeterminate readings, \textit{an only child} may be considered idiomatic in both languages and not indicative of the general properties of DP-internal \textit{only}.\footnote{It is nonetheless curious that the same idiom should surface in both languages. I have no comment on this coincidence at the moment.}

Thus, despite the two objections, the generalization remains that \textit{edinstvennyj} does not independently allow an indeterminate reading, contrary to the predictions that \citet{cb2015} make in their conclusion about languages lacking articles.

\subsection{Licensing of negative polarity items}
Both adverbial and DP-internal \textit{only} license negative polarity items in English:

\begin{exe}
	\ex *(Only) Khalid \textbf{ever} goes to the movies.
	\ex The *(only) poem I \textbf{ever} read in high school was ``The Raven.''
\end{exe}

DP-internal \textit{only} cannot license NPIs outside of its DP:

\begin{exe}
	\ex[*] {The only team that I had heard of \textbf{ever} won the World Cup.}
\end{exe}

(\ref{libo-vs-nibud}) shows \textit{edinstvennyj} licensing two kinds of NPIs, \textit{kto-libo} \citep{pereltsvaig06} and \textit{kto-nibud'} \citep{russneg}.

\begin{exe}
	\ex \label{libo-vs-nibud} \gll Ivan vzjal edinstvennuju knigu, kotoruju (kto-libo / ?kto-nibud') xotel.\\
	Ivan took only book which anybody {} anybody wanted\\
	\glt `Ivan took the only book that anybody wanted.'
\end{exe}

\textit{Edinstvennyj} cannot license \textit{kto-nibud'} outside of its DP:

\begin{exe}
	\ex \label{nibud-out-of-dp} \gll Edinstvennyj u\v{c}itel' vybral (kogo-to / *kogo-nibud').\\
	only teacher picked someone {} anyone\\
	\glt `The only teacher picked someone.'
\end{exe}

In (\ref{nibud-out-of-dp}), \textit{kto-to}	is a positive polarity item that is subject to Principle C of the Binding Theory \citep{russneg}.\footnote{The morphemes \textit{to}, \textit{nibud'} and \textit{libo} are affixes or clitics which may attach to a number of pronouns, including \textit{\v{c}to} `what' (\textit{\v{c}to-to}, \textit{\v{c}to-nibud'}, \textit{\v{c}to-libo}) and \textit{kto} `who' (\textit{kto-to}, \textit{kto-nibud'}, \textit{kto-libo}). Only the underlying pronoun takes case endings, hence forms like \textit{kogo-to}, the genitive and accusative declension of \textit{kto-to}.}

The NPI status of \textit{nibud'}-items is a little unclear, as they are licensed, at least in some circumstances, in non-monotonic contexts like declarative sentences:\footnote{Russian speakers may find (\ref{nibud-decl}) more acceptable with additional context, such as \textit{Boris v laboratorii} `Boris is in the laboratory.'}

\begin{exe}
	\ex \label{nibud-decl} \gll Boris \v{c}to-nibud' delaet.\\
	Boris anything does\\
	\glt `Boris is doing something.' % TODO: May not be the most accurate gloss
\end{exe}

Nevertheless, the ability of \textit{edinstvennyj} to license \textit{libo}-items, clear examples of Russian NPIs, is sufficient to confirm its status as an NPI licenser.

As a brief aside, it is worth considering whether DP-internal \textit{only} licenses a downward entailment as NPI licensers are conventionally assumed to do. A downward entailment holds when a predicate that is true for a superset is also necessarily true for any subset. Sentential negation is the canonical example of a downward-entailing environment, as the entailment from the superset \textit{reptiles} in (\ref{reptiles}) to the subset \textit{snakes} in (\ref{snakes}) shows.

\begin{exe}
	\ex \label{reptiles} No reptiles give birth to live young.
	\ex \label{snakes} No snakes give birth to live young.
\end{exe}

DP-internal \textit{only} only licenses NPIs within its DP, so that is the relevant position to look for a downward entailment. The evidence initially suggests that DP-internal \textit{only} does not create an downward entailment. (\ref{everest-29}) does not entail (\ref{everest-30}), although the set of mountains taller than 30,000 feet is clearly a subset of the set of mountains taller than 29,000 feet. (\ref{russian-book}) does not necessarily entail (\ref{gogol-book})---suppose the only book in the library written by an Russian was actually by Turgenev---even though Gogol is in the set of all Russians.

\begin{exe}
	\ex \label{everest-29} The only mountain greater than 29,000 feet tall is Everest.
	\ex \label{everest-30} The only mountain greater than 30,000 feet tall is Everest.
	\ex \label{russian-book} The only book in the library written by a Russian is already checked out.
	\ex \label{gogol-book} The only book in the library written by Gogol is already checked out.
\end{exe}

However, I do not think that (\ref{everest-29})-(\ref{gogol-book}) constitute a knock-down argument against DP-internal \textit{only}'s downward monotonicity. The failure of the superset sentence to entail the subset sentence stems solely from the possibility of the definite description failing to refer. In other words, (\ref{everest-29}) does in fact entail (\ref{everest-30}) and (\ref{russian-book}) does (\ref{gogol-book}), so long as the definite descriptions \textit{the only mountain greater than 30,000 feet} and \textit{the only book in the library that was written by Gogol} have referents. It is never possible for (\ref{everest-29}) to be true and (\ref{everest-30}) to be false. If (\ref{everest-29}) is true, then (\ref{everest-30}) is either true or undefined.

Nevertheless, other downward-entailing operators do not carry an existence requirement: (\ref{reptiles}) entails (\ref{iguanas}) regardless of whether or not such a creature as a four-toed frilled iguana actually exists.

\begin{exe}
	\ex \label{iguanas} No four-toed frilled iguanas give birth to live young.
\end{exe}

The source of this discrepancy is of course the definite article (or the \textsc{Iota} shift, per \citeauthor{cb2015}), which causes the entire sentence to have an undefined truth value if its NP complement is not a singleton set.

In short, DP-internal \textit{only} creates a downward entailment (though still subject to the uniqueness requirement of the definite article), and thus the NPI-licensing properties of \textit{only} and \textit{edinstvennyj} fit neatly into the classical paradigm.


\subsection{Other properties of \textit{edinstvennyj}}
The remaining properties of \textit{edinstvennyj} to be pinned down are its ability to combine with superlative NPs and plural NPs, and to undergo DP negation. (\ref{not-a-sole}) already showed that DP negation is impossible for \textit{edinstvennyj}. (\ref{plural-edin}) shows that \textit{edinstvennyj} may modify a plural NP.

\begin{exe}
	\ex \label{plural-edin} \gll Oni --- edinstvennye ljudi, kotorym ja doverjaju.\\
	they {} only people which I trust\\
	\glt `They are the only people that I trust.'
\end{exe}

The ungrammaticality of (\ref{super-edin}) demonstrates that \textit{edinstvennyj} cannot modify a superlative NP.

\begin{exe}
	\ex[*] { \label{super-edin} \gll
		\`{E}to edinstvennyj samyj vysokiy neboskr\"{e}b v \v{C}ikago.\\
		this only most tall skyscraper in Chicago\\
		\glt Intended: `It is the single tallest skyscraper in Chicago.'
	}
\end{exe}

\subsection{Summary}
The relevant semantic and syntactic properties of \textit{edinstvennyj} and its potential counterparts are thus:\\

\begin{tabular}{ l | l l l l l }
	& indefinite article & superlative & plural & NPIs & DP negation \\
	\hline
	\textit{edinstvennyj} & no & no & yes & yes & no \\
	\textit{only} & no & no & yes & yes & no \\
	\textit{sole} & yes & yes & yes & yes & yes \\
	\textit{single} & yes & yes & no & marginal & yes \\
	\textit{one} & no & marginal & no & yes & no \\
\end{tabular}

% Awkward way to force more space below table.
\ \\

The properties of \textit{edinstvennyj} are most similar to those of DP-internal \textit{only}, and indeed may be identical, at least within the matrix under consideration. The near-identity of \textit{only} and \textit{edinstvennyj} in a range of circumstances adds support to my comparison between the two words with regards to anti-uniqueness effects.



\section{Summary and further research \label{sec:conclusion}}
The Russian data reviewed in this paper argues for two main positions: that anti-uniqueness effects exist in Russian, but not in argumental positions, and that \textit{edinstvennyj} corresponds most closely to DP-internal \textit{only} out of the array of exclusive adjectives in English.

The Russian data raised the possibility of an expression implying a multiplicity of entities but referring to a specific one among them. I analyzed this unusual phenomenon preliminarily with a set-selection operation sensitive to the division of meaning between presupposition and assertion.

I have shown that a scalar account of \textit{the only} constructions goes a little ways towards clarifying the issue of anti-uniqueness effects. The scalar approach remains promising but will require much further elucidation. It seems to me that an explanation that situates DP-internal \textit{only} as the locus of the anti-uniqueness effects is more desirable than one that requires a wholesale reevaluation of definiteness in English, as \citeauthor{cb2015}'s theory does.



\section*{Acknowledgements}
I am grateful to Alexandr Trubetskoy and Maxim Sonin for their help with the Russian data.



\bibliography{thesis}

\end{document}
