\documentclass{article}

\usepackage{verbatim}
\usepackage{natbib}
\bibliographystyle{linquiry2}
\usepackage{gb4e}

\title{DP-internal \textit{only} and definiteness in English and Russian}
\author{Ian Fisher}
\date{October 7, 2018}

\begin{document}
\maketitle


\begin{abstract}
This thesis draft reviews several issues related to DP-internal \textit{only} and definiteness. It extends the work done by \citeauthor{cb2015} (\citeyear{cb2012a}, \citeyear{cb2012b}, \citeyear{cb2015}) on definiteness and tests some of their predictions against Russian. I show that anti-uniqueness effects are apparent in predicative constructions in Russian, but I fail to confirm the existence of argumental anti-uniqueness effects. I also argue that \textit{edinstvennyj} `(the) only' resists indeterminate interpretation.
\end{abstract}


\section{Background}
\citet{cb2015} observes that in (\ref{scott}) the definite description \textit{the only author of Waverley} fails to refer to any individual.

\begin{exe}
	\ex \label{scott} Scott is not the only author of \textit{Waverley}.
\end{exe}

This fact challenges the common belief that definite descriptions presuppose the uniqueness of the entity that they denote.

\citeauthor{cb2015} distinguish between two related concepts: definiteness, a morphological feature which they argue signals a weak uniqueness presupposition in English; and determinacy, the property of denoting an individual. One of the key contributions of \citet{cb2015} is the proposal that definiteness and determinacy are distinct phenomena in English, i.e. that there are morphological definite descriptions that are not determinate.


\citet{chierchia98} classifies Russian as a language in which NPs may denote either predicates (type $\langle e, t \rangle$) or entities (type $e$), the same typological classification as English. However, since Russian lacks articles, bare count nominals are permitted in argument positions, provided they undergo covert type-shifting \`{a} la \citet{partee86}.

I will refer to the use of \textit{only} inside a determiner phrase as ``DP-internal \textit{only}'' and the use of \textit{only} otherwise as ``adverbial \textit{only}.''

\section{\textit{Tol'ko} and \textit{edinstvennyj}}
In English, the phonetic and orthographic identity of adverbial \textit{only} and DP-internal \textit{only} make it tempting to assume they are the same lexical item with the same semantics.

But in Russian, adverbial \textit{only} and DP-internal \textit{only} are entirely separate words. Adverbial \textit{only} is translated by \textit{tol'ko} while DP-internal \textit{only} is translated by \textit{edinstvennyj}:

\begin{exe}
	\ex \gll \textbf{Tol'ko} studenty pri\v{s}li.\\
	\textbf{only} students came\\
	\glt `Only the students came.'
	\ex \gll Marija vzjala \textbf{edinstvennuju} knigu.\\
	Maria took \textbf{only} book.\\
	\glt `Maria took the only book.'
\end{exe}

The evidence so far is unclear as to whether \textit{edinstvennyj} corresponds to \textit{only} alone or to \textit{the only}.


\section{Anti-uniqueness effects in Russian}
% Diagnostics for predicative/argumental definites in Russian

The second object position of \textit{consider} is a diagnostic for type $\langle e, t \rangle$ in English.

\begin{exe}
	\ex John considers Dr. Jekyll a madman.
	\ex[*] {John considers Dr. Jekyll Mr. Hyde.}
\end{exe}

In Russian the verb \textit{schitat'} `consider'

\begin{exe}
	\ex \gll Ivan schitaet Napoleona velichajshim frantsuzskim soldatom.\\
	Ivan considers Napoleon greatest French soldier\\
	`Ivan considers Napoleon the greatest French soldier.'
	\ex \gll Ivan schitaet Napoleona Borisom.\\
	Ivan considers Napoleon Boris\\
	`Ivan considers Napoleon to be the same person as Boris.'
\end{exe}

\begin{exe}
	\ex \gll Dmitrij vysokij, simpatichnyj, i samyj umnyj student vo vsyom universitete.\\
	Dmitri tall cute and most smart student in all university\\
	`Dmitri is tall, cute and the smartest student in the whole university.'
	\ex[*] {
		\gll Dmitrij vsokiy, simpatichnyj, i Boris.\\
		Dmitri tall cute and Boris\\
		`Dmitri is tall, cute and Boris.'
	}
\end{exe}

% Anti-uniqueness effects in the predicate position

\begin{exe}
	\ex \label{tolstoy} \gll Tolstoj ne edinstvennyj avtor \textit{Vojny i mira}\\
	Tolstoy not only author \textit{War and Peace}\\
	\glt `Tolstoy is not the only author of \textit{War and Peace}.'
\end{exe}

(\ref{tolstoy}) has the same meaning as its English translation. It presupposes that Tolstoy is an author of \textit{War and Peace} and entails that one or more others are also authors. Therefore, \textit{edinstvennyj avtor Vojny i mira} `the only author of \textit{War and Peace}' fails to refer to an individual, just as it does in English, and an anti-uniqueness effect arises.

Anti-uniqueness effects are evident with argumental as well as predicative definites in English. For example, (\ref{only-goal-multiple}) is only compatible with a situation in which multiple goals were scored, including one scored by Anna. In such a situation, the description \textit{the only goal} does not refer to anything, since the existence of multiple goals precludes any one goal being denoted as the only goal.

The second sentence of (\ref{only-goal-multiple}) is a diagnosis for anti-uniqueness effects: if a definite description does not refer, it should not be able to serve as the antecedent of a pronoun. The ungrammaticality of the second sentence of (\ref{only-goal-multiple}) shows that anti-uniqueness effects do arise in the first sentence.

\begin{exe}
	\ex \label{only-goal-multiple} Anna didn't score the only goal. *It was an excellent strike. (multiple goals)
	\ex \label{only-goal-one} Anna didn't score the only goal. It was an excellent strike. (one goal)
\end{exe}

Note however that the first sentence of (\ref{only-goal-multiple}) is in fact ambiguous between negation taking narrow or wide scope. (\ref{only-goal-multiple}) is the narrow-scope reading, in which negation only applies to \textit{the only goal} to mean approximately ``Anna scored not the only goal.'' A wide scope reading, in which negation applies to the entire verb phrase, is also available, meaning ``It wasn't Anna that scored the only goal.'' This reading requires a situation in which one goal was scored (by someone other than Anna), which means that anti-uniqueness should not be in force. And indeed, the second sentence in (\ref{only-goal-one}) is grammatical under these circumstances. The felicity of (\ref{only-goal-one}) becomes more apparent when one appends ``Maria did'' to the first sentence, to force wide scope.

In Russian, the two ambiguous readings of the English sentence correspond to two distinct sentences that differ as to the placement of the negation particle \textit{ne}. In (\ref{only-goal-one-ru}), the negation particle precedes the VP and thus takes wide scope, yielding the reading under which a single goal was scored. As expected, the continuation \textit{On byl otlichnyj udar} `It was an excellent strike' is grammatical.

\begin{exe}
	\ex \label{only-goal-one-ru} \gll Anna \textbf{ne} zabila edinstvennyj gol. On byl otlichnyj udar.\\
	Anna \textbf{not} scored only goal it was excellent strike\\
	\glt `Anna didn't score the only goal. It was an excellent strike.' (one goal)
	\ex \label{only-goal-multiple-ru} \gll Anna zabila \textbf{ne} edinstvennyj gol. On byl otlichnyj udar.\\
	Anna scored \textbf{not} only goal it was excellent strike\\
	\glt `Anna didn't score the only goal. It was an excellent strike.' (multiple goals)
\end{exe}

In (\ref{only-goal-multiple-ru}), the negation particle precedes the object and thus takes narrow scope, yielding the reading under which multiple goals were scored. In contrast to English, though, this reading does not result in an anti-uniqueness effect---the second sentence of (\ref{only-goal-multiple-ru}) is perfectly grammatical in Russian, with the pronoun \textit{on} `it' referring to the goal that Anna scored.

The surprising lack of an anti-uniqueness effect in (\ref{only-goal-multiple-ru}) may be accounted for by assuming that, unlike in English, the negation particle is actually inside the DP. In other words, the string \textit{ne edinstvennyj gol} is a DP headed by a null definite determiner.

\section{Indeterminate \textit{edinstvennyj}}
\citet{cb2015} predict that indeterminate readings should be more widely available in article-less languages than in English. Preliminary evidence from Russian shows that NPs beginning with \textit{edinstvennyj} resist indefinite interpretation. In (\ref{sole-director}), for instance, the preferred translation of the English sentence with the indefinite phrase \textit{a sole director} uses the regular cardinal number \textit{odin} `one' rather than \textit{edinstvennyj}, which is marginal.

\begin{exe}
	\ex \label{sole-director} \gll U etoj kompanii --- (odin/??edinstvennyj) direktor\\
	At this company {} (one/only) director\\
	\glt `This company has a sole director.'
\end{exe}

Note that \textit{odin} is morphologically related to \textit{edinstvennyj}, whose root is \textit{edin}. Similarly, \textit{one} in English may be morphologically related to \textit{only}.

In (\ref{not-a-sole}), another example of indefinite \textit{sole} in English, the translation with \textit{edinstvennyj} is outright ungrammatical. \textit{Odin} must be used.

\begin{exe}
	\ex \label{not-a-sole} \gll Ni (odin/*edinstvennyj) chelovek ne pri\v{s}yol.\\
	Not one/only person not came\\
	\glt `Not a sole person came.'
\end{exe}

% Talk about the construction `odin-edinstvennyj'

\section{Summary and further research}

\section*{Acknowledgements}
I thank Alexandr Trubetskoy for his help with the Russian data.


\bibliography{thesis}

\end{document}