\documentclass{article}

\usepackage{verbatim}
\usepackage{natbib}
\bibliographystyle{linquiry2}
\usepackage{gb4e}

% Cite an author, e.g. Davidson's (1967)
\newcommand{\citegen}[1]{\citeauthor{#1}'s~(\citeyear{#1})}

\title{DP-internal \textit{only} in English and Russian}
\author{Ian Fisher}
\date{November 21, 2018}

\begin{document}
\maketitle

\begin{abstract}
This thesis draft presents an analysis of DP-internal \textit{only} in English and Russian. I review the evidence that \citet{cb2012b, cb2015} present on ``anti-unique'' definites. Building on this empirical foundation, I argue that phrases containing DP-internal \textit{only} are not definites at all. I adopt \citeauthor{cb2015}'s analysis of DP-internal \textit{only} as an operator that presupposes existence and entails uniqueness, but I propose contrary to them that \textit{the} functions as a syntactically required but semantically vacuous determiner when it combines with \textit{only}. Since \textit{the} in such examples is not the definite article, these apparent counterexamples are compatible with Russell's and Frege's theories of definiteness. I show that this analysis allows DP-internal \textit{only} and adverbial \textit{only} to be treated as a single lexical item.
\end{abstract}

\section{Introduction \label{sec:intro}}
A definite description is conventionally assumed to existence and uniqueness. That is, an expression of the form \textit{the P} may only be used when the predicate \textit{P} has a unique satisfier in the domain of discourse. If \textit{P} has no satisfiers or more than one, then the use of \textit{the P} is not licensed.

There exist apparent exceptions to this generalization, however. In particular, there is a class of ``anti-unique'' definites, discussed by \citet{cb2015}, which lack an existence entailment. (\ref{scott}) is an example. The use of the definite description \textit{the only author of Waverley} ought to require that there be a single individual with the ``only author of \textit{Waverley}'' property. But this is plainly not the case; in fact, the primary purpose of uttering (\ref{scott}) is to assert that there is no such individual.

\begin{exe}
	\ex \label{scott} Scott is not the only author of \textit{Waverley}.
\end{exe}

Observe that it is crucially the words \textit{only} and \textit{not} which instigate the anti-uniqueness effect. When either is removed, as in (\ref{scott-wo-only}) and (\ref{scott-wo-not}), then the definite description retains a unique referent.

\begin{exe}
	\ex \label{scott-wo-only} Scott is not the author of \textit{Waverley}.
	\ex \label{scott-wo-not} Scott is the only author of \textit{Waverley}.
\end{exe}

The contrast between (\ref{scott}) and (\ref{scott-wo-only}-\ref{scott-wo-not}) illustrates the surprising interaction of DP-internal \textit{only}, negation and definiteness. The main thrust of this paper is to explore this interaction and its ramifications in Russian, and to offer an alternative proposal to \citegen{cb2015} of the semantics of anti-uniqueness effects.

The paper is organized as follows. Sections \ref{sec:anti-uniqueness-english} and \ref{sec:anti-uniqueness-russian} review the distribution of anti-uniqueness effects in English and Russian, respectively. Section \ref{sec:indeterminate-only} discusses the possibility of DP-internal \textit{only} phrases with indeterminate readings. Section \ref{sec:existence-and-uniqueness} argues for the existence presupposition and uniqueness assertion of DP-internal \textit{only}. Section \ref{sec:edinstvennyj} reviews the syntax and semantics of DP-internal \textit{only} in Russian. Section \ref{sec:two-onlys} compares the DP-internal and DP-external uses of the word \textit{word}. Section \ref{sec:conclusion} concludes the paper.
\section{Anti-uniqueness effects in English \label{sec:anti-uniqueness-english}}
Anti-uniqueness effects occur in English when a definite description containing DP-internal \textit{only} or another exclusive adjective (such as \textit{sole} or \textit{single}) is in the scope of negation.

In the following discussion several terms from \citet{cb2015} are used: ``weak uniqueness,'' a weakened form of uniqueness that allows non-existence (i.e. if uniqueness means $|P| = 1$ then weak uniqueness means $|P| \le 1$); ``definiteness'' in the restricted sense of a morphological feature, most commonly the definite article, which signals a weak uniqueness presupposition in English; ``determinacy,'' the property of denoting an individual (i.e., having type $e$); and ``anti-uniqueness effect,'' to refer to the phenomenon of a definite description lacking an existence\footnote{Despite the name, it is in fact the existence implication that anti-unique definites lack \citep[p. 385]{cb2015}.} entailment, i.e. a definite that is not determinate.

\subsection{Predicative anti-uniqueness in English}
(\ref{scott}), repeated below, is the canonical example of anti-uniqueness in the predicate position.

\begin{exe}
	\exr{scott} Scott is not the only author of \textit{Waverley}.
\end{exe}

Suppose a fictional context where the novel \textit{Waverley} was written by a committee comprising Scott, Macfarlane and Campbell, in which case (\ref{scott}) would be a true utterance.

In such a context, the phrases \textit{author of Waverley} and \textit{only author of Waverley} denote the sets (\ref{def-author}) and (\ref{def-only-author}). In other words, \textit{author of Waverley} is the set of individuals who are authors of \textit{Waverley}, and \textit{only author of Waverley} is the set of individuals who are ``only authors'' of \textit{Waverley}. Since in this context there are by definition no ``only authors'', this set is empty.

\begin{exe}
	\ex \label{def-author} $\textit{author of Waverley} = \lbrace Scott, Macfarlane, Campbell \rbrace$
	\ex \label{def-only-author} $\textit{only author of Waverley} = \emptyset$
\end{exe}

But if the set denoted by \textit{only author of Waverley} is empty, then what could \textit{the only author of Waverley} denote? In conventional theories of definiteness, the use of a phrase \textit{the P} is only possible if there is a single, unique \textit{P} \citep{horn-abbott-2012}, or in the terminology of sets, if \textit{P} denotes a singleton set.

In particular, \citet{frege} and \citet{strawson50} have uniqueness as a presupposition for definite descriptions, \citet{russell} has it as an assertion, and \citet{horn-abbott-2012} have it as an implicature.

All these accounts share uniqueness as an essential part of the meaning of the definite article. What is so surprising about the ``anti-uniqueness effects'' that (\ref{scott}) evinces is the absence, and in fact denial, of uniqueness in a definite description.

\subsection{Argumental anti-uniqueness in English}
As \citet{strawson50} observed, definite descriptions may be used predicatively, as in (\ref{scott}) above and (\ref{napoleon}) below, where \textit{the greatest French soldier} is not used to mention an individual but to attribute a property to Napoleon.

\begin{exe}
	\ex \label{napoleon} Napoleon was the greatest French soldier.
\end{exe}

Anti-uniqueness effects are not unique to predicative definites, however. They are also possible with definites in argument positions, as in (\ref{only-goal}). (\ref{only-goal}) entails that multiple goals were scored, including one by Anna, so just as in (\ref{scott}) the description \textit{the only goal} cannot have a referent.

\begin{exe}
	\ex \label{only-goal} Anna didn't score the only goal.
\end{exe}

There is a simple diagnostic is available to test the determinacy of an argumental definite (and thus presence or absence of an anti-uniqueness effect). If the definite description fails to denote an individual, then it is not able to serve as the antecedent of a pronoun in a subsequent sentence. The contrast between (\ref{the-goal}) and (\ref{only-goal-multiple}) therefore testifies to the presence of an anti-uniqueness effect in (\ref{only-goal}).

\begin{exe}
	\ex \label{the-goal} Anna didn't score [ the goal ]_1. It_1 was an excellent strike.
	\ex \label{only-goal-multiple} Anna didn't score [ the only goal ]_1. \#It_1 was an excellent strike.
\end{exe}

Note that (\ref{only-goal}) is actually ambiguous between a reading where one goal was scored by someone other than Anna and a reading where multiple goals were scored, including one by Anna. Under the one-goal reading, \textit{the only goal} does have a referent and therefore should be able to be a pronoun's antecedent, while it should not be under the multiple-goals reading. (\ref{only-goal-ambig-one}) and (\ref{only-goal-ambig-multiple}) tease apart the two readings with additional context and validate the predictions.

\begin{exe}
	\ex \label{only-goal-ambig-one} One-goal reading: Anna didn't score [ the only goal ]$_1$, Maria did. It$_1$ was an excellent strike.
	\ex \label{only-goal-ambig-multiple} Multiple-goals reading: Anna didn't score [ the only goal ]$_1$, Maria also scored. \#It$_1$ was an excellent strike.
\end{exe}

The two readings correspond to two different scopes of negation. In the one-goal reading, negation takes wide scope over the entire VP \textit{score the only goal}. In the multiple-goals reading, negation takes narrow scope over the argument \textit{the only goal}, yielding an anti-uniqueness effect. The narrow scope of negation in (\ref{only-goal-ambig-multiple}) is evident in the fact that the verb \textit{score} is not interpreted as negated---Anna did score something, on this reading.

Only verbs of creation can induce argumental anti-uniqueness effects in English. When \textit{see} is substituted for \textit{score}, as in (\ref{see-only-goal}), then the referential use of \textit{the only goal} is forced; (\ref{see-only-goal}) can only mean that there was a single goal.\footnote{The multiple-goals reading is still possible with heavy emphasis on \textit{only}, as in: \begin{exe} \ex Anna didn't see the ONLY goal. There was more than one. \end{exe} This example will be discussed further in section \ref{sec:existence-and-uniqueness}.}

\begin{exe}
	\ex \label{see-only-goal} Anna didn't see the only goal.
\end{exe}
\section{The semantics of \textit{edinstvennyj} \label{sec:edinstvennyj}}
The preceding discussion has more or less assumed that \textit{edinstvennyj} in Russian corresponds to DP-internal \textit{only} in English. In this section, I wish to flesh out that assumption by comparing \textit{edinstvennyj} with a number of exclusive nominal modifiers in English. There are several with a similar meaning, in addition to \textit{only}: \textit{sole}, \textit{single}, and \textit{one}. (\ref{osso}), for instance, has the same meaning regardless of the choice of adjective.

\begin{exe}
	\ex \label{osso} The (only/sole/single/one) person to come was Ahmed.
\end{exe}

However, the various words evince distinct semantic and syntactic properties in other circumstances. \citet{cb2012b} catalog the inventory of properties thoroughly. Their conclusions are summarized in the table below.\\

\begin{tabular}{ l | l l l l l }
	& indeterminacy & superlative & plural & NPIs & DP negation \\
	\hline
	\textit{only} & no & no & yes & yes & no \\
	\textit{sole} & yes & yes & yes & yes & yes \\
	\textit{single} & yes & yes & no & marginal & yes \\
	\textit{one} & no & marginal & no & yes & no \\
\end{tabular}

% Awkward way to force more space below table.
\ \\

The judgments in the table are shown by the sentences (\ref{osso-indef})-(\ref{osso-dp-neg}). In (\ref{osso-indef}), the exclusive adjectives are placed in a DP headed by an indefinite article. In (\ref{osso-super}), they combine with a superlative NP. In (\ref{osso-pl}), they combine with a plural NP. In (\ref{osso-npi}), they license or fail to license negative polarity items. In (\ref{osso-dp-neg}), they undergo DP negation.

\begin{exe}
	\ex \label{osso-indef} This company has a(n) (*only/sole/single/*one) director.
	\ex \label{osso-super} The oil spill was the (*only/sole/single/?one) worst environmental disaster in the state's history.
	\ex \label{osso-pl} They are the (only/sole/*single/*one) people we can trust.
	\ex \label{osso-npi} The (only/sole/??single/one) pick-up truck he ever owned was a Chevrolet.
	\ex \label{osso-dp-neg} Not a(n) (*only/sole/single/*one) person came.
\end{exe}

The remainder of the section will test the properties of \textit{edinstvennyj} against this matrix.

\subsection{Licensing of negative polarity items}
Both adverbial and DP-internal \textit{only} license negative polarity items in English:

\begin{exe}
	\ex *(Only) Khalid \textbf{ever} goes to the movies.
	\ex The *(only) poem I \textbf{ever} read in high school was ``The Raven.''
\end{exe}

DP-internal \textit{only} cannot license NPIs outside of its DP:

\begin{exe}
	\ex[*] {The only team that I had heard of \textbf{ever} won the World Cup.}
\end{exe}

(\ref{libo-vs-nibud}) shows \textit{edinstvennyj} licensing two kinds of NPIs, \textit{kto-libo} \citep{pereltsvaig06} and \textit{kto-nibud'} \citep{russneg}.

\begin{exe}
	\ex \label{libo-vs-nibud} \gll Ivan vzjal edinstvennuju knigu, kotoruju (kto-libo / ?kto-nibud') xotel.\\
	Ivan took only book which anybody {} anybody wanted\\
	\glt `Ivan took the only book that anybody wanted.'
\end{exe}

(\ref{libo-vs-nibud2}), the same sentence without \textit{edinstvennyj}, is ungrammatical, proving that it is \textit{edinstvennyj} that licenses the NPIs.

\begin{exe}
	\ex[*] { \label{libo-vs-nibud2} \gll Ivan vzjal knigu, kotoruju (kto-libo / kto-nibud') xotel.\\
	Ivan took book which anybody {} anybody wanted\\
	\glt Intended: `Ivan took the book that somebody wanted.'
	}
\end{exe}

\textit{Edinstvennyj} cannot license \textit{kto-nibud'} outside of its DP:

\begin{exe}
	\ex \label{nibud-out-of-dp} \gll Edinstvennyj u\v{c}itel' vybral (kogo-to / *kogo-nibud').\\
	only teacher picked someone {} anyone\\
	\glt `The only teacher picked someone.'
\end{exe}

In (\ref{nibud-out-of-dp}), \textit{kto-to}	is a positive polarity item that is subject to Principle C of the Binding Theory \citep{russneg}.\footnote{The morphemes \textit{to}, \textit{nibud'} and \textit{libo} are affixes or clitics which may attach to a number of pronouns, including \textit{\v{c}to} `what' (\textit{\v{c}to-to}, \textit{\v{c}to-nibud'}, \textit{\v{c}to-libo}) and \textit{kto} `who' (\textit{kto-to}, \textit{kto-nibud'}, \textit{kto-libo}). Only the underlying pronoun takes case endings, hence forms like \textit{kogo-to}, the genitive and accusative declension of \textit{kto-to}.}

The NPI status of \textit{nibud'}-items is a little unclear, as they are licensed, at least in some circumstances, in non-monotonic contexts like simple affirmative sentences:\footnote{Russian speakers may find (\ref{nibud-decl}) more acceptable with additional context, such as \textit{Boris v laboratorii} `Boris is in the laboratory.'}

\begin{exe}
	\ex \label{nibud-decl} \gll Boris \v{c}to-nibud' delaet.\\
	Boris anything does\\
	\glt `Boris is doing something.'
\end{exe}

Nevertheless, the ability of \textit{edinstvennyj} to license \textit{libo}-items, clear examples of Russian NPIs, is sufficient to confirm its status as an NPI licenser.

\subsection{Other properties of \textit{edinstvennyj}}
The remaining properties of \textit{edinstvennyj} to be pinned down are its ability to combine with superlative NPs and plural NPs, and to undergo DP negation. (\ref{not-a-sole}) already showed that DP negation is impossible for \textit{edinstvennyj}. (\ref{plural-edin}) shows that \textit{edinstvennyj} may modify a plural NP.

\begin{exe}
	\ex \label{plural-edin} \gll Oni --- edinstvennye ljudi, kotorym ja doverjaju.\\
	they {} only people which I trust\\
	\glt `They are the only people that I trust.'
\end{exe}

The ungrammaticality of (\ref{super-edin}) demonstrates that \textit{edinstvennyj} cannot modify a superlative NP.

\begin{exe}
	\ex[*] { \label{super-edin} \gll
		\`{E}to edinstvennyj samyj vysokiy neboskr\"{e}b v \v{C}ikago.\\
		this only most tall skyscraper in Chicago\\
		\glt Intended: `It is the single tallest skyscraper in Chicago.'
	}
\end{exe}

Section \ref{sec:indeterminate-only} showed that \textit{edinstvennyj} permits indeterminate readings.

\subsection{Summary}
The relevant semantic and syntactic properties of \textit{edinstvennyj} and its potential counterparts are thus:\\

\begin{tabular}{ l | l l l l l }
	& indeterminacy & superlative & plural & NPIs & DP negation \\
	\hline
	\textit{edinstvennyj} & yes & no & yes & yes & no \\
	\textit{only} & no & no & yes & yes & no \\
	\textit{sole} & yes & yes & yes & yes & yes \\
	\textit{single} & yes & yes & no & marginal & yes \\
	\textit{one} & no & marginal & no & yes & no \\
\end{tabular}

% Awkward way to force more space below table.
\ \\

The properties of \textit{edinstvennyj} are most similar to those of DP-internal \textit{only}, with the exception of the greater possibility of an indeterminate reading of \textit{edinstvennyj} compared with English \textit{only}. The similarity of \textit{only} and \textit{edinstvennyj} in a range of circumstances supports my comparison between the two words with regards to anti-uniqueness effects.
\section{Anti-uniqueness effects in Russian \label{sec:anti-uniqueness-russian}}
Anti-uniqueness effects are possible in both predicative and argumental definites in Russian, when DP-internal \textit{only} (pronounced as \textit{edinstvennyj} in Russian) is in the scope of negation, just as in English. While there is clear evidence of argumental anti-uniqueness, Russian speakers have mixed judgments on sentences with \textit{edinstvennyj} in the object position. In the idiolects of some speakers, anti-uniqueness effects do not arise in sentences whose English counterparts do have them. Additionally, argumental anti-uniqueness effects are possible even with verbs of non-creation in Russian.

\subsection{Predicative anti-uniqueness in Russian}
Definite descriptions can be predicative in Russian, as (\ref{pred-def}) and (\ref{pred-def2}) show.

\begin{exe}
	\ex \label{pred-def} \gll Dmitrij --- vysokij, simpati\v{c}nyj, i samyj umnyj student vo vs\"{e}m universitete.\\
	Dmitri {} tall cute and most smart student in all university\\
	\glt `Dmitri is tall, cute and the smartest student in the whole university.'
	
	\ex[*] { \label{pred-def2}  \gll Dmitrij --- vysokij, simpati\v{c}nyj, i Boris.\\
	Dmitri {} tall cute and Boris\\
	\glt `Dmitri is tall, cute and Boris.'
	}
\end{exe}

Assuming that (a) adjectives are of type $\langle e, t \rangle$ and proper names are of type $e$ in Russian, and (b) conjuncts must have the same semantic type, then the ability of a definite description in (\ref{pred-def}) to conjoin with an adjective, and the inability of a proper name to do so in (\ref{pred-def2}), indicates that definites can have type $\langle e, t \rangle$. The equivalent sentences without conjunction are grammatical (see (\ref{dmitri-boris}) and (\ref{dmitri-boris2}), so it is crucially the adjectival conjunction that renders the sentence with \textit{Boris} ungrammatical.\footnote{That is, the sentence is ungrammatical on an equative reading where \textit{Boris} has type $e$. It does have a grammatical reading where \textit{Boris} is taken to denote a set of properties associated with ``Boris-ness'', similarly to \textit{such-a} phrases in English: \begin{exe} \ex He's such a Boris.\end{exe} Russian speakers may find the reading more accessible with a name like \textit{Putin} that is more easily given a property reading. Since this property-denoting interpretation of \textit{Boris} plausibly has type $\langle e, t \rangle$, its grammaticality supports my assertion.} Note that the superlative \textit{samyj umnyj student} `smartest student' was used to force the determinate interpretation, since superlatives are inherently determinate but regular bare nominals can be either determinate or indeterminate.

\begin{exe}
	\ex \label{dmitri-boris} \gll Dmitrij --- samyj umnyj student vo vs\"{e}m universitete.\\
	Dmitri {} most smart student in all university\\
	\glt `Dmitri is the smartest student in the whole university.'
	
	\ex \label{dmitri-boris2} \gll Dmitrij --- Boris.\\
	Dmitri {} Boris\\
	\glt `Dmitri is Boris.'
\end{exe}

It has therefore been established that definite descriptions can be predicates in Russian. Do Russian predicative definites exhibit anti-uniqueness effects? (\ref{tolstoy}) and (\ref{tolstoy2}) indicate that they do.

\begin{exe}
	\ex \label{tolstoy} \gll Tolstoj --- edinstvennyj avtor \textit{Vojny i mira}.\\
	Tolstoy {} only author \textit{War and Peace}\\
	\glt `Tolstoy is the only author of \textit{War and Peace}.'
	
	\ex \label{tolstoy2} \gll Tolstoj ne edinstvennyj avtor \textit{Vojny i mira}.\\
	Tolstoy not only author \textit{War and Peace}\\
	\glt `Tolstoy is not the only author of \textit{War and Peace}.'
\end{exe}

(\ref{tolstoy2}) has the same meaning as its English translation. It presupposes that Tolstoy is an author of \textit{War and Peace} (since (\ref{tolstoy}) still entails that he is an author of \textit{War and Peace}) and asserts that one or more others are also authors. Therefore, \textit{edinstvennyj avtor Vojny i mira} `the only author of \textit{War and Peace}' fails to refer to an individual, just as in English, and must be an anti-unique definite.

\subsection{Argumental anti-uniqueness in Russian}
The basic pattern of argumental anti-uniqueness in English holds in Russian as well: nominals in the scope of negation modified by \textit{edinstvennyj} cannot be interpreted as determinate.

The basic paradigm is established by (\ref{anna1})-(\ref{anna3}). The second sentence in each numbered example, separated for clarity, should be read as a continuation of the first. (\ref{anna1}) and (\ref{anna2}) both entail a single lecture, so \textit{edinstvennuju lekciju} `the only lecture' is determinate and the pronoun \textit{ona} `it' in the continuation sentence can refer to it.

(\ref{anna3}) entails multiple lectures, so \textit{edinstvennuju lekciju} should be indeterminate and, as predicted, the pronoun in the continuation sentence cannot refer to it.

\begin{exe}
	\ex \label{anna1} \begin{xlist}
		\ex \gll Anna posetila edinstvennuju lekciju, kotoruju pro\v{c}ital Xomskij, kogda byl v na\v{s}em universitete.\\
		Anna attended only lecture which gave Chomsky when was at our university\\
		\glt `Anna went to the only lecture that Chomsky gave at our university.'
		
		\ex \gll Ona byla o lingvistike.\\
		it was about linguistics\\
		\glt `It was about linguistics.'
	\end{xlist}
	
	\ex \label{anna2} \begin{xlist}
		\ex \gll Anna ne posetila edinstvennuju lekciju, kotoruju pro\v{c}ital Xomskij, kogda byl v na\v{s}em universitete.\\
		Anna not attended only lecture which gave Chomsky when was at our university\\
		\glt `Anna didn't go to the only lecture that Chomsky gave at our university.'
		
		\ex \gll Ona byla o lingvistike.\\
		it was about linguistics\\
		\glt `It was about linguistics.'
	\end{xlist}
	
	\ex \label{anna3} \begin{xlist} 
		\ex \gll Anna posetila ne edinstvennuju lekciju, kotoruju pro\v{c}ital Xomskij, kogda byl v na\v{s}em universitete.\\
		Anna attended not only lecture which gave Chomsky when was at our university\\
		\glt `Anna went to one of the lectures that Chomsky gave at our university.'
		
		\ex \gll \# Ona byla o lingvistike.\\
		{} it was about linguistics\\
		\glt `It was about linguistics.'
	\end{xlist}
\end{exe}

Note that Russian uses two different sentences to express the same two readings as sentences that are ambiguous in English, like (\ref{only-goal}) (though see below for an exception).

The Russian data is more complicated than the tidy contrast in (\ref{anna1})-(\ref{anna3}) would suggest, however. For one thing, some speakers \textit{do} permit the continuation in (\ref{anna3}). This phenomenon will be addressed in section \ref{sec:no-anti-unique}. For another, some speakers find (\ref{anna2}) marginal and prefer to state it as in (\ref{anna2.1}). In examples to follow, I will use the variants corresponding to (\ref{anna2}) and not (\ref{anna2.1}) for consistency, as all speakers agreed that one or the other was grammatical.

\begin{exe}
	\ex \label{anna2.1} \gll Ne Anna posetila edinstvennuju lekciju, kotoruju pro\v{c}ital Xomskij, kogda byl v na\v{s}em universitete.\\
	Not Anna attended only lecture which gave Chomsky when was at our university\\
	\glt `It wasn't Anna that went to the only lecture that Chomsky gave at our university.'
\end{exe}

Some speakers interpret (\ref{anna3}) as entailing that Anna attended more than one of Chomsky's lectures, while others interpret it to mean that she attended only one (but that Chomsky gave more than one).

The word \textit{edinstvennyj} requires greater context in Russian than DP-internal \textit{only} does in English. Most speakers judge (\ref{bad-edinstvennyj}) to be bad, for example. For that reason, the \textit{edinstvennyj} phrases throughout this section bear relative clauses or prepositional phrases to give the necessary context for their use to be grammatical in Russian.

\begin{exe}
	\ex[*] { \label{bad-edinstvennyj} \gll Anna zabil edinstvennyj gol.\\
	Anna scored only gol.\\
	\glt Intended: `Anna scored the only goal.'
	}
\end{exe}

Despite these caveats, the generalization remains that argumental anti-uniqueness occurs in Russian where \citeauthor{cb2015} predict it would. This generalization is borne out across a range of different verbs and sentences:

\begin{exe}
	\ex \begin{xlist}
		\ex \gll Marija napisala edinstvennuoe xoro\v{s}oe so\v{c}inenie vo vs\"{e}m klasse.\\
		Maria wrote only good essay in entire class\\
		\glt `Maria wrote the only good essay in the entire class.'
		
		\ex \gll Ono bylo o russkoj literature.\\
		it was about Russian literature\\
		\glt `It was about Russian literature.'
	\end{xlist}
		
	\ex \begin{xlist}
		\ex \gll Marija ne napisala edinstvennuoe xoro\v{s}oe so\v{c}inenie vo vs\"{e}m klasse.\\
		Maria not wrote only good essay in entire class\\
		\glt `Maria didn't write the only good essay in the entire class.'
		
		\ex \gll Ono bylo o russkoj literature.\\
		it was about Russian literature\\
		\glt `It was about Russian literature.'
	\end{xlist}

	\ex \label{maria3} \begin{xlist}
		\ex \gll Marija napisala ne edinstvennuoe xoro\v{s}oe so\v{c}inenie vo vs\"{e}m klasse.\\
		Maria wrote not only good essay in entire class\\
		\glt `Maria wrote one of the good essays in the class.'
	
		\ex \gll \# Ono bylo o russkoj literature.\\
		{} it was about Russian literature\\
		\glt `It was about Russian literature.'
	\end{xlist}

	\ex \begin{xlist}
		\ex \gll Boris proizn\"{e}s edinstvennuju xoro\v{s}uju re\v{c}' na svad'be.\\
		Boris gave only good speech at wedding\\
		\glt `Boris gave the only good speech at the wedding.'
		
		\ex \gll Ono bylo o molodo\v{z}\"{e}nax.\\
		it was about newlyweds\\
		\glt `It was about the newlyweds.'
	\end{xlist}

	\ex \begin{xlist}
		\ex \gll Boris ne proizn\"{e}s edinstvennuju xoro\v{s}uju re\v{c}' na svad'be.\\
		Boris not gave only good speech at wedding\\
		\glt `Boris didn't give the only good speech at the wedding.'
		
		\ex \gll Ono bylo o molodo\v{z}\"{e}nax.\\
		it was about newlyweds\\
		\glt `It was about the newlyweds.'
	\end{xlist}
	
	\ex \label{boris3} \begin{xlist}
		\ex \gll Boris proizn\"{e}s ne edinstvennuju xoro\v{s}uju re\v{c}' na svad'be.\\
		Boris gave not only good speech at wedding\\
		\glt `Boris gave one of the good speeches at the wedding.'
		
		\ex \gll \# Ono bylo o molodo\v{z}\"{e}nax.\\
		{} it was about newlyweds\\
		\glt `It was about the newlyweds.'
	\end{xlist}
\end{exe}

Interestingly, Russian also evinces anti-uniqueness effects with verbs of non-creation like \textit{uvidet'} `to see' and \textit{probovat'} `to taste':

\begin{exe}
	\ex \begin{xlist}
		\ex \gll Lena uvidela edinstvennogo krokodila, kotoryj byl v zooparke.\\
		Lena saw only crocodile which was at zoo\\
		\glt `Lena saw the only crocodile at the zoo.'
		
		\ex \gll On byl dlinoy tri metra.\\
		it was lengthwise three meters\\
		\glt `It was three meters long.'
	\end{xlist}

	\ex \begin{xlist}
		\ex \gll Lena ne uvidela edinstvennogo krokodila, kotoryj byl v zooparke.\\
		Lena not saw only crocodile which was at zoo\\
		\glt `Lena didn't see the only crocodile at the zoo.'
		
		\ex \gll On byl dlinoy tri metra.\\
		it was lengthwise three meters\\
		\glt `It was three meters long.'
	\end{xlist}
	
	\ex \label{lena3} \begin{xlist}
		\ex \gll Lena uvidela ne edinstvennogo krokodila, kotoryj byl v zooparke.\\
		Lena saw not only crocodile which was at zoo\\
		\glt `Lena saw one of the crocodiles at the zoo.'
		
		\ex \gll \# On byl dlinoy tri metra.\\
		{} it was lengthwise three meters\\
		\glt `It was three meters long.'
	\end{xlist}

	\ex \begin{xlist}
		\ex \gll Ol'ga poprobovala edinstvennyj tort, kotoryj byl na ve\v{c}erinke.\\
		Olga tasted only cake which was at party\\
		\glt `Olga tasted the only cake at the party.'

		\ex \gll On byl \v{s}okoladnyj.\\
		it was chocolate\\
		\glt `It was chocolate.'
	\end{xlist}

	\ex \begin{xlist}
		\ex \gll Ol'ga ne poprobovala edinstvennyj tort, kotoryj byl na ve\v{c}erinke.\\
		Olga not tasted only cake which was at party\\
		\glt `Olga didn't taste the only cake at the party.'
		
		\ex \gll On byl \v{s}okoladnyj.\\
		it was chocolate\\
		\glt `It was chocolate.'
	\end{xlist}
	
	\ex \label{olga3} \begin{xlist}
		\ex \gll Ol'ga poprobovala ne edinstvennyj tort, kotoryj byl na ve\v{c}erinke.\\
		Olga tasted not only cake which was at party\\
		\glt `Olga tasted one of the cakes at the party.'
		
		\ex \gll \# On byl \v{s}okoladnyj.\\
		{} it was chocolate\\
		\glt `It was chocolate.'
	\end{xlist}
\end{exe}

In summary, argumental anti-uniqueness effects arise in not only where they do in English---when DP-internal \textit{only} is in the scope of negation with a verb of creation---but with verbs of non-creation as well.

\subsection{No anti-uniqueness in some idiolects \label{sec:no-anti-unique}}
The data presented above will be analyzed in greater depth in section \ref{sec:existence-and-uniqueness}. But a set of limited but systematic counterexamples should be dealt with first. As noted previously, not all Russian speakers find the continuation in (\ref{anna3}) ungrammatical. The combination in (\ref{no-anti-unique}) is grammatical for these speakers, and similarly for (\ref{maria3}), (\ref{boris3}), (\ref{lena3}) and (\ref{olga3}).

\begin{exe}
	\ex \label{no-anti-unique} \begin{xlist}
		\ex \gll Anna posetila ne edinstvennuju lekciju, kotoruju pro\v{c}ital Xomskij, kogda byl v na\v{s}em universitete.\\
		Anna attended not only lecture which gave Chomsky when was at our university\\
		\glt `Anna went to one of the lectures Chomsky gave at our university.'
		
		\ex \gll Ona byla o lingvistike.\\
		it was about linguistics\\
		\glt `It was about linguistics.'
	\end{xlist}
\end{exe}

In order to express these sentences idiomatically in English, one has to resort to using the indefinite \textit{one of the lectures} in (\ref{no-anti-unique}), but in Russian the same definite descriptions is used as in (\ref{anna1}) and (\ref{anna2}).

How can it be that (\ref{no-anti-unique}) entails a multiplicity of lectures and yet allows for a singular reference? The first step towards answering this question is to clarify the syntactic structure of \textit{ne edinstvennyj lekciju} `not (the) only lecture.' The constituent negation particle \textit{ne} must either be outside the DP, as in (\ref{external-ne}), or internal to it, as in (\ref{internal-ne}).

\begin{exe}
	\ex \label{external-ne} Anna posetila ne [_{DP} edinstvennuju lekciju].
	\ex \label{internal-ne} Anna posetila [_{DP} ne edinstvennuju lekciju].
\end{exe}

The meaning of (\ref{no-anti-unique}) indicates that \textit{edinstvennyj} is being negated, since \textit{edinstvennyj} entails singularity while the matrix sentence entails multiplicity. It is not obvious how this semantic relationship could be explained if \textit{ne} were external to the DP, because a (phonetically null) determiner would intervene between \textit{ne} and \textit{edinstvennyj} and prevent them from composing directly. On the other hand, if both words were internal to the DP, then a compositional analysis would be natural.

In that case, it must be the DP \textit{ne edinstvennuju lekciju} (rather than \textit{edinstvennuju lekciju} with \textit{ne} outside) that refers to the lecture that Anna attended, out of the multiple lectures that were given.

This account raises a serious issue. \textit{Edinstvennuju lekciju} presumably denotes a singleton set. The negation particle \textit{ne} normally functions as the set complement operation, which means that \textit{ne edinstvennuju lekciju} ought to denote the complement of a singleton set, which would have more than one element. But a set with more than one element should not be able to be interpreted as determinate.

The difficulty is not so much a matter of an inexpressive theory as a true peculiarity in the facts of the Russian language in this instance. It is quite unusual that \textit{ne edinstvennuju lekciju} `the not-only lecture' can simultaneously entail a multiplicity of lectures while denoting a single one.

I suggest two approaches: one which makes use of a selection operation on sets to pick out a singular referent, and one which postulates an alternative set akin to those induced by adverbial \textit{only}.

What would be required in the semantics is some operation to pick out one of the multiplicity of goals. This selection operation, which would extract a singular referent from a multiplicity, must be constrained by the form of the rest of the sentence: \textit{ona} in (\ref{no-anti-unique}) may not refer to just any lecture that Chomsky gave, but specifically the one that Anna attended.

A phrase yielding an antecedent other than what it strictly denotes is not unknown in natural languages: complement anaphora are just such a phenomenon. A complement anaphor is a pronoun whose antecedent is the complement of some set previously denoted \citep{nouwen03, schwarz09}. In (\ref{kennedy}), the antecedent of \textit{they} in the second sentence is the complement of \textit{few congressmen}, i.e. few congressmen admire Kennedy so the majority of them think he is incompetent. Only certain quantifiers allow complement anaphora, as (\ref{kennedy2}) shows.

\begin{exe}
	\ex \label{kennedy} Few congressmen admire Kennedy. They think he's incompetent.
	\ex \label{kennedy2} A few congressmen admire Kennedy. \#They think he's incompetent.
\end{exe}

The same pattern holds in Russian:

\begin{exe}
	\ex \gll Malo kto iz kongressmenov vosxi\v{s}\v{c}aetsja Kennedi. Oni dumajut, \v{c}to on neumelyj.\\
	Few who from congressmen admire Kennedy they think that he incompetent\\
	\glt `Few congressmen admire Kennedy. They think he's incompetent.'
	\ex \gll Neskol'ko kongressmenov vosxi\v{s}\v{c}ajutsja Kennedi. \#Oni dumajut, \v{c}to on neumelyj.\\
	{A few} congressmen admire Kennedy they think that he incompetent\\
	\glt `A few congressmen admire Kennedy They think he's incompetent.'
\end{exe}

Complement anaphora illustrate that the antecedent of a pronoun need not always be explicitly present in the semantics, as long as it can be derived from some entity that is present. Complement anaphora involve the set complement relationship; the data from Russian suggests that some individual-selection operation on sets may also be available.

The second possibility is that \textit{edinstvennyj} induces an alternative set. Alternative sets were originally proposed to model the semantics of the DP-external use of \textit{only} \citep{rooth85, rooth92}.
A sentence with \textit{only} like (\ref{adverbial-only}) involves both the presupposition that Min-ji goes to the movies and the assertion that no one else does, where ``no one else'' is constrained to a set of plausible contextual alternatives to Min-ji.

\begin{exe}
	\ex \label{adverbial-only} Only Min-ji goes to the movies.
\end{exe}

Similarly, (\ref{no-anti-unique}) involves both a single lecture that Anna attended and a set of lectures which Anna did not attend. The phrase \textit{ne edinstvennyj lekciju} would behave similarly to the \textit{only Min-ji} in (\ref{adverbial-only}), in that both simultaneously entail multiplicity (multiple lectures, multiple people whose movie-going habits are under consideration) while denoting a single one (the lecture that Anna attended, Min-ji).

Under this account, \textit{ne edinstvennuju lekciju} would denote the single lecture that Anna attended, but would also carry an entailment about the alternative set, i.e. the additional lectures she did not attend.

The two approaches I have proposed are not incompatible. It could be that the set selection operation is sensitive to the alternative set that DP-internal \textit{only} induces in this case. At any rate, I hope to have made the ability of \textit{ne edinstvennuju lekciju} in some idiolects to entail multiple lectures but denote a single one a little less puzzling.
\section{A theory of \textit{the} and \textit{only} \label{sec:my-theory}}
Two approaches are possible to account for the absence of a uniqueness entailment in (\ref{scott}) and (\ref{only-goal}): to loosen the uniqueness requirement for all definite descriptions, or to declare that the phrases with DP-internal \textit{only} are not definites at all. \citeauthor{cb2015} adopt the former approach. In this section, I will present a theory along the latter lines, in which DP-internal \textit{only} presupposes existence and asserts uniqueness, and the lexical item \textit{the} that combines with \textit{only} is a semantically vacuous determiner.

\subsection{Existence presupposition}
Since the definite article\footnote{Although later I will argue that \textit{the} in \textit{the only} is not actually the definite article, here I assume a skeptical reader who has not yet been convinced of the point.} carries an existence presupposition itself, it is difficult to tease apart whether it is \textit{only} or \textit{the} which contributes the existence presupposition in examples where both words are present. Regardless, it is clear from the shared entailment of Scott's authorship in (\ref{exist-presup1}) and its negated counterpart (\ref{exist-presup2}) that definite descriptions with DP-internal \textit{only} have an existence presupposition.

\begin{exe}
	\ex \label{exist-presup1} Scott is the only author of \textit{Waverley}.
		\begin{xlist}
			\ex Scott is an author of \textit{Waverley}.
			\ex There are no other authors of \textit{Waverley}.
		\end{xlist}
	\ex \label{exist-presup2} Scott is not the only author of \textit{Waverley}.
		\begin{xlist}
			\ex Scott is an author of \textit{Waverley}.
			\ex There are other authors of \textit{Waverley}.
		\end{xlist}
\end{exe}

To say that DP-internal \textit{only} has an existence presupposition is a bit of a stipulation as it by definition cannot be separated from the definite article. There is stronger evidence that DP-internal \textit{only} has a uniqueness assertion, and if that is the case then the definite article would no longer be compatible with DP-internal \textit{only} since its uniqueness presupposition would clash with \textit{only}'s uniqueness assertion. If the definite article is out, then \textit{only} is the only word left which could plausibly carry the existence presupposition.

\subsection{Uniqueness assertion}
DP-internal \textit{only} asserts uniqueness. This fact is demonstrated first of all by anti-uniqueness effects. Since uniqueness can be cancelled by negation in such examples, it cannot be a presupposition.

Additional evidence supports the point. In (\ref{green-roof-the}), the second speaker cannot felicitously challenge the uniqueness of \textit{house with a green roof}. In (\ref{green-roof-the-only}), the same exchange but with \textit{the only} instead of \textit{the}, the second speaker is free to challenge the uniqueness of the phrase's referent, indicating that uniqueness is at-issue and thus a semantic assertion.

\begin{exe}
	\ex \label{green-roof-the} Is it true that John lives in the house with a green roof? \\
	    - No, he lives next door. \\
	    - \#No, there are two houses with a green roof.
	\ex \label{green-roof-the-only} Is it true that John lives in the only house with a green roof? \\
	    - No, he lives next door. \\
	    - No, there are two houses with a green roof.
\end{exe}

Similar evidence comes directly from \citet{cb2015}:

% TODO: Think some more about Courtney's comments.
\begin{exe}
	\ex \#He's not the ambassador to Spain---there are two.
	\ex He's not the only ambassador to Spain---there are two.
\end{exe}

Only the uniqueness of the phrase with \textit{only} may be negated---a clear indication that \textit{the only ambassador to Spain} lacks a uniqueness presupposition.

\subsection{Contribution of the definite article}
In fact, the existence presupposition and uniqueness assertion are already presented in \citegen{cb2015} proposed logical form for \textit{only}, given below.

\begin{exe}
	\ex \textit{only}: $ \lambda P . \lambda x . [ \partial(P(x)) \land \forall y [ x \ne y \to \neg P(y) ] ] $
\end{exe}

\citeauthor{cb2015} use the partial operator $\partial$ to indicate the presupposed content. Notice that presupposing $P(x)$ amounts to presupposing existence, because the $P(a)$ that will result once \textit{only} composes with some referential entity $a$ logically entails $\exists x . P(x)$. And the second conjunct is an assertion of the uniqueness of $x$ relative to the predicate $P$, so existence and uniqueness are already built in to \citeauthor{cb2015}'s definition and redundantly encoded in their theory by the \textsc{Iota} type-shift.

Where my proposal diverges from \citegen{cb2015} is in the role of the definite article in phrases with DP-internal \textit{only}. \citeauthor{cb2015} assert that it is the regular definite article that appears in every other definite description. This is why they must substantially change its semantics to fit the evidence. In my proposal, \textit{the} in \textit{the only} is not the definite article, but a semantically vacuous determiner.

This is not merely an \textit{ad hoc} stipulation. There is independent evidence that \textit{the} may be used in non-definite contexts. For example, the phrase \textit{the unicorn} in (\ref{unicorn}) has a kind reading rather than a definite one.

\begin{exe}
	\ex \label{unicorn} The unicorn is a rare beast.
\end{exe}

Certain expressions like \textit{read the newspaper} and \textit{take the bus} also do not seem to involve proper definites, as the referents of \textit{the newspaper} and \textit{the bus} need not be familiar or unique. They are essentially identical to saying \textit{read a newspaper} and \textit{take a bus}.

The existence of other examples where the word \textit{the} does not have the semantics of the definite article supports my hypothesis that \textit{the} in \textit{the only} is likewise not the definite article.
\section{\citet{cb2015} \label{sec:coppock-beaver}}
The cornerstone of \citeauthor{cb2015}'s theory is that the definite and indefinite articles in English are identity functions, and the definite article carries an additional weak uniqueness presupposition. In the formulae below, $\partial$ stands for \citegen{beaver92} partial operator which is used to model presuppositions compositionally: $\partial(\phi)$ is true if $\phi$ is true and undefined otherwise.

\begin{exe}
	\ex $\textit{the} = \lambda P . \lambda x . [\partial(|P| \le 1) \land P(x)]$
	\ex $\textit{a(n)} = \lambda P . \lambda x . P(x)$
\end{exe}

A typical definite like \textit{the table} would be given the formula in (\ref{the-table}).

\begin{exe}
	\ex \label{the-table} $\textit{the table} = \lambda x . [ \partial(|\textsc{Table}| \le 1) \land \textsc{Table}(x) ]$
\end{exe}

A consequence of \citeauthor{cb2015}'s definition of the definite article is that definite descriptions are of type $\langle e, t \rangle$. Of course, definite descriptions commonly appear in argument positions where they are of type $e$. To allow for this, \citeauthor{cb2015} propose that the covert type shifters \textsc{Iota} and \textsc{A} from \citet{partee86} apply in English to yield determinate and indeterminate readings of DPs.

Definites are always determinate, except for \textit{only} NPs in the scope of negation, and indefinites are always indeterminate, so \citeauthor{cb2015} need an account of why \textsc{Iota} can never apply to indefinites, and \textsc{A} can only apply to definites that contain \textit{only}. They account for this with the principles of Maximize Presupposition and Type Simplicity. Informally, Maximize Presupposition states that if there are two possible words whose meanings are identical, then the one with the greater presupposition must be chosen. Per \citeauthor{cb2015}, the indefinite and definite article have the same meaning, but the definite article has an extra presupposition of weak uniqueness, so in a situation where weak uniqueness is in the common ground, the definite article must be chosen.

Type Simplicity is the preference for simpler types, all else being equal. \textsc{Iota} has type $\langle et, e \rangle$ while \textsc{A} has type $\langle et, \langle et, t \rangle \rangle$, so \textsc{Iota} would be preferred unless its licensing condition (uniqueness) is not met.

Thus the principles of Maximize Presupposition and Type Simplicity ensure that definiteness and determinacy, and indefiniteness and indeterminacy, coincide in English except in the case of \textit{only} NPs.

In summary, \citeauthor{cb2015}'s theory of definiteness has the following components:

\begin{itemize}
	\item The definite and indefinite articles are identity functions. The definite article additionally carries a weak uniqueness presupposition.
	\item Definite and indefinite are fundamentally predicative and have type $\langle e, t \rangle$ before type-shifting.
	\item DPs receive their determinacy or indeterminacy through the \textsc{Iota} and \textsc{A} covert type shifts.
\end{itemize}

The Russian data in this paper should pose no particular issues for \citeauthor{cb2015}'s analysis---its role was to establish a cross-linguistic understanding of \textit{only} NPs and definiteness. Nevertheless, \citeauthor{cb2015}'s theory leaves something to be desired. Their proposal is essentially a wholesale reimagining of the structure of definiteness in English, and a redefinition of the definite and indefinite articles that removes nearly all their semantic content. The basis for this is quite limited: all of their counterexamples crucially involve \textit{only} NPs in the scope of negation.

The reason that \citeauthor{cb2015} offer such an ambitious proposal is that their counterexamples directly contradict the Fregean and Russellians accounts of definiteness, that is, accounts of definiteness with existence and uniqueness as the meaning of the definite article. As long as one maintains that the phrases with DP-internal \textit{only} are regular definite descriptions, then indeterminate \textit{only} NPs are a flat contradiction that cannot be reconciled.

Familiarity theories of definiteness, which are prominent alternatives to Frege's and Russell's account, are equally ill-equipped to deal with definites with DP-internal \textit{only}. Familiarity theories locate the fundamental difference between indefinites and definites in novelty and familiarity: the definite article is used when the referent is familiar to both speaker and listener, and the indefinite article is used when it is familiar only to the speaker \citep{heim82}.

In the exchange in (\ref{newcastle}), the referent of \textit{the only goal} is clearly not familiar to the speaker (hence why \textit{the goal} is not licensed), but it can nonetheless be used felicitously.

\begin{exe}
	\ex \label{newcastle} - What happened in the match this morning? \\
	- Not much. Newcastle scored the *(only) goal.
\end{exe}

However, as I have shown in section \ref{sec:my-theory}, if one abandons the assumption that \textit{the only} phrases are proper definites, then no modifications to the theory of definiteness are necessary, and the semantic account of anti-uniqueness effects can be formulated in a manner that is consistent with their limited scope of application.
\section{DP-internal \textit{only} and adverbial \textit{only} \label{sec:two-onlys}}
So far I have exclusively discussed the DP-internal usage of the word \textit{only}. There is also a more common usage which I will term ``adverbial \textit{only}'' for convenience.\footnote{Although \textit{only} in fact has a different distribution than other adverbs, for example: \begin{exe} \ex (Only/??Quickly) John finished the race. \ex John finished the race (*only/quickly). \end{exe} The exact syntactic status of this usage of \textit{only} is outside the scope of this paper.} An example of the adverbial usage is (\ref{adverb-only}).

\begin{exe}
	\ex \label{adverb-only} Only Omar takes the bus.
\end{exe}

DP-internal and adverbial \textit{only} share many properties, including the basic inference pattern of existence and uniqueness and the licensing of negative polarity. The purpose of this section is to argue that my analysis of DP-internal \textit{only} does not preclude a unified account of the two \textit{only}s, though the details of what a unified account would look like are beyond the scope of this paper.

\subsection{Compositional semantics}
(\ref{only-lf}), repeated below, is the logical form of DP-internal \textit{only} proposed by \citet{cb2015}.

\begin{exe}
	\exr{only-lf} \textit{only}: $ \lambda P . \lambda x . [ \partial(P(x)) \land \forall y [ x \ne y \to \neg P(y) ] ] $
\end{exe}

The formula for DP-internal \textit{only} in (\ref{only-lf}) also adequately captures the meaning of adverbial \textit{only} in (\ref{adverb-only}). Suppose the following translations:

\begin{exe}
	\ex \textit{Omar}: $o$
	\ex \textit{takes the bus}: $\lambda x . \textsc{Take-bus}(x)$
\end{exe}

If \textit{only} composes first with \textit{takes the bus} and then with \textit{Omar}, then the logical form of the entire sentence would be (\ref{omar-lf}).

\begin{exe}
	\ex \label{omar-lf} \textit{Only Omar takes the bus}: $\partial(\textsc{Take-bus}(o)) \land \forall y [ o \ne y \to \neg \textsc{Take-bus}(y) ]$
\end{exe}

(\ref{omar-lf}) works out to presuppose that Omar takes the bus and assert that no one else does, which is the correct meaning of the sentence. This division of presupposition and assertion is the standard analysis of simple sentences with \textit{only} \citep{horn69}.

The one caveat is that syntactically it would be more plausible for \textit{only} to compose first with \textit{Omar} and then with \textit{takes the bus}. Of course this could be accomplished by switching the order of the parameters $P$ and $x$ in (\ref{only-lf}), but then it would be difficult to derive the correct semantics for the DP-internal usage of \textit{only}.

There may be technical issues (though seemingly tractable ones) in realizing a unified account of the two \textit{only}s in the framework of compositional semantics. Nonetheless, the difficulty resides mostly in the syntax, and that the semantics of the two words are near-identical.

% TODO: This claim ^ is a little strong.

\subsection{Alternative sets and association with focus}
Alternative sets are a fundamental part of the semantics of adverbial \textit{only} \citep{rooth85, rooth92}. We have seen in section \ref{sec:my-theory} that alternative sets may help explain how a negated \textit{edinstvennyj} NP may still allow reference in some idiolects of Russian.

More generally, \textit{only} NPs with restricting clauses or PPs involve alternative sets. (\ref{rebecca-only-bus}), for instance, carries a stronger implication of the existence of alternative buses to other cities than (\ref{rebecca-bus}) without \textit{only} does.

\begin{exe}
	\ex \label{rebecca-only-bus} Rebecca took the only bus to Wichita.
	\ex \label{rebecca-bus} Rebecca took the bus to Wichita.
\end{exe}

When \textit{only} NPs lack a restricting clause, like in (\ref{raj-only-cookie}), then the presence of an alternative set becomes harder to diagnose. But again, the crucial difference between (\ref{raj-only-cookie}) and (\ref{raj-cookie}), its counterpart without \textit{only}, is that (\ref{raj-only-cookie}) carries a wider entailment of uniqueness (in the sense that (\ref{raj-cookie}) but not (\ref{raj-only-cookie}) would be licensed at an event with more than one cookie), and to a certain degree implies the expectation that there would have been more than one cookie. It is with regard to these alternative, non-existent cookies that (\ref{raj-only-cookie}) expresses its meaning.

\begin{exe}
	\ex \label{raj-only-cookie} Raj ate the only cookie.
	\ex \label{raj-cookie} Raj ate the cookie.
\end{exe}

Besides inducing alternative sets, \citet{rooth85} showed that adverbial \textit{only} associates with focus. Consider (\ref{focused-bill}) and (\ref{focused-sue}), where capitalization marks semantic focus.

\begin{exe}
	\ex \label{focused-bill} Mary only introduced BILL to Sue.
	\ex \label{focused-sue} Mary only introduced Bill to SUE.
\end{exe}

In a situation where Mary introduced Bill and Dave to Sue, and Dave to Mary, (\ref{focused-bill}) would be false but (\ref{focused-sue}) would be true. But without \textit{only}, both sentences would be true. The crux of association with focus is thus the fact that \textit{only} interacts with semantic focus in a manner that affects truth conditions.

DP-internal \textit{only} likewise associates with focus. Suppose that Mary has a red hatchback, Mike has a red pick-up truck, and Susan has a blue sedan. In that context (\ref{focused-hatchback}) would be true but (\ref{focused-red}) would be false insofar as it implies that other people also have hatchbacks. Again, removing \textit{only} renders both sentences true in the given context.

\begin{exe}
	\ex \label{focused-hatchback} Margaret is the only one with a red HATCHBACK.
	\ex \label{focused-red} Margaret is the only one with a RED hatchback.
\end{exe}

The evidence is fairly strong that DP-internal \textit{only} associates with focus and induces alternative sets, although the alternation in unrestricted \textit{only} NPs like (\ref{raj-only-cookie}) is more difficult to detect.

\subsection{Negative polarity items and downward entailment}
Section \ref{sec:edinstvennyj} showed that DP-internal \textit{only}, like adverbial \textit{only}, licenses negative polarity items. Negative polarity items are typically permitted only in downward entailment environments, which are environments that allow an inference from supersets to subsets. Sentential negation is the canonical example of a downward-entailing environment, as the entailment from the superset \textit{reptiles} in (\ref{reptiles}) to the subset \textit{snakes} in (\ref{snakes}) shows.

\begin{exe}
	\ex \label{reptiles} No reptiles give birth to live young.
	\ex \label{snakes} No snakes give birth to live young.
\end{exe}

The scope of adverbial \textit{only} is not a traditional downward entailment environment. (\ref{vegetables}) does not entail (\ref{broccoli}).

\begin{exe}
	\ex \label{vegetables} Only John eats vegetables.
	\ex \label{broccoli} Only John eats broccoli.
\end{exe}

However, a weaker version of downward entailment called Strawson entailment does apply \citep{fintel99}. Strawson entailment carries the additional requirement that a sentence's presuppositions be satisfied. In (\ref{broccoli}), if the presupposition that John eats broccoli is satisfied, then (\ref{vegetables}) does in fact entail (\ref{broccoli}).

An identical pattern emerges with DP-internal \textit{only}. (\ref{ca-30}) entails (\ref{ca-35}) so long as the existence presupposition of \textit{only} is satisfied with regard to states with more than 30 million and 35 million inhabitants. (\ref{russian-book}) entails (\ref{gogol-book}) if a single entity satisfying the description \textit{book in the library written by Gogol} exists.

\begin{exe}
	\ex \label{ca-30} The only state with more than 30 million inhabitants is California.
	\ex \label{ca-35} The only state with more than 35 million inhabitants is California.
	\ex \label{russian-book} The only book in the library written by a Russian is already checked out.
	\ex \label{gogol-book} The only book in the library written by Gogol is already checked out.
\end{exe}

Both \textit{only}s are Strawsonian downward-entailing NPI licensers.

\subsection{Cross-linguistic perspective}
DP-internal \textit{only} is translated as \textit{edinstvennyj} in Russian. Adverbial \textit{only} corresponds to a different lexical item, \textit{tol'ko}:

\begin{exe}
	\ex \gll Tol'ko studenty pri\v{s}li.\\
	only students came\\
	\glt `Only the students came.'
\end{exe}

The same lexical distinction between adverbial and DP-internal \textit{only} is made in Chinese (Shizhe Huang, p.c.), Spanish and German \citep{mcnally08}. On the other hand, French uses the same word, \textit{seul}:\footnote{I thank Ma\"{e}lys Gl\"{u}ck for this data. French also has an adverb \textit{seulement}, morphologically derived from \textit{seul}, which corresponds to some usages of adverbial \textit{only} in English.}

\begin{exe}
	\ex \gll Seuls les \'{e}tudiants sont venus.\\
	Only the students are came\\
	\glt `Only the students came.'
	\ex \gll Julia a \'{e}crit le seul bon essai.\\
	Julia has written the only good essay\\
	\glt `Julia wrote the only good essay.'
\end{exe}

The cross-linguistic evidence on the lexical identity of the two \textit{only}s is mixed and does not clearly support either a unified or a distinct account. However, the range of properties that DP-internal and adverbial \textit{only} share suggest that a unified account is feasible, and at any rate my analysis of DP-internal \textit{only} does not seem to be inconsistent with such an account.
\section{Conclusion \label{sec:conclusion}}
On the basis of evidence from English and Russian, I have proposed a theory of \textit{only} NPs in which \textit{only} presupposes existence and uniqueness and \textit{the} is a semantically vacuous determiner, rather than the definite article as \citet{cb2015} argue. My proposal has the advantage that it is compatible with Fregean theories of definiteness.

% TODO: Add something about adverbial and DP-internal only
\section*{Acknowledgements}
I am grateful to Alexandr Trubetskoy, Maxim Sonin, Sophie Chochaeva, and Ivan Tseytlin for their help with the Russian data.

\bibliography{thesis}

\end{document}