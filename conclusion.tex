\section{Conclusion \label{sec:conclusion}}
The approach of this paper has been two-pronged: to extend \citegen{cb2015} observation of the absence of existence entailments with \textit{only} NPs to Russian, and to argue for an account of them based on the unique semantics of DP-internal \textit{only} instead of broad changes to the theory of definiteness.

The Russian data showed that \textit{edinstvennyj} NPs behave identically to \textit{only} NPs in English with regard to the uniqueness presupposition, except that argumental \textit{edinstvennyj} NPs lack a uniqueness presupposition even with verbs of non-creation. Also, in some idiolects of Russian negated argumental \textit{edinstvennyj} NPs remain determinate. Further research is needed to fully understand this phenomenon.

I have shown that adopting \citeauthor{cb2015}'s theory that DP-internal \textit{only} presupposes existence and asserts uniqueness, in conjunction with my argument that \textit{the} in \textit{the only} constructions is not the definite article at all but a semantically vacuous determiner, allows for a theory of anti-uniqueness effects that preserves the traditional understanding of definites.

% TODO: Add something about adverbial and DP-internal only