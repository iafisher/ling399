\section{Conclusion \label{sec:conclusion}}
The approach of this paper has been two-pronged: to extend \citegen{cb2015} observation of anti-uniqueness effects to Russian, and to argue for an account of them based on the unique semantics of DP-internal \textit{only} instead of broad changes to the theory of definiteness.

The Russian data showed evidence of anti-uniqueness in both predicative and argumental definites, just as in English. However, in some idiolects of Russian argumental anti-uniqueness is apparently not possible, at least in the typical constructions where they arise in English. Further research is needed to fully understand the distribution of anti-uniqueness effects in Russian.

I have argued that DP-internal \textit{only} presupposes existence and asserts uniqueness, and that expressions containing it are not properly considered definite descriptions at all. This account allows for a theory of anti-uniqueness effects that preserves the traditional understanding of definites.