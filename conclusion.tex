\section{Conclusion \label{sec:conclusion}}
On the basis of evidence from English and Russian, I have proposed a theory of \textit{only} NPs in which \textit{only} presupposes existence and uniqueness and \textit{the} is a semantically vacuous determiner, rather than the definite article. This theory was able to explain the ambiguity of argumental \textit{only} NPs in English and the non-ambiguity in Russian in terms of raising. Compared to \citet{cb2015}, the theory has the advantage that it is compatible with Fregean theories of definiteness.

There are several promising avenues for future research. More work must be done on the relationship between adverbial and DP-internal \textit{only}. Section \ref{sec:two-onlys} showed that a unified account is not out of the question, but such an account has yet to be formulated in detail. Examining whether the two words are lexically identical in a greater range of languages would also be fruitful.

The evidence in this paper is restricted to English and Russian. Whether or not \textit{only} and definiteness interact unexpectedly in other languages remains to be seen. French, which like English has articles and no lexical distinction between adverbial and DP-internal \textit{only}, is a particularly good candidate for further investigation. All else being equal, one would expect to see the same absence of a uniqueness presupposition with \textit{only} NPs in French as in English.