\section{DP-internal \textit{only} in Russian \label{sec:edinstvennyj}}
The preceding discussion has more or less assumed that \textit{edinstvennyj} in Russian corresponds to DP-internal \textit{only} in English. A comparison of \textit{edinstvennyj} with a number of exclusive adjectives in English will flesh out that assumption. There are several adjectives with a similar meaning to \textit{only}, including \textit{sole}, \textit{single}, and \textit{one}. (\ref{osso}), for instance, has the same meaning regardless of the choice of adjective.

\begin{exe}
	\ex \label{osso} The (only/sole/single/one) person to come was Ahmed.
\end{exe}

However, these words evince distinct semantic and syntactic properties in other circumstances. \citet{cb2012a} catalog the inventory of properties thoroughly. Their conclusions are summarized in the table below.\\

\begin{tabular}{ l | l l l l l }
	& indefiniteness & superlatives & plurals & NPIs & DP negation \\
	\hline
	\textit{only} & no & no & yes & yes & no \\
	\textit{sole} & yes & yes & yes & yes & yes \\
	\textit{single} & yes & yes & no & marginal & yes \\
	\textit{one} & yes & marginal & no & yes & yes \\
\end{tabular}

% Awkward way to force more space below table.
\ \\

(\ref{osso-indef})-(\ref{osso-dp-neg}) make the entries in the table concrete. (\ref{osso-indef}) tests the possibility of an indefinite interpretation. (\ref{osso-super}) and (\ref{osso-pl}) test superlative and plural NPs, respectively. (\ref{osso-dp-neg}) tests the possibility of DP negation.

\begin{exe}
	\ex \label{osso-indef} This company has (*an only/a sole/a single/one) director.
	\ex \label{osso-super} The oil spill was the (*only/sole/single/?one) worst environmental disaster in the state's history.
	\ex \label{osso-pl} They are the (only/sole/*single/*one) people we can trust.
	\ex \label{osso-npi} The (only/sole/??single/one) pick-up truck he ever owned was a Chevrolet.
	\ex \label{osso-dp-neg} Not (*an only/a sole/a single/one) person came.
\end{exe}

The remainder of the section will test the properties of \textit{edinstvennyj} against this matrix.

\subsection{Indefiniteness}
Unlike in English, \textit{edinstvennyj} NPs (the Russian equivalent of \textit{only} NPs) may be indefinite, not only with constituent negation as shown in section \ref{sec:only-nps-russian}, but also in a variety of other circumstances: 

\begin{exe}
	\ex \label{sole-director} \gll U \`{e}toj kompanii --- (odin/edinstvennyj) direktor.\\
	At this company {} (one/only) director\\
	\glt `This company has a sole director.'
	
	\ex \label{single-good-essay} \gll Marija napisala edinstvennoe xoro\v{s}oe so\v{c}inenie za vsju \v{z}izn'.\\
	Maria wrote only good essay in entire life\\
	\glt `Maria wrote a single good essay in her entire life.'
	
	\ex \label{single-good-speech} \gll Boris ne proizn\"{e}s ni edinstvennoj xoro\v{s}oj re\v{c}i na svad'be.\\
	Boris not gave not only good speech at wedding\\
	\glt `Boris didn't give a single good speech at the wedding.'
\end{exe}

(\ref{sole-director}) is an existential construction, in which \textit{edinstvennyj} is synonymous with \textit{odin} `one.' In (\ref{single-good-essay}), indefiniteness is forced by the prepositional phrase \textit{za vsju \v{z}izn'} `in her entire life' which precludes a definite reading. In (\ref{single-good-speech}), double negation with the construction \textit{ne} \ldots \textit{ni} renders the \textit{edinstvennyj} NP indefinite.

\subsection{Licensing of negative polarity items}
Both adverbial and DP-internal \textit{only} license negative polarity items in English:

\begin{exe}
	\ex *(Only) Khalid \textbf{ever} goes to the movies.
	\ex The *(only) poem I \textbf{ever} read in high school was ``The Raven.''
\end{exe}

DP-internal \textit{only} cannot license NPIs outside of its DP:

\begin{exe}
	\ex[*] {The only team that I had heard of \textbf{ever} won the World Cup.}
\end{exe}

\textit{Edinstvennyj} licenses two kinds of NPIs, \textit{kto-libo} \citep{pereltsvaig06} and \textit{kto-nibud'} \citep{russneg}:

\begin{exe}
	\ex \label{libo-vs-nibud} \gll Ivan vzjal edinstvennuju knigu, kotoruju (kto-libo / ?kto-nibud') xotel.\\
	Ivan took only book which anybody {} anybody wanted\\
	\glt `Ivan took the only book that anybody wanted.'
\end{exe}

(\ref{libo-vs-nibud2}), the same sentence except without \textit{edinstvennyj}, is ungrammatical, proving that it is \textit{edinstvennyj} that licenses the NPIs.

\begin{exe}
	\ex[*] { \label{libo-vs-nibud2} \gll Ivan vzjal knigu, kotoruju (kto-libo / kto-nibud') xotel.\\
	Ivan took book which anybody {} anybody wanted\\
	\glt Intended: `Ivan took the book that somebody wanted.'
	}
\end{exe}

\textit{Edinstvennyj} cannot license \textit{kto-nibud'} outside of its DP:

\begin{exe}
	\ex \label{nibud-out-of-dp} \gll Edinstvennyj u\v{c}itel' vybral (kogo-to / *kogo-nibud').\\
	only teacher picked someone {} anyone\\
	\glt `The only teacher picked someone.'
\end{exe}

In (\ref{nibud-out-of-dp}), \textit{kto-to}	is a positive polarity item that is subject to Principle C of the Binding Theory \citep{russneg}.\footnote{The morphemes \textit{to}, \textit{nibud'} and \textit{libo} are affixes or clitics which may attach to a number of pronouns, including \textit{\v{c}to} `what' (\textit{\v{c}to-to}, \textit{\v{c}to-nibud'}, \textit{\v{c}to-libo}) and \textit{kto} `who' (\textit{kto-to}, \textit{kto-nibud'}, \textit{kto-libo}). Only the underlying pronoun takes case endings, hence forms like \textit{kogo-to}, the genitive and accusative declension of \textit{kto-to}.}

\subsection{Other properties of \textit{edinstvennyj}}
The remaining properties of \textit{edinstvennyj} to be pinned down are its ability to combine with superlative NPs and plural NPs, and to undergo DP negation.

DP negation is impossible for \textit{edinstvennyj}:

\begin{exe}
	\ex \label{not-a-sole} \gll Ni (odin/*edinstvennyj) \v{c}elovek ne pri\v{s}\"{e}l.\\
	Not one/only person not came\\
	\glt `Not a sole person came.'\footnote{Russian \textit{ni} is a negative concordance particle in this case, rather than double negation.}
\end{exe}

\textit{Edinstvennyj} (in its plural form \textit{edinstvennye}) may modify a plural NP.

\begin{exe}
	\ex \label{plural-edin} \gll Oni --- edinstvennye ljudi, kotorym ja doverjaju.\\
	they {} only people which I trust\\
	\glt `They are the only people that I trust.'
\end{exe}

\textit{Edinstvennyj} cannot generally modify a superlative NP:

\begin{exe}
	\ex[*] { \label{super-edin} \gll
		\`{E}to edinstvennyj samyj vysokiy neboskr\"{e}b v \v{C}ikago.\\
		this only most tall skyscraper in Chicago\\
		\glt Intended: `It is the single tallest skyscraper in Chicago.'
	}
\end{exe}

However, there are marginal examples where \textit{edinstvennyj} can combine with a superlative, with undertones of hyperbole:

\begin{exe}
	\ex[??] {
		\gll Ona edinstvennaja samaja krasivaja \v{z}en\v{s}ina vo vs\"{e}m mire.\\
		she only most beautiful woman in entire world\\
		\glt `She is the single most beautiful woman in the whole world.'
	}
\end{exe}

\subsection{Summary}
The relevant semantic and syntactic properties of \textit{edinstvennyj} and its potential counterparts are thus:\\

\begin{tabular}{ l | l l l l l }
	& indefiniteness & superlatives & plurals & NPIs & DP negation \\
	\hline
	\textit{edinstvennyj} & yes & marginal & yes & yes & no \\
	\textit{only} & no & no & yes & yes & no \\
	\textit{sole} & yes & yes & yes & yes & yes \\
	\textit{single} & yes & yes & no & marginal & yes \\
	\textit{one} & yes & marginal & no & yes & yes \\
\end{tabular}

% Awkward way to force more space below table.
\ \\

The properties of \textit{edinstvennyj} are most similar to those of DP-internal \textit{only}, with the exception of the greater possibility of an indefinite reading of \textit{edinstvennyj} compared with English \textit{only}. The similarity of \textit{only} and \textit{edinstvennyj} in a range of circumstances supports my comparison between the two words with regards to definiteness.
