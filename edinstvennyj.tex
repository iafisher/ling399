\section{The semantics of \textit{edinstvennyj} \label{sec:edinstvennyj}}
The preceding discussion has more or less assumed that \textit{edinstvennyj} in Russian corresponds to DP-internal \textit{only} in English. In this section, I wish to flesh out that assumption by comparing \textit{edinstvennyj} with a number of exclusive nominal modifiers in English. There are several with a similar meaning, in addition to \textit{only}: \textit{sole}, \textit{single}, and \textit{one}. (\ref{osso}), for instance, has the same meaning regardless of the choice of adjective.

\begin{exe}
	\ex \label{osso} The (only/sole/single/one) person to come was Ahmed.
\end{exe}

However, the various words evince distinct semantic and syntactic properties in other circumstances. \citet{cb2012b} catalog the inventory of properties thoroughly. Their conclusions are summarized in the table below.\\

\begin{tabular}{ l | l l l l l }
	& indeterminacy & superlative & plural & NPIs & DP negation \\
	\hline
	\textit{only} & no & no & yes & yes & no \\
	\textit{sole} & yes & yes & yes & yes & yes \\
	\textit{single} & yes & yes & no & marginal & yes \\
	\textit{one} & no & marginal & no & yes & no \\
\end{tabular}

% Awkward way to force more space below table.
\ \\

The judgments in the table are shown by the sentences (\ref{osso-indef})-(\ref{osso-dp-neg}). In (\ref{osso-indef}), the exclusive adjectives are placed in a DP headed by an indefinite article. In (\ref{osso-super}), they combine with a superlative NP. In (\ref{osso-pl}), they combine with a plural NP. In (\ref{osso-npi}), they license or fail to license negative polarity items. In (\ref{osso-dp-neg}), they undergo DP negation.

\begin{exe}
	\ex \label{osso-indef} This company has a(n) (*only/sole/single/*one) director.
	\ex \label{osso-super} The oil spill was the (*only/sole/single/?one) worst environmental disaster in the state's history.
	\ex \label{osso-pl} They are the (only/sole/*single/*one) people we can trust.
	\ex \label{osso-npi} The (only/sole/??single/one) pick-up truck he ever owned was a Chevrolet.
	\ex \label{osso-dp-neg} Not a(n) (*only/sole/single/*one) person came.
\end{exe}

The remainder of the section will test the properties of \textit{edinstvennyj} against this matrix.

\subsection{Licensing of negative polarity items}
Both adverbial and DP-internal \textit{only} license negative polarity items in English:

\begin{exe}
	\ex *(Only) Khalid \textbf{ever} goes to the movies.
	\ex The *(only) poem I \textbf{ever} read in high school was ``The Raven.''
\end{exe}

DP-internal \textit{only} cannot license NPIs outside of its DP:

\begin{exe}
	\ex[*] {The only team that I had heard of \textbf{ever} won the World Cup.}
\end{exe}

(\ref{libo-vs-nibud}) shows \textit{edinstvennyj} licensing two kinds of NPIs, \textit{kto-libo} \citep{pereltsvaig06} and \textit{kto-nibud'} \citep{russneg}.

\begin{exe}
	\ex \label{libo-vs-nibud} \gll Ivan vzjal edinstvennuju knigu, kotoruju (kto-libo / ?kto-nibud') xotel.\\
	Ivan took only book which anybody {} anybody wanted\\
	\glt `Ivan took the only book that anybody wanted.'
\end{exe}

(\ref{libo-vs-nibud2}), the same sentence without \textit{edinstvennyj}, is ungrammatical, proving that it is \textit{edinstvennyj} that licenses the NPIs.

\begin{exe}
	\ex[*] { \label{libo-vs-nibud2} \gll Ivan vzjal knigu, kotoruju (kto-libo / kto-nibud') xotel.\\
	Ivan took book which anybody {} anybody wanted\\
	\glt Intended: `Ivan took the book that somebody wanted.'
	}
\end{exe}

\textit{Edinstvennyj} cannot license \textit{kto-nibud'} outside of its DP:

\begin{exe}
	\ex \label{nibud-out-of-dp} \gll Edinstvennyj u\v{c}itel' vybral (kogo-to / *kogo-nibud').\\
	only teacher picked someone {} anyone\\
	\glt `The only teacher picked someone.'
\end{exe}

In (\ref{nibud-out-of-dp}), \textit{kto-to}	is a positive polarity item that is subject to Principle C of the Binding Theory \citep{russneg}.\footnote{The morphemes \textit{to}, \textit{nibud'} and \textit{libo} are affixes or clitics which may attach to a number of pronouns, including \textit{\v{c}to} `what' (\textit{\v{c}to-to}, \textit{\v{c}to-nibud'}, \textit{\v{c}to-libo}) and \textit{kto} `who' (\textit{kto-to}, \textit{kto-nibud'}, \textit{kto-libo}). Only the underlying pronoun takes case endings, hence forms like \textit{kogo-to}, the genitive and accusative declension of \textit{kto-to}.}

The NPI status of \textit{nibud'}-items is a little unclear, as they are licensed, at least in some circumstances, in non-monotonic contexts like simple affirmative sentences:\footnote{Russian speakers may find (\ref{nibud-decl}) more acceptable with additional context, such as \textit{Boris v laboratorii} `Boris is in the laboratory.'}

\begin{exe}
	\ex \label{nibud-decl} \gll Boris \v{c}to-nibud' delaet.\\
	Boris anything does\\
	\glt `Boris is doing something.'
\end{exe}

Nevertheless, the ability of \textit{edinstvennyj} to license \textit{libo}-items, clear examples of Russian NPIs, is sufficient to confirm its status as an NPI licenser.

\subsection{Other properties of \textit{edinstvennyj}}
The remaining properties of \textit{edinstvennyj} to be pinned down are its ability to combine with superlative NPs and plural NPs, and to undergo DP negation. (\ref{not-a-sole}) already showed that DP negation is impossible for \textit{edinstvennyj}. (\ref{plural-edin}) shows that \textit{edinstvennyj} may modify a plural NP.

\begin{exe}
	\ex \label{plural-edin} \gll Oni --- edinstvennye ljudi, kotorym ja doverjaju.\\
	they {} only people which I trust\\
	\glt `They are the only people that I trust.'
\end{exe}

The ungrammaticality of (\ref{super-edin}) demonstrates that \textit{edinstvennyj} cannot modify a superlative NP.

\begin{exe}
	\ex[*] { \label{super-edin} \gll
		\`{E}to edinstvennyj samyj vysokiy neboskr\"{e}b v \v{C}ikago.\\
		this only most tall skyscraper in Chicago\\
		\glt Intended: `It is the single tallest skyscraper in Chicago.'
	}
\end{exe}

Section \ref{sec:indeterminate-only} showed that \textit{edinstvennyj} permits indeterminate readings.

\subsection{Summary}
The relevant semantic and syntactic properties of \textit{edinstvennyj} and its potential counterparts are thus:\\

\begin{tabular}{ l | l l l l l }
	& indeterminacy & superlative & plural & NPIs & DP negation \\
	\hline
	\textit{edinstvennyj} & yes & no & yes & yes & no \\
	\textit{only} & no & no & yes & yes & no \\
	\textit{sole} & yes & yes & yes & yes & yes \\
	\textit{single} & yes & yes & no & marginal & yes \\
	\textit{one} & no & marginal & no & yes & no \\
\end{tabular}

% Awkward way to force more space below table.
\ \\

The properties of \textit{edinstvennyj} are most similar to those of DP-internal \textit{only}, with the exception of the greater possibility of an indeterminate reading of \textit{edinstvennyj} compared with English \textit{only}. The similarity of \textit{only} and \textit{edinstvennyj} in a range of circumstances supports my comparison between the two words with regards to anti-uniqueness effects.