\section{Anti-uniqueness effects in Russian \label{sec:anti-unique-ru}}
Anti-uniqueness effects are possible in both predicative and argumental definites in Russian, when DP-internal \textit{only} (pronounced as \textit{edinstvennyj} in Russian) is in the scope of negation, just as in English. While there is clear evidence of argumental anti-uniqueness, Russian speakers have mixed judgments on sentences with \textit{edinstvennyj} in the object position. In the idiolects of some spekaers, anti-uniqueness effects do not arise in sentences whose English counterparts do have them. Additionally, argumental anti-uniqueness effects are possible even with verbs of non-creation in Russian.

\subsection{Predicative anti-uniqueness in Russian}
Definite descriptions can be predicative in Russian, as (\ref{pred-def}) shows.

\begin{exe}
	\ex \label{pred-def} \gll Dmitrij --- vysokij, simpati\v{c}nyj, i (samyj umnyj student vo vs\"{e}m universitete / *Boris).\\
	Dmitri {} tall cute and most smart student in all university {} Boris\\
	\glt `Dmitri is tall, cute and (the smartest student in the whole university / Boris).'
\end{exe}

Assuming that (a) adjectives are of type $\langle e, t \rangle$ and proper names are of type $e$ in Russian, and (b) conjuncts must have the same semantic type, then the ability of a definite description in (\ref{pred-def}) to conjoin with an adjective, and the inability of a proper name to do so, indicates that definites can have type $\langle e, t \rangle$. The equivalent sentence without conjunction is grammatical (see (\ref{dmitri-boris})), so it is crucially the adjectival conjunction that renders the sentence with \textit{Boris} ungrammatical.\footnote{That is, the sentence is ungrammatical on an equative reading where \textit{Boris} has type $e$. It does have a grammatical reading where \textit{Boris} is taken to denote a set of properties associated with ``Boris-ness'', similarly to \textit{such-a} phrases in English: \begin{exe} \ex He's such a Boris.\end{exe} Russian speakers may find the reading more accessible with a name like \textit{Putin} that is more easily given a property reading. Since this property-denoting interpretation of \textit{Boris} plausibly has type $\langle e, t \rangle$, its grammaticality supports my assertion.} Note that the superlative \textit{samyj umnyj student} `smartest student' was used to force the determinate interpretation, since superlatives inherently cannot be indeterminate but regular bare nominals can be.

\begin{exe}
	\ex \label{dmitri-boris} \gll Dmitrij --- (samyj umnyj student vo vs\"{e}m universitete / Boris).\\
	Dmitri {} most smart student in all university {} Boris\\
	\glt `Dmitri is (the smartest student in the whole university / Boris).'
\end{exe}

It has therefore been established that definite descriptions can be predicates in Russian. Do Russian predicative definites exhibit anti-uniqueness effects? (\ref{tolstoy}) indicates that they do.

\begin{exe}
	\ex \label{tolstoy} \gll Tolstoj ne edinstvennyj avtor \textit{Vojny i mira}\\
	Tolstoy not only author \textit{War and Peace}\\
	\glt `Tolstoy is not the only author of \textit{War and Peace}.'
\end{exe}

(\ref{tolstoy}) has the same meaning as its English translation. It presupposes that Tolstoy is an author of \textit{War and Peace} (since (\ref{tolstoy}) without negation still entails that he is an author of \textit{War and Peace}) and asserts that one or more others are also authors. Therefore, \textit{edinstvennyj avtor Vojny i mira} `the only author of \textit{War and Peace}' fails to refer to an individual, just as in English, and an anti-uniqueness effect arises.

\subsection{Argumental anti-uniqueness in Russian}
The basic pattern of argumental anti-uniqueness in English holds in Russian as well: nominals in the scope of negation modified by \textit{edinstvennyj} cannot be interpreted as determinate.

The basic paradigm is established by (\ref{anna1})-(\ref{anna3}). The second sentence in each numbered example, separated for clarity, should be read as a continuation of the first. (\ref{anna1}) and (\ref{anna2}) both entail a single lecture, so \textit{edinstvennuju lekciju} `the only lecture' is determinate and the pronoun \textit{ona} `it' in the continuation sentence can refer to it.

Note that Russian uses two different sentences to express the same two readings as sentences that are ambiguous in English, like (\ref{only-goal}) (though see below for an exception).

(\ref{anna3}) entails multiple lectures, so \textit{edinstvennuju lekciju} should be indeterminate and, as predicted, the pronoun in the continuation sentence cannot refer to it.

\begin{exe}
	\ex \label{anna1} \begin{xlist}
		\ex \gll Anna posetila edinstvennuju lekciju, kotoruju pro\v{c}ital Xomskij, kogda byl v na\v{s}em universitete.\\
		Anna attended only lecture which gave Chomsky when was at our university\\
		\glt `Anna went to the only lecture that Chomsky gave at our university.'
		
		\ex \gll Ona byla o lingvistike.\\
		it was about linguistics\\
		\glt `It was about linguistics.'
	\end{xlist}
	
	\ex \label{anna2} \begin{xlist}
		\ex \gll Anna ne posetila edinstvennuju lekciju, kotoruju pro\v{c}ital Xomskij, kogda byl v na\v{s}em universitete.\\
		Anna not attended only lecture which gave Chomsky when was at our university\\
		\glt `Anna didn't go to the only lecture that Chomsky gave at our university.'
		
		\ex \gll Ona byla o lingvistike.\\
		it was about linguistics\\
		\glt `It was about linguistics.'
	\end{xlist}
	
	\ex \label{anna3} \begin{xlist} 
		\ex \gll Anna posetila ne edinstvennuju lekciju, kotoruju pro\v{c}ital Xomskij, kogda byl v na\v{s}em universitete.\\
		Anna attended not only lecture which gave Chomsky when was at our university\\
		\glt `Anna went to one of the lectures Chomsky gave at our university.'
		
		\ex \gll \# Ona byla o lingvistike.\\
		{} it was about linguistics\\
		\glt `It was about linguistics.'
	\end{xlist}
\end{exe}

The Russian data is more complicated than the tidy contrast in (\ref{anna1})-(\ref{anna3}) would suggest, however. For one thing, some speakers \textit{do} permit the continuation in (\ref{anna3}). This phenomenon will be addressed in section \ref{sec:no-anti-unique}. For another, some speakers find (\ref{anna2}) marginal and prefer to state it as in (\ref{anna2.1}).

\begin{exe}
	\ex \label{anna2.1} \gll Ne Anna posetila edinstvennuju lekciju, kotoruju pro\v{c}ital Xomskij, kogda byl v na\v{s}em universitete.\\
	Not Anna attended only lecture which gave Chomsky when was at our university\\
	\glt `It wasn't Anna that went to the only lecture that Chomsky gave at our university.'
\end{exe}

Finally, some speakers interpret (\ref{anna3}) as entailing that Anna attended more than one of Chomsky's lectures, while others interpret it to mean that she attended only one (but that Chomsky gave more than one).

Despite these caveats, the generalization remains that argumental anti-uniqueness occurs in Russian where \citeauthor{cb2015} predict it would. This generalization is borne out across a range of different verbs and sentences:\footnote{In the examples, the variant corresponding to (\ref{anna2}) and not (\ref{anna2.1}) is used for simplicity, but all speakers found one or the other variant grammatical.}

\begin{exe}
	\ex \begin{xlist}
		\ex \gll Marija napisala edinstvennuoe xoro\v{s}oe so\v{c}inenie vo vs\"{e}m klasse.\\
		Maria wrote only good essay in entire class\\
		\glt `Maria wrote the only good essay in the entire class.'
		
		\ex \gll Ono bylo o russkoj literature.\\
		it was about Russian literature\\
		\glt `It was about Russian literature.'
	\end{xlist}
		
	\ex \begin{xlist}
		\ex \gll Marija ne napisala edinstvennuoe xoro\v{s}oe so\v{c}inenie vo vs\"{e}m klasse.\\
		Maria not wrote only good essay in entire class\\
		\glt `Maria didn't write the only good essay in the entire class.'
		
		\ex \gll Ono bylo o russkoj literature.\\
		it was about Russian literature\\
		\glt `It was about Russian literature.'
	\end{xlist}

	\ex \label{maria3} \begin{xlist}
		\ex \gll Marija napisala ne edinstvennuoe xoro\v{s}oe so\v{c}inenie vo vs\"{e}m klasse.\\
		Maria wrote not only good essay in entire class\\
		\glt `Maria wrote one of the good essays in the class.'
	
		\ex \gll \# Ono bylo o russkoj literature.\\
		{} it was about Russian literature\\
		\glt `It was about Russian literature.'
	\end{xlist}

	\ex \begin{xlist}
		\ex \gll Boris proizn\"{e}s edinstvennuju xoro\v{s}uju re\v{c}' na svad'be.\\
		Boris gave only good speech at wedding\\
		\glt `Boris gave the only good speech at the wedding.'
		
		\ex \gll Ono bylo o molodo\v{z}\"{e}nax.\\
		it was about newlyweds\\
		\glt `It was about the newlyweds.'
	\end{xlist}

	\ex \begin{xlist}
		\ex \gll Boris ne proizn\"{e}s edinstvennuju xoro\v{s}uju re\v{c}' na svad'be.\\
		Boris not gave only good speech at wedding\\
		\glt `Boris didn't give the only good speech at the wedding.'
		
		\ex \gll Ono bylo o molodo\v{z}\"{e}nax.\\
		it was about newlyweds\\
		\glt `It was about the newlyweds.'
	\end{xlist}
	
	\ex \label{boris3} \begin{xlist}
		\ex \gll Boris proizn\"{e}s ne edinstvennuju xoro\v{s}uju re\v{c}' na svad'be.\\
		Boris gave not only good speech at wedding\\
		\glt `Boris gave one of the good speeches at the wedding.'
		
		\ex \gll \# Ono bylo o molodo\v{z}\"{e}nax.\\
		{} it was about newlyweds\\
		\glt `It was about the newlyweds.'
	\end{xlist}
\end{exe}

Interestingly, Russian also evinces anti-uniqueness effects with verbs of non-creation like \textit{uvidet'} `to see' and \textit{probovat'} `to taste':

\begin{exe}
	\ex \begin{xlist}
		\ex \gll Lena uvidela edinstvennogo krokodila, kotoryj byl v zooparke.\\
		Lena saw only crocodile which was at zoo\\
		\glt `Lena saw the only crocodile at the zoo.'
		
		\ex \gll On byl dlinoy tri metra.\\
		it was lengthwise three meters\\
		\glt `It was three meters long.'
	\end{xlist}

	\ex \begin{xlist}
		\ex \gll Lena ne uvidela edinstvennogo krokodila, kotoryj byl v zooparke.\\
		Lena not saw only crocodile which was at zoo\\
		\glt `Lena didn't see the only crocodile at the zoo.'
		
		\ex \gll On byl dlinoy tri metra.\\
		it was lengthwise three meters\\
		\glt `It was three meters long.'
	\end{xlist}
	
	\ex \label{lena3} \begin{xlist}
		\ex \gll Lena uvidela ne edinstvennogo krokodila, kotoryj byl v zooparke.\\
		Lena saw not only crocodile which was at zoo\\
		\glt `Lena saw one of the crocodiles at the zoo.'
		
		\ex \gll \# On byl dlinoy tri metra.\\
		{} it was lengthwise three meters\\
		\glt `It was three meters long.'
	\end{xlist}

	\ex \begin{xlist}
		\ex \gll Ol'ga poprobovala edinstvennyj tort, kotoryj byl na ve\v{c}erinke.\\
		Olga tasted only cake which was at party\\
		\glt `Olga tasted the only cake at the party.'

		\ex \gll On byl \v{s}okoladnyj.\\
		it was chocolate\\
		\glt `It was chocolate.'
	\end{xlist}

	\ex \begin{xlist}
		\ex \gll Ol'ga ne poprobovala edinstvennyj tort, kotoryj byl na ve\v{c}erinke.\\
		Olga not tasted only cake which was at party\\
		\glt `Olga didn't taste the only cake at the party.'
		
		\ex \gll On byl \v{s}okoladnyj.\\
		it was chocolate\\
		\glt `It was chocolate.'
	\end{xlist}
	
	\ex \label{olga3} \begin{xlist}
		\ex \gll Ol'ga poprobovala ne edinstvennyj tort, kotoryj byl na ve\v{c}erinke.\\
		Olga tasted not only cake which was at party\\
		\glt `Olga tasted one of the cakes at the party.'
		
		\ex \gll \# On byl \v{s}okoladnyj.\\
		{} it was chocolate\\
		\glt `It was chocolate.'
	\end{xlist}
\end{exe}



\subsection{No anti-uniqueness in some idiolects \label{sec:no-anti-unique}}
The data presented above will be analyzed in more depth in section \ref{sec:exist-unique}. But a pattern of limited but systematic counterexamples should be dealt with first. As noted previously, not all Russian speakers evince anti-uniqueness in (\ref{anna3}). That is to say, the combination in (\ref{no-anti-unique}) is grammatical for these speakers, and similarly for the sentences in (\ref{maria3}), (\ref{boris3}), (\ref{lena3}) and (\ref{olga3}).

\begin{exe}
	\ex \label{no-anti-unique} \begin{xlist}
		\ex \gll Anna posetila ne edinstvennuju lekciju, kotoruju pro\v{c}ital Xomskij, kogda byl v na\v{s}em universitete.\\
		Anna attended not only lecture which gave Chomsky when was at our university\\
		\glt `Anna went to one of the lectures Chomsky gave at our university.'
		
		\ex \gll Ona byla o lingvistike.\\
		it was about linguistics\\
		\glt `It was about linguistics.'
	\end{xlist}
\end{exe}

In order to express these sentences idiomatically in English, one has to resort to using the indefinite \textit{one of the lectures} in (\ref{no-anti-unique}), but in Russian there is no hint that it is anything other than the regular definite description that is seen in (\ref{anna1}) and (\ref{anna2}).

How can it be that (\ref{no-anti-unique}) entails a multiplicity of lectures and yet allows for a singular reference? The first step towards answering this question is to clarify the syntactic composition of \textit{ne edinstvennyj lekciju} `not (the) only lecture.' The constituent negation particle \textit{ne} must either be outside the DP, as in (\ref{external-ne}), or internal to it, as in (\ref{internal-ne}).

\begin{exe}
	\ex \label{external-ne} Anna posetila ne [_{DP} edinstvennyj lekciju].
	\ex \label{internal-ne} Anna posetila [_{DP} ne edinstvennyj lekciju].
\end{exe}

The meaning of (\ref{no-anti-unique}) indicates that \textit{edinstvennyj} is being negated, since \textit{edinstvennyj} entails singularity while the matrix sentence entails multiplicity. It is not obvious how this semantic relationship and the ungrammaticality of (\ref{ne-no-edin}) could be explained if \textit{ne} where external to the DP. It would appear that the syntactic and semantic properties of \textit{ne edinstvennyj gol} are most easily explained if \textit{ne} and \textit{edinstvennyj} compose directly, implying that \textit{ne} must be internal to the DP since otherwise the phonologically null head of DP would intervene between the two words.

In that case, it must be the DP \textit{ne edinstvennyj lekciju} that refers to the lecture that Anna attended, out of the multiple lectures that were given.

This account raises a serious issue. \textit{Edinstvennyj lekciju} presumably denotes a singleton set. The negation particle \textit{ne} normally functions as the set complement operation, which means that \textit{ne edinstvennyj lekciju} ought to denote the complement of a singleton set, which should have more than one element. But a set with more than one element should not be able to be interpreted as determinate.

The difficulty is not so much a matter of an inexpressive theory as a true peculiarity in the facts of the Russian language in this instance. It is quite unusual that \textit{ne edinstvennyj lekciju} `the not-only lecture' can simultaneously entail a multiplicity of lectures while denoting a single one.

I suggest two approaches: one which makes use of a selection operation on sets to pick out a singular referent, and one which postulates an alternative set akin to those induced by adverbial \textit{only}.

What would be required in the semantics to model this peculiarity is some operation to pick out one of the multiplicity of goals. This selection operation, which would extract a singular referent from a multiplicity, must be constrained by the form of the rest of the sentence: \textit{ona} in (\ref{no-anti-unique}) may not refer to just any lecture that Chomsky gave, but specifically the one that Anna attended.

The idea of a phrase yielding an antecedent other than the set it denotes is not unprecedented in the literature: complement anaphora are just such a phenomenon. A complement anaphor is a pronoun that has as its antecedent the complement of some set previously denoted \citep{nouwen03, schwarz09}. In (\ref{kennedy}), the antecedent of \textit{they} in the second sentence is the complement of \textit{few congressmen}, i.e. few congressmen admire Kennedy so the majority of them think he is incompetent.

\begin{exe}
	\ex \label{kennedy} Few congressmen admire Kennedy. They think he's incompetent.
\end{exe}

Not all quantifiers allow complement anaphora, though:

\begin{exe}
	\ex A few congressmen admire Kennedy. \#They think he's incompetent.
\end{exe}

The same paradigm occurs in Russian:

\begin{exe}
	\ex \gll Malo kto iz kongressmenov vosxi\v{s}\v{c}aetsja Kennedi. Oni dumajut, \v{c}to on neumelyj.\\
	Few who from congressmen admire Kennedy they think that he incompetent\\
	\glt `Few congressmen admire Kennedy. They think he's incompetent.'
	\ex \gll Neskol'ko kongressmenov vosxi\v{s}\v{c}ajutsja Kennedi. \#Oni dumajut, \v{c}to on neumelyj.\\
	{A few} congressmen admire Kennedy they think that he incompetent\\
	\glt `A few congressmen admire Kennedy They think he's incompetent.'
\end{exe}

Complement anaphora illustrate that the antecedent of a pronoun need not always be explicitly present in the semantics, as long as it can be derived from some entity that is present. Complement anaphora involve the set complement relationship; the data from Russian suggests that some individual-selection operation on sets may also be available.

The second possibility is that \textit{edinstvennyj} induces an alternative set, in the same way that adverbial \textit{only} is traditionally understood to \citep{rooth85, rooth92}. A sentence with adverbial \textit{only} like (\ref{adverbial-only}) involves both the presupposition that Min-ji goes to the movies and the assertion that no one else does, where ``no one else'' is constrained to a set of plausible contextual alternatives to Min-ji.

\begin{exe}
	\ex \label{adverbial-only} Only Min-ji goes to the movies.
\end{exe}

Similarly, (\ref{no-anti-unique}) involves both a single lecture that Anna attended and a set of lectures which Anna did not attend. The phrase \textit{ne edinstvennyj lekciju} would behave similarly to the \textit{only Min-ji} in (\ref{adverbial-only}), in that both simultaneously entail multiplicity (multiple lectures, multiple people whose movie-going habits are under consideration) while denoting a single one (the lecture that Anna attended, Min-ji). This idea is supported by the fact that the continuation in (\ref{adverbial-only-cont}) is perfectly grammatical.

% TODO: This is weak

\begin{exe}
	\ex \label{adverbial-only-cont} She loves horror films.
\end{exe}

The two approaches I have proposed are not incompatible. It could be that the set selection operation is sensitive to the alternative set that DP-internal \textit{only} induces in this case. At any rate, I hope to have made the ability of \textit{ne edinstvennyj gol} to imply multiple goals but denote a single one a little less puzzling.