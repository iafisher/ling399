\section{Existence and uniqueness entailments \label{sec:existence-and-uniqueness}}
% TODO: Integrate this

Two approaches are possible to account for the absence of a uniqueness entailment in (\ref{scott}) and (\ref{only-goal}): to loosen the uniqueness requirement for all definite descriptions, or to declare that the phrases with DP-internal \textit{only} are not definites at all. \citeauthor{cb2015} adopt the former approach. In section \ref{sec:existence-and-uniqueness} I will present a theory along the latter lines.

The cornerstone of \citeauthor{cb2015}'s theory is that the definite and indefinite articles in English are identity functions, and the definite article carries an additional weak uniqueness presupposition. In the formulae below, $\partial$ stands for \citegen{beaver92} partial operator which is used to model presuppositions compositionally: $\partial(\phi)$ is true if $\phi$ is true and undefined otherwise.

\begin{exe}
	\ex $\textit{the} = \lambda P . \lambda x . [\partial(|P| \le 1) \land P(x)]$
	\ex $\textit{a(n)} = \lambda P . \lambda x . P(x)$
\end{exe}

A typical definite like \textit{the table} would be given the formula in (\ref{the-table}).

\begin{exe}
	\ex \label{the-table} $\textit{the table} = \lambda x . [ \partial(|\textsc{Table}| \le 1) \land \textsc{Table}(x) ]$
\end{exe}

A consequence of \citeauthor{cb2015}'s definition of the definite article is that definite descriptions are of type $\langle e, t \rangle$. Of course, definite descriptions commonly appear in argument positions where they must have type $e$. To allow for this, \citeauthor{cb2015} propose that the covert type shifters \textsc{Iota} and \textsc{A} from \citet{partee86} apply in English to yield determinate and indeterminate readings of DPs.

Definites are always determinate, except in the case of anti-uniqueness effects, and indefinites are always indeterminate, so \citeauthor{cb2015} need an account of why \textsc{Iota} can never apply to indefinites, and \textsc{A} can only apply to anti-unique definites. They account for this with the principles of Maximize Presupposition and Type Simplicity. Informally, Maximize Presupposition states that if there are two possible words whose meanings are identical, then the one with the greater presupposition must be chosen. Per \citeauthor{cb2015}, the indefinite and definite article have the same meaning, but the definite article has an extra presupposition of weak uniqueness, so in a situation where weak uniqueness is in the common ground, the definite article must be chosen.

Type Simplicity is the preference for simpler types, all else being equal. \textsc{Iota} has type $\langle et, e \rangle$ while \textsc{A} has type $\langle et, \langle et, t \rangle \rangle$, so \textsc{Iota} would be preferred unless its licensing condition (uniqueness) is not met.

Thus the principles of Maximize Presupposition and Type Simplicity ensure that definiteness and determinacy, and indefiniteness and indeterminacy, coincide in English except in the case of anti-uniqueness effects.

In summary, \citeauthor{cb2015}'s theory of definiteness has the following components:

\begin{itemize}
	\item The definite and indefinite articles are identity functions. The definite article additionally carries a weak uniqueness presupposition.
	\item Definite and indefinite are fundamentally predicative and have type $\langle e, t \rangle$ before type-shifting.
	\item DPs receive their determinacy or indeterminacy through the \textsc{Iota} and \textsc{A} covert type shifts.
\end{itemize}

The Russian data in this paper should pose no particular issues for \citeauthor{cb2015}'s analysis---its role was to establish a cross-linguistic understanding of anti-uniqueness effects. Nevertheless, \citeauthor{cb2015}'s theory leaves something to be desired. Their proposal is essentially a wholesale reimagining of the structure of definiteness in English, and a redefinition of the definite and indefinite articles that remove nearly all their semantic content. The basis for this is quite limited: all of their counterexamples crucially involve DP-internal \textit{only} or other exclusive adjectives in the scope of negation.

The reason that \citeauthor{cb2015} offer such an ambitious proposal is that their counterexamples directly contradict the Fregean and Russellians accounts of definiteness, that is, accounts of definiteness with existence and uniqueness as the meaning of the definite article. As long as one maintains that the phrases with DP-internal \textit{only} are regular definite descriptions, then anti-uniqueness effects are a flat contradiction that cannot be reconciled.

Familiarity theories of definiteness, which are prominent alternatives to Frege's and Russell's account, are equally ill-equipped to deal with definites with DP-internal \textit{only}. Familiarity theories locate the fundamental difference between indefinites and definites in novelty and familiarity: the definite article is used when the referent is familiar to both speaker and listener, and the indefinite article is used when it is familiar only to the speaker \citep{heim82}.

In the exchange in (\ref{newcastle}), the referent of \textit{the only goal} is clearly not familiar to the speaker (hence why \textit{the goal} is not licensed), but it can nonetheless be used felicitously.

\begin{exe}
	\ex \label{newcastle} - What happened in the match this morning? \\
	    - Not much. Newcastle scored the *(only) goal.
\end{exe}

So if neither the Fregean-Russellian nor familiarity theories of definiteness can handle DP-internal \textit{only} and its anti-uniqueness effects, then the only option besides what \citeauthor{cb2015} propose is to argue that phrases with DP-internal \textit{only} are not definites at all.

In fact, there is good evidence that DP-internal \textit{only} itself presupposes existence and asserts uniqueness, and thus in some sense fills the same role that \textit{the} does in definite descriptions. If \textit{only} itself is the locus of existence and uniqueness, then it is not too farfetched to imagine that \textit{the} in \textit{the only} phrases is not the definite article at all, but a semantically vacuous determiner that is present solely for syntactic reasons.

In this section, I argue for the existence presupposition and uniqueness assertion of DP-internal \textit{only} in turn, and suggest the role that \textit{the} plays in such examples.

\subsection{Existence presupposition}
Since the definite article\footnote{Although later I will argue that \textit{the} in \textit{the only} is not actually the definite article, here I assume a skeptical reader who has not yet been convinced of the point.} carries an existence presupposition itself, it is difficult to tease apart whether it is \textit{only} or \textit{the} which contributes the existence presupposition in examples where both words are present. Regardless, it is clear from the shared entailment of Scott's authorship in (\ref{exist-presup1}) and its negated counterpart (\ref{exist-presup2}) that definite descriptions with DP-internal \textit{only} have an existence presupposition.

\begin{exe}
	\ex \label{exist-presup1} Scott is the only author of \textit{Waverley}.
		\begin{xlist}
			\ex Scott is an author of \textit{Waverley}.
			\ex There are no other authors of \textit{Waverley}.
		\end{xlist}
	\ex \label{exist-presup2} Scott is not the only author of \textit{Waverley}.
		\begin{xlist}
			\ex Scott is an author of \textit{Waverley}.
			\ex There are other authors of \textit{Waverley}.
		\end{xlist}
\end{exe}

To say that DP-internal \textit{only} has an existence presupposition is a bit of a stipulation as it by definition cannot be separated from the definite article. There is stronger evidence that DP-internal \textit{only} has a uniqueness assertion, and if that is the case then the definite article would no longer be compatible with DP-internal \textit{only} since its uniqueness presupposition would clash with \textit{only}'s uniqueness assertion. If the definite article is out, then \textit{only} is the only word left which could plausibly carry the existence presupposition.

\subsection{Uniqueness assertion}
DP-internal \textit{only} asserts uniqueness. This fact is demonstrated first of all by anti-uniqueness effects. Since uniqueness can be cancelled by negation in such examples, it cannot be a presupposition.

Additional evidence supports the point. In (\ref{green-roof-the}), the second speaker cannot felicitously challenge the uniqueness of \textit{house with a green roof}. In (\ref{green-roof-the-only}), the same exchange but with \textit{the only} instead of \textit{the}, the second speaker is free to challenge the uniqueness of the phrase's referent, indicating that uniqueness is at-issue and thus a semantic assertion.

\begin{exe}
	\ex \label{green-roof-the} Is it true that John lives in the house with a green roof? \\
	    - No, he lives next door. \\
	    - \#No, there are two houses with a green roof.
	\ex \label{green-roof-the-only} Is it true that John lives in the only house with a green roof? \\
	    - No, he lives next door. \\
	    - No, there are two houses with a green roof.
\end{exe}

Similar evidence comes directly from \citet{cb2015}:

\begin{exe}
	\ex \#He's not the ambassador to Spain---there are two.
	\ex He's not the only ambassador to Spain---there are two.
\end{exe}

Only the uniqueness of the phrase with \textit{only} may be negated---a clear indication that \textit{the only ambassador to Spain} lacks a uniqueness presupposition.

\subsection{Contribution of the definite article}
In fact, the existence presupposition and uniqueness assertion are already presented in \citegen{cb2015} proposed logical form for \textit{only}, given below.

\begin{exe}
	\ex \textit{only}: $ \lambda P . \lambda x . [ \partial(P(x)) \land \forall y [ x \ne y \to \neg P(y) ] ] $
\end{exe}

\citeauthor{cb2015} use the partial operator $\partial$ to indicate the presupposed content. Notice that presupposing $P(x)$ amounts to presupposing existence, because the $P(a)$ that will result once \textit{only} composes with some referential entity $a$ logically entails $\exists x . P(x)$. And the second conjunct is an assertion of the uniqueness of $x$ relative to the predicate $P$, so existence and uniqueness are already built in to \citeauthor{cb2015}'s definition and redundantly encoded in their theory by the \textsc{Iota} type-shift.

Where my proposal diverges from \citegen{cb2015} is in the role of the definite article in phrases with DP-internal \textit{only}. \citeauthor{cb2015} assert that it is the regular definite article that appears in every other definite description. This is why they must substantially change its semantics to fit the evidence. In my proposal, \textit{the} in \textit{the only} is not the definite article, but a semantically vacuous determiner.

This is not merely an \textit{ad hoc} stipulation. There is independent evidence that \textit{the} may be used in non-definite contexts. For example, the phrase \textit{the unicorn} in (\ref{unicorn}) has a kind reading rather than a definite one.

\begin{exe}
	\ex \label{unicorn} The unicorn is a rare beast.
\end{exe}

Certain expressions like \textit{read the newspaper} and \textit{take the bus} also do not seem to involve proper definites, as the referents of \textit{the newspaper} and \textit{the bus} need not be familiar or unique. They are essentially identical to saying \textit{read a newspaper} and \textit{take a bus}.

If \textit{the} does not always correspond to the definite article in English, then there is little reason to assert that it is indeed the definite article in examples with DP-internal \textit{only} which plainly lack the uniqueness presupposition that characterizes proper definites.