\section{Existence and uniqueness entailments \label{sec:existence-and-uniqueness}}
The Russian data in this paper should pose no particular issues for \citeauthor{cb2015}'s analysis---its role was to establish a cross-linguistic understanding of anti-uniqueness effects. Nevertheless, \citeauthor{cb2015}'s theory leaves something to be desired. Essentially they propose a wholesale reimagining of the structure of definiteness in English, and a redefinition of the definite and indefinite articles that remove nearly all their semantic content. The basis for this large-scale operation is quite limited: all of their counterexamples crucially involve DP-internal \textit{only} or other exclusive adjectives in the scope of negation.

The reason that \citeauthor{cb2015} offer such an ambitious proposal is that their counterexamples directly contradict their the Fregean and Russellians accounts of definiteness. As long as one maintains that the phrases with DP-internal \textit{only} are regular definite descriptions, there can be no reconciliation.

As a brief aside, it is worth considering an alternative to the Fregean and Russellian accounts of definites which locates the fundamental difference between indefinites and definites in familiarity: the definite article is used when the reference is familiar to both speaker and listener, and the indefinite article is used when it is familiar only to the speaker \citep{heim82}.

The familiarity theory of definites is equally ill-equipped to deal with definites with DP-internal \textit{only}. In the discourse in (\ref{newcastle}), for instance, the referent of \textit{the only goal} is clearly not familiar to the speaker (hence why \textit{the goal} is not licensed), but it can nonetheless be used felicitously.

\begin{exe}
	\ex \label{newcastle} - What was the score in this morning's match? \\
	    - Newcastle scored the *(only) goal.
\end{exe}

So if neither the Fregean-Russellian nor familiarity theories of definiteness can handle DP-internal \textit{only} and its anti-uniqueness effects, then the only option besides what \citeauthor{cb2015} argue for is to argue that 

In fact, there is good evidence that DP-internal \textit{only} itself presupposes existence and asserts uniqueness, and thus in some sense fills the role that \textit{the} is assumed to in definite descriptions. If \textit{only} itself is the locus of existence and uniqueness, then it is not too farfetched to imagine that \textit{the} in \textit{the only} phrases is not the definite article at all, but a semantically vacuous determiner that is present solely for syntactic reasons.

In this section, I argue for the existence presupposition and uniqueness assertion of DP-internal \textit{only} in turn, and suggest the role that \textit{the} plays in such examples.

\subsection{Existence presupposition}
Since the definite article\footnote{Although later I will argue that \textit{the} in \textit{the only} is not the definite article, here I assume a skeptical reader who has not yet been convinced of the point.} carries an existence presupposition itself, it is difficult to tease apart whether it is truly \textit{only} or in fact \textit{the} which contributes the existence presupposition. Regardless, it is clear from the shared entailment of Scott's authorship in (\ref{exist-presup1}) and its negated counterpart (\ref{exist-presup2}) that definite descriptions with DP-internal \textit{only} have an existence presupposition.

\begin{exe}
	\ex \label{exist-presup1} Scott is the only author of \textit{Waverley}.
		\begin{xlist}
			\ex Scott is an author of \textit{Waverley}.
			\ex There are no other authors of \textit{Waverley}.
		\end{xlist}
	\ex \label{exist-presup2} Scott is not the only author of \textit{Waverley}.
		\begin{xlist}
			\ex Scott is an author of \textit{Waverley}.
			\ex There are other authors of \textit{Waverley}.
		\end{xlist}
\end{exe}

To say that DP-internal \textit{only} has an existence presupposition is something of a stipulation as it by definition cannot be separated from the definite article. There is stronger evidence that DP-internal \textit{only} has a uniqueness assertion.

\subsection{Uniqueness assertion}
DP-internal \textit{only} asserts uniqueness. This fact is supported first of all by anti-uniqueness effects. Since uniqueness can be cancelled by negation in such examples, it cannot be a presupposition.

Additional evidence establishes a strong contrast with the definite article. In (\ref{green-roof-the}), the second speaker cannot felicitously challenge the uniqueness of \textit{house with a green roof}. In (\ref{green-roof-the-only}), the same discourse but with \textit{the only} instead of \textit{the}, the second speaker is welcome to offer a challenge to the uniqueness of the phrase's referent, indicating that uniqueness is at-issue and thus a semantic assertion.

\begin{exe}
	\ex \label{green-roof-the} Is is true that John lives in the house with a green roof? \\
	    - No, he lives next door. \\
	    - \#No, there are two houses with a green roof.
	\ex \label{green-roof-the-only} Is is true that John lives in the only house with a green roof? \\
	    - No, he lives next door. \\
	    - No, there are two houses with a green roof.
\end{exe}

Similar evidence comes directly from \citet{cb2015}:

\begin{exe}
	\ex \#He's not the ambassador to Spain---there are two.
	\ex He's not the only ambassador to Spain---there are two.
\end{exe}

Only the uniqueness of the phrase with \textit{only} may be negated.

\subsection{Contribution of the definite article}
In fact, the existence presupposition and uniqueness assertion are already presented in \citegen{cb2015} proposed logical form for \textit{only}, given below.

\begin{exe}
	\ex \textit{only}: $ \lambda P . \lambda x . [ \partial(P(x)) \land \forall y [ x \ne y \to \neg P(y) ] ] $
\end{exe}

\citeauthor{cb2015} use the partial operator $\partial$ to indicate the presupposed content. Notice that presupposing $P(x)$ amounts to presupposing existence, and that the second conjunct is an assertion of the uniqueness of $x$ relative to the predicate $P$.

Where this proposal differs from \citegen{cb2015} is in the role of the definite article in phrases with DP-internal \textit{only}. \citeauthor{cb2015} assert that it is the regular definite article that appears in every other definite description. This is why they must substantially change its semantics to fit the evidence. In my proposal, \textit{the} in \textit{the only} is not the definite article, but a semantically vacuous determiner.

This is not merely an \textit{ad hoc} stipulation. There is independent evidence that \textit{the} may be used in non-definite contexts. For example, the phrase \textit{the unicorn} in (\ref{unicorn}) has a kind reading rather than a definite one.

\begin{exe}
	\ex \label{unicorn} The unicorn is a rare beast.
\end{exe}

Certain expressions like \textit{read the newspaper} and \textit{take the bus} also do not seem to involve proper definites, as the referents of \textit{the newspaper} and \textit{the bus} need not be familiar or unique. They are essentially identical to saying \textit{read a newspaper} and \textit{take a bus}.

If \textit{the} does not always correspond to the definite article in English, then there is little reason to assert that it is indeed the definite article in examples with DP-internal \textit{only} which plainly lack the uniqueness presupposition that characterizes proper definites.